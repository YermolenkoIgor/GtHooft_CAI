\documentclass[main.tex]{subfiles}
\begin{document}

\section{Больше проблем}\label{ch8}

Могут потребоваться годы, десятилетия, возможно столетия, чтобы прийти к всеобъемлющей теории квантовой гравитации в сочетании с теорией квантовой материи, которая будет тщательно разработанным расширением Стандартной модели. Только тогда у нас будет возможность выделить то, что является основой для онтологической теории (beable операторы); а шаблоны в этой теории будут спроецированы на beable-базис. Только тогда мы можем сказать, действительно ли работает ИКА. Вполне возможно, однако, что мы сможем использовать ИКА в качестве руководства для достижения таких теорий. Это было основной мотивацией для написания этой книги.

\subsection{Каким будет КА для СМ?}\label{ch8.1}

Осталось множество проблем. Первая из них была встречена автором сразу: как убедить большинство исследователей, которые работают в этой области на протяжении многих десятилетий, что высказанные здесь взгляды вполне могут быть в основном правильными. В частности, мы приводим наиболее важные выводы:

\begin{itemize}
\item Есть только одна, по сути классическая, дискретная вселенная, а не бесконечность различных вселенных, как об этом говорится в интерпретации многих миров, независимо от того, руководствуется ли она пилотной волной, которая активна во всех этих мирах.
\item Вероятности Борна верны в точности, не требуя каких-либо исправлений в любой форме, поскольку они были помещены в векторы состояний (шаблоны) для описания распределения вероятностей начальных состояний. 
\item «Коллапс волновой функции» происходит автоматически, не требуя каких-либо поправок к уравнению Шредингера, таких как нелинейности. Это потому, что вселенная находится в одном онтологическом состоянии от Большого взрыва и далее, тогда как конечный результат эксперимента также всегда будет единичным. Конечное состояние никогда не может быть в суперпозиции, такой как живая кошка, наложенная (superimposed) на мертвую кошку.
\item Теория, лежащая в основе, вполне может быть локальной, но преобразование классических уравнений в гораздо более эффективные квантовые уравнения связано с некоторой степенью нелокальности, которая не оставляет следа в физических уравнениях, кроме хорошо известных, очевидных «квантовых чудес».
\item Следует найти больше моделей, в которых трансформация может быть детально проработана; это может привести к следующему поколению стандартных моделей с «ограничениями клеточных автоматов», которые могут быть проверены экспериментально.
\end{itemize}

Проблема, как систематически организовывать такие поиски, очень сложна. Предположительно, должна использоваться процедура, соответствующая второму квантованию, но пока не ясно, как это сделать для фермионных полей. Проблема в том, что мы также должны заменить уравнение Дирака для первых квантованных частиц детерминированным. Это может быть сделано для свободных, безмассовых частиц, что является не плохим началом, но также недостаточно хорошим для продолжения. Затем у нас есть несколько элементарных представлений о бозонных силовых полях, а также о внушительном струноподобном возбуждении (более подробно это описано во второй части), но опять же, пока не известно, как объединить их в нечто, что может конкурировать с известной стандартной моделью сегодня.

\subsection{Проблема иерархии}\label{ch8.2}

Существует более глубокая причина, по которой любую детальную теорию, включающую масштаб Планка, квантовую механику и относительность, может быть чрезвычайно трудно сформулировать. Это эмпирический факт, что существует два или более радикально различных масштаба, имеющих особое значение: масштаб Планка и масштаб(ы) наиболее значимых частиц в системе. Одна из удивительных закономерностей в нашем мире состоит в том, что эти различные масштабы разнятся на много порядков друг от друга. Масштаб Планка составляет $\approx 10^{19}$ ГэВ, ядерный масштаб $\approx 1$ ГэВ, в то время как есть также электроны и нейтрино при $ 10^{-11}$ ГэВ.

Теория первоначальных различий в масштабах, которые необходимы для функционирования вселенной так, как она функционирует, до сих пор совершенно неясна. Можно добавить, что нам доступно очень мало экспериментов, которые достигают точности лучше, чем 1 к $10^{11}$, не говоря уже о $10^{19}$, так что сомнительно, что какой-либо из фундаментальных принципов, относящихся к одной шкале, все еще действителен в другой. Нет недостатка в спекулятивных идеях, объясняющих происхождение этих чисел. Самое простое наблюдение, которое можно сделать, состоит в том, что математика с довольно низким уровнем развития может легко генерировать числа таких величин, но заставить их возникать естественным образом в фундаментальных теориях Природы совсем не легко.

Большинство теорий физики масштаба Планка, таких как теория суперструн и петлевая квантовая гравитация, не упоминают о происхождении больших чисел, тогда как, как мы полагаем, хорошие теории должны это делать.\footnote{Важным исключением является теория антропного принципа, состоящая в том, что некоторые числа очень велики или очень малы, просто потому, что это единственные значения, которые могут дать планеты с цивилизованными существами на них. Эта идея существует уже некоторое время, но, по понятным причинам, она не собирает большого количества последователей.} В дискретных клеточных автоматах, безусловно, можно заметить, что, если решетки в пространстве и времени играют какую-либо роль, то могут получится штуки, происходящие только когда решетка точно регулярна - имеет нулевую кривизну. Кривизна нашей вселенной чрезвычайно мала, она контролируется числами, даже более экстремальными, чем упомянутые шкалы: космологическая постоянная описывается безразмерным числом порядка $10^{-122}$. Это может означать, что мы действительно имеем дело с регулярной решеткой, но она должна приспосабливаться к редким решеточным дефектам.

В общем, универсальная теория должна объяснить возникновение очень редких событий, таких как массовые члены, вызывающие перемежаемость в фермионах, таких как электроны. Мы действительно считаем, что модели клеточных автоматов находятся в хорошем положении, чтобы допускать особые события, которые очень редки, но слишком рано пытаться понять их.

Короче говоря, наиболее естественный способ включения иерархий шкал в нашу теорию не ясен.

\end{document}