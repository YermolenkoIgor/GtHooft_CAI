\documentclass[main.tex]{subfiles}
\begin{document}

\section{Квантовая гравитация}\label{ch6}

Планковский масштаб упоминался много раз. Это шкала времени, длин, масс и энергий, где все три великие физические теории играют одинаково значимые роли: специальная теория относительности (где важна скорость света $c$), квантовая механика (с постоянной Планка $\hbar$) и теория Ньютоновской гравитации (с постоянной Ньютона $G$).

\begin{equation}\label{6.1}
	\begin{aligned} c &=299792458 \,\mathrm{m} / \mathrm{s} \\ \hbar &=1.05457 \times 10^{-34} \,\mathrm{kg}\, \mathrm{m}^{2} / \mathrm{s} \\ G &=6.674 \times 10^{-11}\, \mathrm{m}^{3} \mathrm{kg}^{-1} \mathrm{s}^{-2} \end{aligned}
\end{equation}
\begin{equation}\label{6.2}
	\begin{array}{ll}{\text { the Planck length, }} & {L_{\mathrm{Pl}}=\sqrt{\frac{G \hbar}{c^{3}}}=1.616 \times 10^{-35} \mathrm{m}} \\ {\text { the Planck time, }} & {T_{\mathrm{Pl}}=\sqrt{\frac{G \hbar}{c^{5}}}=5.391 \times 10^{-44} \mathrm{s}} \\ {\text { the Planck mass, }} & {M_{\mathrm{Pl}}=\sqrt{\frac{c \hbar}{G}}=21.76 \mu \mathrm{g}} \\ {\text { and the Planck energy, }} & {E_{\mathrm{Pl}}=\sqrt{\frac{c^{5} \hbar}{G}}=1.956 \times 10^{9} \mathrm{J}}\end{array}
\end{equation}
В этой области физики можно ожидать, что и Специальная и Общая теория относительности и Квантовая Механика будут иметь отношение к делу, но полный синтез этих направлений еще не достигнут - фактически, наши ярые поиски  такого синтеза была одной из главных мотиваций для этой работы.

Частенько звучит мысль, что длина Планка - это наименьшая значимая шкала длины в физике, а время Планка - это наименьшая шкала времени, в которой все может происходить, но все куда интересней. Общая теория относительности, как известно, приводит к искривлению пространства и времени, поэтому, если говорить о некоторой пространственно-временной <<решетке>>, можно ожидать, что кривизна вызовет в ней дефекты. В качестве альтернативы можно было бы предположить, что решеточное поведение также может быть реализовано путем введения отсечения в локальных масштабах импульса и энергии (так называемое отсечение полосы пропускания [57]); однако с таким отсечением детерминистические модели трудно построить.

Также важно отметить, что общая теория относительности основана на локальной группе автоморфизмов. Это означает, что временные трансляции локально определены, так что можно ожидать, что гравитация будет существенной для реализации требований локальности гамильтониана. Масса, энергия и импульс являются локальными источниками гравитационных полей, но есть еще такая заковырка.

Гравитация - это дестабилизирующая сила. Заставляя массы притягивать друг друга, она порождает большие массы и, следовательно, увеличивается притяжение. Это может привести к гравитационной имплозии. Напротив, электрические и магнитные заряды действуют отталкивающе (если они имеют одинаковые знаки), что делает электромагнетизм намного более устойчивым, чем гравитация как система сил.

Когда происходит гравитационная имплозия, могут образоваться черные дыры. Микроскопические черные дыры должны играть существенную роль на планковских масштабах, поскольку они могут действовать как виртуальные частицы, участвуя в колебаниях вакуума. Когда кто-то пытается включить черные дыры в общую теорию, возникают трудности. Согласно стандартным расчетам, черные дыры испускают элементарные частицы, и этот эффект (эффект Хокинга [47, 48]) позволяет вычислить плотность квантовых состояний, связанных с черными дырами. Эта плотность очень велика, но поскольку черные дыры увеличиваются в размерах, число состояний растет не так быстро, как можно было бы ожидать: оно растет экспоненциально с размером поверхности, а не с инкапсулированным объемом. Квантовые состояния, которые можно ожидать в объеме черной дыры, таинственным образом исчезают.

Мы ожидаем, что все это окажет глубокое влияние на предполагаемые детерминированные модели, которые могут лежать в основе квантовой теории. Дискретность пространства и времени мы получаем бесплатно, поскольку можно  утверждать, что число квантовых состояний внутри объема V никогда не превысит количество оных на поверхности  черной дыры, занимающей тот же объем. В этом заключается эффект, называемый «голографическим принципом» [81, 117]. Локальность может получиться естественно с использованием группы автоморфизмов, как уже упоминалось. И все же искривление пространства-времени вызывает проблемы. В книге природы пока еще слишком много тайн.

\end{document}