
\documentclass[aps,%
12pt,%
final,%
oneside,
onecolumn,%
musixtex, %
superscriptaddress,%
centertags]{article} %% 



\renewcommand{\baselinestretch}{1.5}

%\usepackage[left=2.0cm,right=1.5cm,
   % top=2.5cm,bottom=2.5cm,bindingoffset=0cm]{geometry}
\usepackage[english,russian]{babel}
\usepackage[utf8]{inputenc}
\usepackage{scrextend}
\usepackage[12pt]{extsizes}
\usepackage[a4paper, left=20mm, right=20mm, top=20mm, bottom=20mm]{geometry}

%\documentclass[a4paper,fontsize=14bp]{extreport}
%всякие настройки по желанию%
\usepackage[colorlinks=true,linkcolor=black,unicode=true]{hyperref}% включить ссылки
\usepackage{euscript}
\usepackage{supertabular}
\usepackage[pdftex]{graphicx}
\usepackage{amsthm,amssymb, amsmath}
\usepackage{textcomp}
\usepackage[noend]{algorithmic}
\usepackage[ruled]{algorithm}
\usepackage{ulem} % sout
\usepackage{cite} % ссылки по Госту 
%
%\usepackage[nooneline]{caption} \captionsetup[table]{justification=raggedright} %\captionsetup[figure]{justification=centering,labelsep=endash}

\selectlanguage{russian}

%\renewcommand{\rmdefault}{ftm} % Times New Roman
%\frenchspacing
\usepackage{subfiles}


\begin{document}

\begin{titlepage} 
\begin{center}

% Title
\textbf{\Large Герард ’т Хоофт}\\[0.5cm]
\textbf{\LARGE "Интерпретация клеточных автоматов в квантовой механике"}\footnote{G. 't Hooft, The Cellular Automaton Interpretation of Quantum Mechanics,
Fundamental Theories of Physics 185, \href{https://doi.org/10.1007/978-3-319-41285-6_2}{DOI}}\\[3.0cm]


\begin{flushright} \large
\emph{Вёрстка и перевод:} \\
магистрант \\
\textsc{Ермоленко И. П.}
\end{flushright}


\vfill 

% Bottom of the page
{\large {Волгоград}} \par
{\large {2020 г.}}
\end{center} 
\end{titlepage}

\newcommand{\openxs}{\newpage Данная глава распространяется в соответствии с условиями международной лицензии Creative Commons Attribution 4.0 International License (http://creativecommons.org/licenses/by/4.0/), которая разрешает использование, копирование, адаптацию, распространение и воспроизведение на любом носителе или в любом формате, при условии, что вы укажете автора(ов) и источник, предоставите ссылку на лицензию Creative Commons и укажете все внесенные изменения.
Изображения или другие материалы третьих лиц, представленные в данной главе, включены в лицензию Creative Commons, если не указано иное; если такой материал не включен в лицензию Creative Commons и соответствующее действие не разрешено нормативным актом, пользователям необходимо будет получить разрешение владельца лицензии на копирование, адаптацию или воспроизведение материала.\newpage}



Эта книга никоим образом не предназначена для замены стандартной теории квантовой механики. Читателю, еще не досконально знакомому с основными понятиями квантовой теории, рекомендуется сначала изучить ее по классическим учебникам, и только потом приступать к чтению книги, чтобы узнать, что доктрина называемая «квантовой механикой» рассматривается как часть чудесного математического механизма, который помещает физические явления в более широкий контекст, и только потом, как теория природы. 

Нынешняя версия книги претерпела ряд изменений. Были добавлены некоторые новинки, такие как нетрадиционный взгляд на стрелу времени, а другие аргументы были дополнительно уточнены. Книга теперь разделена на две части. Часть I посвящена многим концептуальным вопросам, не требующим чрезмерных расчетов. Часть II добавляет к этому наши методы расчета, иногда возвращаясь к концептуальным вопросам. Неизбежно, текст в одной части будет идти внахлест с формулировками из другой, но это дает частям некую автономию. Эта книга не роман, который должен быть прочитан от начала до конца, а коллекция описаний и произвольностей, которые будут использоваться в качестве ссылки. Различные части могут быть прочитаны в случайном порядке. Некоторые аргументы повторяются несколько раз, но каждый раз в другом контексте.

\newpage
\tableofcontents
\newpage

\numberwithin{equation}{section}
\part{Интерпретация клеточного автомата: общая доктрина}
\subfile{Ch1}\openxs 
\subfile{Ch2}\openxs
%\subfile{Ch3}\openxs
%\subfile{Ch4}\openxs
%\subfile{Ch5}\openxs
%\subfile{Ch6}\openxs
%\subfile{Ch7}\openxs
%\subfile{Ch8}\openxs
%\subfile{Ch9}\openxs
%\subfile{Ch10}\openxs
%\part{Методика вычислений}
%\subfile{Ch11}\openxs
%\subfile{Ch12}\openxs
%\subfile{Ch13}\openxs
%\subfile{Ch14}\openxs
%\subfile{Ch15}\openxs
%\subfile{Ch16}\openxs
%\subfile{Ch17}\openxs
%\subfile{Ch18}\openxs
%\subfile{Ch19}\openxs
%\subfile{Ch20}\openxs
%\subfile{Ch21}\openxs
%\subfile{Ch22}\openxs
%\subfile{Ch23}\openxs
\end{document}

