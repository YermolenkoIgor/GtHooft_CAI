
\documentclass[aps,%
12pt,%
final,%
oneside,
onecolumn,%
musixtex, %
superscriptaddress,%
centertags]{article} %% 



\renewcommand{\baselinestretch}{1.5}

%\usepackage[left=2.0cm,right=1.5cm,
   % top=2.5cm,bottom=2.5cm,bindingoffset=0cm]{geometry}
\usepackage[english,russian]{babel}
\usepackage[utf8]{inputenc}
\usepackage{scrextend}
\usepackage[12pt]{extsizes}
\usepackage[a4paper, left=20mm, right=20mm, top=20mm, bottom=20mm]{geometry}

%\documentclass[a4paper,fontsize=14bp]{extreport}
%всякие настройки по желанию%
\usepackage[colorlinks=true,linkcolor=black,unicode=true]{hyperref}% включить ссылки
\usepackage{euscript}
\usepackage{supertabular}
\usepackage[pdftex]{graphicx}
\usepackage{amsthm,amssymb, amsmath}
\usepackage{textcomp}
\usepackage[noend]{algorithmic}
\usepackage[ruled]{algorithm}
\usepackage{ulem} % sout
\usepackage{cite} % ссылки по Госту 
%
%\usepackage[nooneline]{caption} \captionsetup[table]{justification=raggedright} %\captionsetup[figure]{justification=centering,labelsep=endash}

\selectlanguage{russian}

%\renewcommand{\rmdefault}{ftm} % Times New Roman
%\frenchspacing
\usepackage{subfiles}


\begin{document}

\begin{titlepage} 
\begin{center}

% Title
\textbf{\Large Герард ’т Хоофт}\\[0.5cm]
\textbf{\LARGE "Интерпретация клеточных автоматов в квантовой механике"}\footnote{G. 't Hooft, The Cellular Automaton Interpretation of Quantum Mechanics,
Fundamental Theories of Physics 185, \href{https://doi.org/10.1007/978-3-319-41285-6_2}{DOI}}\\[3.0cm]


\begin{flushright} \large
\emph{Вёрстка и перевод:} \\
магистрант \\
\textsc{Ермоленко И. П.}
\end{flushright}


\vfill 

% Bottom of the page
{\large {Волгоград}} \par
{\large {2020 г.}}
\end{center} 
\end{titlepage}

Эта книга никоим образом не предназначена для замены стандартной теории квантовой механики. Читателю, еще не досконально знакомому с основными понятиями квантовой механики, рекомендуется сначала изучить эту теорию из рекомендуемых учебников, и только потом приступать к чтению книги, чтобы узнать, что доктрина называемая «квантовой механикой» рассматривается как часть чудесного математического механизма, который помещает физические явления в более широкий контекст, и только потом, как теория природы. 

Нынешняя версия была тщательно изменена. Были добавлены некоторые новинки, такие как нетрадиционный взгляд на стрелу времени, а другие аргументы были дополнительно уточнены. Книга теперь разделена на две части. Часть I посвящена многим концептуальным вопросам, не требующим чрезмерных расчетов. Часть II добавляет к этому наши методы расчета, иногда возвращаясь к концептуальным вопросам. Неизбежно, текст в обеих частях часто будет относиться к обсуждениям в другой части, но они могут быть изучены отдельно. Эта книга не роман, который должен быть прочитан от начала до конца, а коллекция описаний и произвольностей, которые будут использоваться в качестве ссылки. Различные части могут быть прочитаны в случайном порядке. Некоторые аргументы повторяются несколько раз, но каждый раз в другом контексте.

\newpage

% Table of contents
\tableofcontents

\newpage

\part{Интерпретация клеточного автомата: общая доктрина}


\numberwithin{equation}{section}
\subfile{Ch1}
\subfile{Ch2}
%\subfile{Ch3}

\end{document}

