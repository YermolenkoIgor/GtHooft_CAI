\documentclass[main.tex]{subfiles}
\begin{document}


\section{Выводы}\label{ch10}

В более ранней версии этого текста к названию этой книги был добавлен подзаголовок: взгляд на квантовую природу нашей вселенной. Это вызвало возражения: «ваш взгляд кажется более классическим, чем все, что мы видели раньше!» На самом деле, это может быть оспорено. Мы утверждаем, что классические базовые законы можно превратить в квантово-механические законы не оставив от последних практически и следа. Мы настаиваем на том, что у нас тут выходит настоящая квантовая механика, включая все «квантовые странности». Природа нашей вселенной является квантово-механической, но она может иметь классическое объяснение. Основные классические законы могут рассматриваться как полностью классические. Мы показываем, как «квантово-механические» вероятности могут возникать из полностью классических вероятностей.

Являются ли предоставленные здесь взгляды «превосходящими» другие интерпретации, мы оставляем на усмотрение читателя. Собственное мнение автора должно быть понятным. Я не верю в выдумки, с которыми выступают некоторые из моих коллег - им бы я предпочел ту же оригинальную копенгагенскую интерпретацию без каких-либо изменений.

Может показаться странным, что наша теория, в отличие от большинства других подходов, не содержит никаких странных видов стохастических дифференциальных уравнений, никакой “квантовой логики”, никакой бесконечности других вселенных, никакой пилотной волны, а только совершенно обычные уравнения движения, которые мы едва ли смогли бы определить, поскольку они могут быть почти чем угодно. Наш самый важный момент заключается в том, что нас не должны пугать запрещающие теоремы, если они содержат мелкие шрифты и эмоциональные компоненты в своих аргументах. Небольшой отпечаток, который мы обнаруживаем, - это предполагаемое отсутствие сильных локальных и нелокальных корреляций в начальном состоянии. Наши модели показывают, что должны быть такие сильные корреляции. Корреляции не требуют сверхсветовых сигналов, не говоря уже о сигналах, идущих назад во времени.

Эмоциональным ингредиентом является идея о том, что сохранение онтологической природы волновой функции потребовало бы какого-то «заговора», поскольку считается невероятным, что законы природы сами могут позаботиться об этом. Наша точка зрения в том, что они, очевидно, могут. Как только мы это осознаем, мы можем рассмотреть возможность изучения очень простых локальных теорий.

Является ли клеточный автомат эквивалентным квантовой теории или квантовой теории поля? Как указано выше, ответ: формально да, но в большинстве случаев квантовые уравнения не будут сильно напоминать реальный мир. Получить локальность в квантовом смысле из клеточного автомата, который является классическим локальным, сложно, и в большинстве случаев трудно доказать положительность гамильтониана или ограниченность гамильтоновой плотности.

В принципе почти тривиально получить «квантовую механику» из классических теорий. Мы продемонстрировали, как это можно сделать с помощью такой классической системы, как ньютоновские планеты, движущиеся вокруг Солнца. Но тогда возникают трудности, которые, конечно, объясняют, почему наше предложение не так просто. Позитивность гамильтониана является одним из главных камней преткновения. Мы можем обеспечить его соблюдение, но тогда необходимо тщательно проанализировать правдоподобие наших моделей. В самом конце мы должны признать, что проблема, скорее всего, связана с загадками квантовой гравитации. Наше нынешнее понимание квантовой гравитации предполагает, что дискретизированная информация распространяется в искривленном пространственно-временном многообразии; это трудно согласовать с различными свойствами непрерывной симметрии Природы, такими как инвариантность Лоренца. Так что, да, эти вопросы трудно исправить, но мы предполагаем, что эти трудности будут только косвенно связаны с проблемой интерпретации квантовой механики.

Собственно, данная книга построена вокруг этих вопросов и инструментов, необходимых для их решения - инструментов  традиционной квантовой механики, такими как группы симметрии и теорема Нетера.

Следует проводить различие между, с одной стороны, явными теориями, касающимися судьбы квантовой механики в малейшей значимой шкале расстояний в физике, и с другой стороны, предложениями по интерпретации сегодняшних открытий, касающихся квантовых явлений.

Наши теории относительно самого маленького масштаба природных явлений все еще очень неполны. Теория суперструн прошла долгий путь, но, похоже, делает наш взгляд более непрозрачным, чем хотелось бы; во всяком случае, в этой книге мы исследовали только довольно упрощенные модели, из которых, по крайней мере, понятно, о чем они говорят.

\subsection{ИКА}\label{ch10.1}

Таким образом, у нас собралось достаточно измышлений чтобы очертить конкретную интерпретацию того, что в действительности представляет собой квантовая механика. Технические детали базовой теории не имеют здесь большого значения. Нужно только предположить, что существует некоторая онтологическая теория; это будет теория, которая описывает явления на очень малых масштабах в терминах законов эволюции, которые обрабатывают биты и байты информации. Эти законы эволюции могут быть «настолько локальными, насколько это возможно», требуя непосредственного взаимодействия только ближайших соседей. Информация также строго дискретна, в том смысле, что каждый «планковский» объем пространства может содержать только несколько битов и байтов. Мы также подозреваем, что биты и байты обрабатываются как функция локального времени, в том смысле, что только конечный объем обработки информации может иметь место в конечном объеме пространства-времени. С другой стороны, можно подозревать, что происходит некоторая форма потери информации, так что информация может рассматриваться как занимающая элементы поверхности, а не элементы объема, но этот вариант недостаточно проработан.

В любом случае, в своей самой основной форме, эта локальная теория обрабатываемой информации, не требует правильного формулирования каких-либо гильбертовых пространств или принципов суперпозиции. На самом базовом уровне физики (но только там) биты и байты, которые мы обсуждаем, являются классическими битами и байтами. На этом уровне кубиты не играют никакой роли, в отличие от более стандартных подходов, рассматриваемых в современной литературе. Гильбертово пространство входит только тогда, когда мы хотим применить мощный математический механизм для решения вопроса о том, как эти законы эволюции порождают крупномасштабное, возможно коллективное, поведение данных.

Наша теория для интерпретации наблюдаемых явлений теперь ясна: человечество обнаружило, что явления в масштабе расстояний и энергии Стандартной Модели (которая включает в себя расстояния, значительно большие, и энергии, намного меньшие, чем планковский масштаб), могут быть уловлены путем постулирования эффективных шаблонов. Шаблоны - это элементы гильбертова пространства, которые образуют основу, которую можно выбирать разными способами (частицы, поля, запутанные объекты и т. д.), что позволяет нам вычислять коллективное поведение решений эволюционных уравнений, которые требуют использования Гильбертова пространства и линейных операций в этом пространстве. Оригинальные наблюдаемые (beables) могут быть выражены как суперпозиции наших шаблонов. Выбор суперпозиций разнится от случая к случаю. Это странно, но не непостижимо. По-видимому, существует мощная схема преобразований симметрии, позволяющая нам использовать одни и те же шаблоны при многих различных обстоятельствах. Правило для преобразования beables в шаблоны и обратно является сложным и не однозначным; Как именно должны быть сформулированы правила для всех объектов, о которых мы знаем во вселенной, неизвестно или не понято, но их необходимо оставить для дальнейших исследований.

Самое главное, что исходные онтологические beables не допускают никакой суперпозиции, так же как мы не можем осмысленно наложить планеты, но шаблоны, с которыми мы сравниваем beables, являются элементами гильбертова пространства и требуют хорошо известных принципов суперпозиции.

Вторым элементом в нашей ИКА является то, что объекты, которые мы обычно называем классическими, такие как планеты и люди, а также результаты измерений, могут быть напрямую получены из beables, в принципе, без вмешательства шаблонов.

Конечно, если мы хотим знать, как работают наши измерительные устройства, мы используем наши шаблоны, и это является источником обычной «проблемы измерения». Что часто изображается как загадки в квантовой теории: проблема измерения, «коллапс волновой функции» и кошка Шредингера полностью прояснены в ИКА. Все волновые функции, которые когда-либо возникнут в нашем мире, могут казаться суперпозициями наших шаблонов, но они полностью достигают пика, "сворачиваются", как только мы используем приемлемый базис. Поскольку классические устройства также достигают пика в beables базисе, их волновые функции коллапсируют. Для этого не требуется никакого нарушения уравнения Шредингера, напротив, шаблоны, а косвенно и beables, точно подчиняются уравнению Шредингера.

Короче говоря, не сами степени свободы Природы допускают суперпозицию, а просто мы используем шаблоны, которые являются искусственными суперпозициями онтологических состояний Природы. Тот факт, что мы сталкиваемся с, по-видимому, неизбежными парадоксами, касающимися суперпозиции естественных состояний, обусловлен нашим интуитивным представлением о том, что наши шаблоны каким-то образом представляют реальность. Если вместо этого мы начнем с онтологических состояний, которые нам когда-нибудь удастся охарактеризовать, так называемые «квантовые тайны» исчезнут.

\subsection{Контрфактическая определенность}\label{ch10.2}

Предположим, у нас есть оператор $\hat{\mathcal O}_1$, значение которого невозможно измерить, поскольку было измерено значение другого оператора, $\hat{\mathcal O}_2$, а $ [\hat{\mathcal O}_1, \hat{\mathcal O}_2] \neq 0$. Контрфактическая реальность - это предположение, что, тем не менее, оператор $\hat{\mathcal O}_1$ принимает некоторое значение, даже если мы этого не знаем. Часто предполагается, что теории скрытых переменных подразумевают контрфактуальную определенность. Мы должны категорически подчеркнуть, что в интерпретации клеточных автоматов такого предположения не делается. В этой теории оператор $\hat{\mathcal O}_2$, значение которого было измерено, по-видимому, состоит из онтологических наблюдаемых (beables). Оператор $\hat{\mathcal O}_1$ по определению не является онтологическим и поэтому не имеет четко определенного значения, по той же причине, почему в планетной системе оператор обмена $Earth-Mars$, собственные значения которого равны $\pm 1$, не имеет ни одного из этих значений; это не определено, несмотря на то, что планеты развиваются классически, и, несмотря на то, что доктрина Копенгагена будет диктовать, что эти собственные значения наблюдаемы!

Хитрость в ИКА применительно к атомам и молекулам заключается в том, что часто априори неизвестно, какие из наших операторов являются beables, а какие - переменными (changeables) или superimposables (как определено в разделах \ref{ch2.1.1} и \ref{ch5.5.1}). Это известно только a posteriori. Как так? Мы используем шаблоны для описания атомов и молекул, и эти шаблоны дают нам настолько тщательно перемешанное представление о полном наборе наблюдаемых в теории, что мы остаемся в неведении, пока кто-то не решит что-то измерить.

Кажется, что простой акт измерения посылает сигнал назад во времени и / или со сверхсветовой скоростью в другие части вселенной, чтобы информировать наблюдателей, какие из их наблюдаемых могут быть измерены, а какие нет. Разумеется это не так. Просто мы узнаем, что можно точно измерить и какие измерения дадут неопределенные результаты \textbf{Уточнить}. Неравенства Белла и CHSH нарушаются, как и должно быть в квантовой теории поля, в то время как квантовая теория поля запрещает возможность посылать сигналы быстрее света или назад в прошлое.

\subsection{Супердетерминизм и Заговор}\label{ch10.3}

Супердетерминизм может быть определен так, чтобы подразумевать, что не только все физические явления объявляются прямыми следствиями физических законов, которые нигде не оставляют ничего случайного (что мы называем «детерминизмом»), но он также подчеркивает, что сами наблюдатели ведут себя в соответствии с теми же законами. Они также не могут совершать какие-либо причудливые поступки без причины в ближайшем прошлом, а также в далеком прошлом. Само по себе это утверждение настолько очевидно, что для его обоснования потребуется небольшая дискуссия. Тот факт, что наблюдатель не может сбросить свое измерительное устройство без изменения физических состояний в прошлом, обычно считается неуместным для нашего описания физических законов. ИКА утверждает, что это так. Дальнейшие объяснения были даны в разд. \ref{ch3.8}, где мы попытались демистифицировать «свободную волю».

Часто утверждается, что если мы хотим, чтобы какое-либо сверхдетерминированное явление привело к нарушению неравенств Белла-CHSH, то это потребует сговора между наблюдаемым распадающимся атомом и процессами, происходящими в сознании Алисы и Боба, что было бы заговором такого рода, который не должен допускаться ни в одной приличной теории для законов естества. Идею о том, что может существовать естественный механизм, который управляет поведением Алисы и Боба, часто трудно принять.

Однако в ТКА естественные законы запрещают возникновение состояний, в которых beables накладываются друг на друга. Ни Алиса, ни Боб никогда не смогут создать такие состояния, вращая свои поляризационные фильтры. Действительно, состояние, в котором находится их разум, является онтологическим с точки зрения beables состояний, и наши несчастные не смогут это изменить.

Супердетерминизм следует рассматривать в связи с корреляциями на пространственно-подобных расстояниях. Мы утверждаем, что корреляции не просто есть, но и также чрезвычайно сильны. Состояние, которое мы называем «вакуумным состоянием», полно корреляций. Квантовая теория поля (обсуждаемая в части II, раздел \ref{ch20}) должна быть прямым следствием вытекающим из онтологической теории. Все 2-частичные математические ожидания, называемые пропагаторами, не исчезают как внутри, так и снаружи светового конуса. Кроме того, там не исчезают многочастичные ожидаемые значения; действительно, при аналитическом допиливании эти амплитуды превращаются в полноценную матрицу рассеяния, которая инкапсулирует все законы физики, которые подразумеваются теорией. В Гл. \ref{ch3.6}, показано, как 3-точечная корреляция может в принципе порождать нарушение неравенства CHSH, как того требует квантовая механика. 

Принимая во внимание эти корреляционные функции и закон, который гласит, что beables никогда не будут накладываться друг на друга, мы теперь подозреваем, что этот закон запрещает Алисе и Бобу менять свое мнение таким образом, чтобы эти корреляционные функции больше не действовали.

\subsubsection{Роль запутанности}\label{ch10.3.1}

Мы не претендуем на последнее слово в теореме Белла. Можно перефразировать наши наблюдения несколько иным способом. Причина, по которой стандартная копенгагенская теория квантовой механики нарушает Белла, заключается в том, что два фотона (или электроны, или какие бы то ни было частицы, которые рассматриваются) являются квантово запутанными. В стандартной теории предполагается, что Алиса может изменить свои настройки, не затрагивая фотон, который находится на пути к Бобу. Возможно ли это, если фотон Алисы и фотон Боба запутаны?

Согласно ИКА, мы использовали шаблоны для описания запутанных фотонов, на которые смотрят Алиса и Боб, и это приводит к нарушению неравенств CHSH. В действительности эти шаблоны отражали относительные корреляции онтологических переменных, лежащих в основе этих фотонов. Чтобы описать запутанные состояния как beables, необходима их коррелятивность. Мы предполагаем, что это тот случай, когда частицы распадаются на запутанные пары, потому что распад должен объясняться флуктуациями вакуума (см. Раздел \ref{ch5.7.5}), в то время как вакуум также не может быть единственным трансляционно-инвариантным онтологическим состоянием.

Сброс эксперимента Алисы без внесения изменений рядом с Бобом приведет к состоянию, которое в терминах квантовой механики не ортогонально исходному и, следовательно, не является онтологическим. Тот факт, что новое состояние не ортогонально предыдущему, вполне соответствует стандартным квантовомеханическим описаниям; в конце концов, фотон Алисы был заменен суперпозицией.

Остается вопрос, как так могло получиться. Если клеточный автомат остается таким, как он находится рядом с Бобом, почему «контрфактивное» состояние не ортогонально ему? ИКА говорит об этом, поскольку классическая конфигурация ее аппарата изменилась, и мы заявили, что любое изменение классической установки приводит к изменению онтологического состояния, которое является переходом к ортогональному вектору в гильбертовом пространстве.

Мы не можем исключить возможность того, что кажущаяся нелокальность в отображении онтического шаблона связана с трудностью идентификации правильного гамильтониана для стандартной квантовой теории в терминах онтических состояний. Мы должны найти гамильтониан, который является интегралом локальной гамильтоновой плотности. Строго говоря, здесь могут быть осложнения; Гамильтониан, который мы используем, часто является приближенным, очень хорошим, но он игнорирует некоторые тонкие нелокальные эффекты. Это будет дополнительно объяснено в части II.

\subsubsection{Выбор базиса}\label{ch10.3.2}

Некоторые физики предыдущих поколений считали, что дифференциация различных наборов базисов очень важна. Являются ли частицы «настоящими частицами» в импульсном пространстве или в конфигурационном пространстве? Каковы «истинные» переменные электромагнетизма, фотонов или электрических и магнитных полей?\footnote{В. Лэмб хорошо известен своей цитатой: ''я предлагаю, чтобы для использования слова фотон требовалась лицензия, и я предлагаю давать такую лицензию только должным образом квалифицированным людям.''} Современный взгляд состоит в том, чтобы подчеркнуть, что любой базис служит нашим целям так же, как и любой другой, поскольку, как правило, ни одно из традиционно выбранных базисных пространств не является действительно онтологическим.

В этом отношении гильбертово пространство следует рассматривать как любое обычное векторное пространство, такое, как пространство, в котором определены все координаты планет в нашей планетарной системе. Не имеет значения, какую систему координат мы используем. Должна ли ось $Z$ быть ортогональной к локальной поверхности Земли? Параллельной земной оси вращения? Эклиптике? Без разницы, ведь выбор координат несущественен. Очевидно, это иллюстрируется в нотации Дирака. Это красивое обозначение является чрезвычайно общим и как таковое оно вполне подходит для обсуждений, представленных в этой книге.

Но затем, после того как мы объявили, что все наборы базисов одинаково хороши, поскольку они описывают какой-то знакомый физический процесс или событие, мы рискуем предположить, что можно сделать выбор некого особого базиса. Может существовать базис, в которой все сверхмикроскопические физические наблюдаемые являются beables; это наблюдаемые величины в масштабе Планка, все они должны быть диагональными в этом базисе. Кроме того, в этом базисе волновая функция состоит только из нулей и единицы. Этот специальный базис, называемый «онтологическим», сегодня скрыт от нас, но в принципе должна быть возможность идентифицировать его.\footnote{Там может быть больше чем один, неравноценный выбор, как описано выше (разд. \ref{ch22.3}).} Эта книга о поиске такого базиса. Это не основа частиц, поля частиц, атомов или молекул, а нечто гораздо более сложное, чтобы на него можно было положиться .

Мы подчеркиваем, что также все классические наблюдаемые, описывающие звезды и планеты, автомобили, людей и, в конечном итоге, указатели на детекторы, будут диагональными в том же онтологическом базисе. Это имеет решающее значение, и это было объяснено в разд. \ref{ch4.2}.

\subsubsection{Корреляции и скрытая информация}\label{ch10.3.3}

Существенным элементом в нашем анализе могут быть наблюдения, изложенные в разд. \ref{ch3.7.1}. Было отмечено, что детали онтологического базиса должны нести важную информацию о будущем, но скрытно: информация нелокальна. Простой факт заключается в том, что во всех наших моделях онтологическая природа состояния, в котором находится вселенная, будет сохраняться во времени: как только мы находимся в четко определенном онтологическом состоянии, в отличие от шаблона, эта функция будет сохранена во времени (закон сохранения онтологии). Это не позволяет Алисе и Бобу выбрать любую настройку, которая бы «измеряла» шаблон, который не является онтологическим. Таким образом, эта особенность предотвращает противоречивую определенность. Даже помет мыши не может не подчиниться этому принципу.

Принимая CAI, мы принимаем идею, что все события в этой вселенной сильно коррелированы, но таким образом, что мы не можем использовать это на практике. Было бы справедливо сказать, что эти функции все еще несут в себе чувство таинственности, которое необходимо больше исследовать, но единственный способ сделать это - поиск более совершенных моделей.

\subsection{Важность вторичного квантования}\label{ch10.4}

Мы понимаем, что, несмотря на все наблюдения, сделанные в этой книге, наши аргументы получили бы гораздо большую поддержку, если бы можно было построить более конкретные модели, в которых надежные вычисления подтверждают наши выводы. У нас есть конкретные модели, но, мягко говоря, они были далеко не идеальными для объяснения наших соображений. Что действительно нужно, так это модель или набор моделей, которые, очевидно, подчиняются квантово-механическим уравнениям, хотя они являются классическими и детерминированными по своей структуре. В идеале мы должны найти модели, которые воспроизводят релятивистские квантованные теории поля с взаимодействиями, прям как стандартная модель.

Использование второго квантования будет объяснено далее в разд. \ref{ch15.2.3} и \ref{ch22.1} части II. Мы начнем с произвольного гамильтониана и вставим взаимодействия на более поздней стадии, процедура, которая обычно предполагается возможной с использованием разложения возмущения. Используемая здесь уловка заключается в том, что теория свободных частиц будет явно локальной, и взаимодействия, представленные редкими переходами, также будут локальными. Взаимодействия вводятся путем постулирования новых переходов, которые создают или уничтожают частицы. Все члены в этом разложении являются локальными, поэтому мы имеем локальный гамильтониан. Чтобы справиться с полной теорией, нужно выполнить полное разложение возмущений. Соблюдая все правила квантовой теории возмущений, мы должны получить описание цельной модели взаимодействия в квантовых терминах. Действительно, мы должны воспроизвести подлинную квантовую теорию поля таким образом.

Что, прям так? Строго говоря, разложение возмущений не сходится, как это объясняется также в разд. \ref{ch22.1}. Однако тогда мы можем утверждать, что это нормальная ситуация в квантовой теории поля. Разложения возмущений формально всегда расходятся, но они - лучшее, что у нас есть - они действительно позволяют нам делать чрезвычайно точные вычисления на практике. Следовательно, воспроизведение этих пертурбативных разложений, независимо от того, насколько хорошо они сходятся, - это все, что нам нужно сделать в наших квантовых теориях.

Принципиально сходящееся выражение для гамильтониана существует, но оно совершенно иное. Различия в нелокальных условиях, которые мы обычно не наблюдаем. Посмотрите на точное выражение, уравнение (\ref{2.8}), впервые обработанное в разд. \ref{ch2.1} гл. \ref{ch2}

\begin{equation}\label{10.1}
	H_{\mathrm{op}} \delta t=\pi-i \sum_{n=1}^{\infty} \frac{1}{n}\left(U_{\mathrm{op}}(n \delta t)-U_{\mathrm{op}}(-n \delta t)\right)
\end{equation}

Оно сходится, за исключением самого вакуумного состояния. Состояния с низкой энергией, состояния, очень близкие к вакууму, являются состояниями, в которых сходимость является чрезмерно медленной. Следовательно, как было объяснено ранее, в теорию вкрапляются условия, которые являются чрезвычайно нелокальными.

Это не означает, что клеточный автомат будет нелокальным; он настолько локальный, насколько это возможно, но получается, что если мы хотим описать его с бесконечной точностью в терминах квантовой механики, то придется использовать гамильтониан могущий генерировать нелокальности. Можно рассматривать эти нелокальности как результат того, что наш гамильтониан заменяет разностные уравнения времени, связывая экземпляры, разделенные интегральными кратными $\delta t$, дифференциальными уравнениями во времени; состояния между четко определенными временными точками обязательно являются нелокальными функциями физически соответствующих состояний.

Можно было бы возразить, что не должно быть никаких оснований пытаться заполнить промежутки между шагами целочисленного времени, но тогда не существовало бы аддитивной энергетической функции, которая нам нужна для стабилизации решений наших уравнений. Возможно, нам придется использовать комбинацию классической целочисленной функции Гамильтона (гл. \ref{ch19} Части II), а периодически определяемый гамильтониан связывает только целочисленные временные шаги, но точно, как это сделать правильно, все еще находится в стадии исследования. Мы еще не нашли наилучшего возможного подхода к построению искомого оператора Гамильтона для наших шаблонных состояний. Теория вторичного квантования расскажет о нашей лучшей попытке, но пока мы не смогли воспроизвести довольно сложную структуру симметрии Стандартной модели, в частности инвариантность Пуанкаре.

Пока у нас нет явной модели, которая в той степени, в которой это необходимо, воспроизводит стандартные модельные взаимодействия, мы не можем проверить правильность нашего подхода. За неимением оных, пока перебьемся расчетами моделей, которые являются максимально реалистичными.
Возможностей для экспериментальных проверок теории КА, к сожалению, мало, и они далеки друг от друга. Наша основная цель состояла не в том, чтобы изменить квантовые законы физики, а скорее в том, чтобы получить более прозрачное описание того, что на самом деле происходит. Это должно помочь нам построить физические модели, особенно в масштабе Планка, и, если мы добьемся успеха, эти модели должны быть проверены экспериментально. Большинство наших прогнозов отрицательны: не будет заметного отклонения от уравнения Шредингера и вероятностного выражения Борна.

Более интересное негативное предсказание касается квантовых компьютеров, как было объяснено в разд. \ref{ch5.8}. Квантовые устройства должны, в принципе, позволять выполнять очень большое количество вычислений «одновременно», так что есть математические проблемы, которые могут быть решены гораздо быстрее, чем на стандартных классических компьютерах. Тогда почему эти квантовые компьютеры еще не созданы? Похоже, на пути стоят некоторые «технические трудности»: используемые квантовые биты - кубиты - имеют тенденцию к декогереции. Средство защиты, которое, как ожидается, устранит такие мешающие функции, представляет собой комбинацию превосходного оборудования (кубиты должны быть как можно лучше изолированы от среды) и программного обеспечения: коды исправления ошибок. И тогда квантовый компьютер должен научиться совершать математические чудеса.

Согласно теории КА, наш мир является просто квантово-механическим, потому что было бы слишком сложно даже приблизительно проследить клеточные состояния в планковских пространственно-временных масштабах. Если бы мы могли следить за этими данными, мы могли бы сделать это с помощью классических устройств. Это означает, что классический компьютер должен быть способен воспроизводить результаты квантового компьютера, если его ячейки памяти и вычислительное время будут масштабированы до планковских размеров. Никто не может построить такие классические компьютеры, так что квантовые компьютеры действительно смогут творить чудеса; но есть и четкие границы: квантовый компьютер никогда не превзойдет классические компьютеры, если их масштабировать до планковских размеров. Это означало бы, что числа с миллионами цифр никогда не могут быть разложены на простые множители, даже с помощью квантового компьютера. Мы предсказываем, что в нашей способности избежать декогеренции кубитов будут существовать фундаментальные ограничения. Это нетривиальное предсказание, но не из тех, которыми можно гордиться.

\end{document}

