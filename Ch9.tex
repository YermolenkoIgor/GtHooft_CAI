\documentclass[main.tex]{subfiles}
\begin{document}


\section{Открытые вопросы и направления дальнейших исследований}\label{ch9}

\subsection{Неотрицательность гамильтониана}\label{ch9.1}

Как мы подробно расскажем во второй части, вполне вероятно, что сила гравитации будет решающим компонентом в решении оставшихся трудностей в \textit{ТКА}. Одним из аргументов в пользу этого является то, что гравитация частично основана на существовании локальных трансляций времени, которые являются переопределением времени, которое зависит от местоположения в пространстве. Генератором этих преобразований является гамильтонова плотность, которая здесь должна быть локальным оператором. В то же время важно, чтобы теория гравитации имела нижнюю границу плотности энергии. По-видимому, гравитация зависит именно от этих двух важных требований к гамильтонову оператору, которые вызывают у нас некоторые проблемы, и поэтому, по-видимому, проблема квантования гравитации и проблема превращения квантовой механики в детерминированную теорию динамики, будут решены одновременно.

С другой стороны, можно утверждать, что нелокальность квантово-индуцированного гамильтониана - это именно то, что нам нужно, чтобы объяснить теорему Белла.

Роль силы тяжести в нашей теории до конца не ясна. Поэтому можно надеяться, что включение гравитационной силы может быть отложено, и что модели клеточных автоматов можно описать и на плоских решетках пространства-времени. Во второй части мы увидим, что наш PQ-формализм (гл. \ref{ch16}) позволяет разделить пространственно-подобные координаты на две части: целые числа, которые определяют местоположение точки на решетке, и дробные или периодические координаты, которые можно использовать для позиционирования точки в одной ячейке решетки, или просто как канонически сопряженные переменные, связанные с дискретизированной переменной импульса. Здесь, однако, приспособление для некомпактных симметрий, таких как инвариантность Лоренца, чрезвычайно сложно.

Периодически будет подниматься проблема, состоящая в том, что гамильтонианы, воспроизводимые в наших моделях, чаще всего, по-видимому, не имеют нижней границы. Эта проблема будет дополнительно изучена в части II, в разделах \ref{ch14}, \ref{ch22} и  \ref{ch19.1}. Перечислим свойства, которые вызывают конфликты при объединении:

\begin{enumerate}
	\item \textit{H} должен быть постоянным во времени: 

	\begin{equation}\label{9.1}
		\frac{d}{dt}H = 0
	\end{equation}

	\item \textit{H} должен быть ограничен снизу: 

	\begin{equation}\label{9.2}
		\langle \psi \mid H \mid \psi \rangle \ge 0 
	\end{equation}

	\item \textit{H} должен быть обобщенным, то есть он должен быть суммой (или интегралом) локальных членов:

	\begin{equation}\label{9.3}
		H = \sum_{\vec x} \mathcal H(\vec x)
	\end{equation}

	\item И наконец, он должен генерировать уравнения движения:

	\begin{equation}\label{9.4}
		U(t) = e^{-iHt}
	\end{equation}

	или, что эквивалентно, для всех состояний $\mid \psi \rangle$ в гильбертовом пространстве

	\begin{equation}\label{9.5}
		\frac{d}{dt} \mid \psi \rangle = -i H \mid \psi \rangle
	\end{equation}

\end{enumerate}

В стандартных квантовых теориях обычно начинают с некой классической модели и подвергают ее общеизвестной процедуре "квантования". Тогда обычно нетрудно найти оператор \textit{H}, который удовлетворяет этим условиям. Однако, когда мы начнем с классической модели и будем непосредственно искать подходящий гамильтонов оператор, генерирующий его закон эволюции (\ref{9.5}), он будет редко подчиняться как (\ref{9.2}), так и (\ref{9.3}).

Все эти уравнения абсолютно необходимы для понимания квантовой механики. В частности, важность оценки (\ref{9.2}) иногда недооценивается. Если бы границы не было, ни одно из знакомых решений уравнения Шредингера не было бы устойчивым; любое, бесконечно малое возмущение в \textit{H} приведет к решению с энергией E, которое распадается на комбинацию двух пространственно разделенных решений с энергиями $E + \delta E$ и $-\delta E$.

Все решения уравнения Шредингера имели бы такую нестабильность, которая описывала бы мир, совершенно отличающийся от квантового мира, к которому мы привыкли.

По этой причине мы всегда пытаемся найти выражение для \textit{H}, для которого существует нижняя граница (которую мы впоследствии можем нормализовать, чтобы она была точно равна нулю). С формальной точки зрения должно быть легко найти такой гамильтониан. Каждая классическая модель должна предусматривать описание в виде простых моделей раздела \ref{ch2.2.1}, (набор зубчатых колес), и мы видели на рис. \ref{i2.3}, что можно впоследствии настроить константы $\delta E_i$ таким образом, чтобы существовала граница (\ref{9.2}), даже если существует бесконечно много зубчатых колес с неограниченным числом зубьев.

Однако необходимо и третье условие - гамильтонова плотность $\mathcal H(\vec x)$. Локальность соответствует требованию, что на расстояниях, превышающих некоторый предел близкого расстояния, эти плотности Гамильтона должны коммутировать (уравнение (\ref{5.22})).

Есть подозрение, что окончательное решение, удовлетворяющее всем нашим требованиям, будет заключаться в квантовании гравитации, когда мы знаем, что должна существовать локальная гамильтонова плотность, которая генерирует локальные диффеоморфизмы времени. В других трактатах по интерпретации квантовой механики важная роль, которую могла бы играть гравитационная сила, упоминается редко.

Некоторые авторы подозревают, что гравитация является новым элементарным источником "квантовой декогеренции", но такие доводы вряд ли могут быть убедительными. В этих рассуждениях гравитация рассматривается пертурбативно (закон Ньютона рассматривается как аддитивная сила, в то время как черные дыры и рассеяние между гравитационными волнами игнорируются). Как возмущающая, дальнодействующая сила, гравитация ничем не отличается от электромагнитных сил. Декогеренция [95] - это концепция, которую мы здесь полностью избегаем (см. Раздел \ref{ch3.5}).

Поскольку мы не решили проблему положительности гамильтониана, у нас нет систематической процедуры для управления видом гамильтонианов, которые могут быть получены из клеточных автоматов. В идеале мы должны попытаться аппроксимировать гамильтонову плотность Стандартной модели. Приблизительные решения, которые лишь незначительно нарушают наши требования, не сильно помогут из-за проблемы с иерархией (см. Разд. \ref{ch8.2}): Гамильтониан стандартной модели (или лагранжиан, который можно поставить вместо него) требует точной настройки. Это означает, что очень маленькие несоответствия в масштабе Планка приведут к очень большим ошибкам в масштабе Стандартной модели. Проблема иерархии не решена, так что это серьезное препятствие.

Касательно локальности гамильтониана, может быть еще заковырка, скрытая в квантовой гравитации применительно к черным дырам. Незаконченное исследование указывает на некоммутативное свойство координат пространства-времени, приводящее к неопределенным соотношениям вида

\begin{equation}\label{9.6}
	\delta x\cdot \delta t = \mathcal{O}(L^2_{PL})
\end{equation}
           
где $L_{Pl}$ - планковская длина. Считая, что гамильтониан является генератором бесконечно малых трансляций времени (в отличие от операторов эволюции, которые производят конечные сдвиги во времени), можно предположить, что гамильтониан требует, чтобы $\delta E$ было бесконечно малым. В этом случае $\delta x$ велико, так что можно ожидать, что гамильтонова плотность будет нелокальным оператором. Тот факт, что локальность восстанавливается, когда $\delta t$ остается конечным, означает, что клеточный автомат все еще может быть локальным.



\subsection{Вторичное квантование в детерминированной теории}\label{ch9.2}

Когда Дирак вывел свое знаменитое уравнение для описания волновой функции электрона, он понял, что у него есть проблема: уравнение допускает решения с положительной энергией, но они отражаются решениями, в которых энергия, включая энергию покоя, отрицательна. Релятивистское выражение для энергии частицы с импульсом $\vec p$ (в единицах, где скорость света $c$ = 1), имеет вид

\begin{equation}\label{9.7}
	E = \pm \sqrt{m^2 + \vec p^2}
\end{equation}

Если уравнение в частных производных дает такое выражение с квадратным корнем, практически невозможно наложить на волновую функцию такие условия, чтобы исключить нежелательный отрицательный знак, если только не ввести нелокальность - цена, которую Дирак был не готов заплатить. Он придумал более естественный выход:

\textit{Электронов очень много, и решение по N-электронам подчиняется принципу Паули: волновая функция должна менять знак при обмене любой пары электронов. На практике это означает, что каждый электрон должен занимать разные энергетические уровни. Уровни, занятые одним электроном, не могут претендоваться другими электронами. Таким образом, Дирак предположил, что все уровни отрицательной энергии обычно заполнены электронами, так что другие не могут туда попасть. Вакуумное состояние по определению является самым низким энергетическим состоянием, поэтому все отрицательные энергетические уровни там заняты. Если вы положите дополнительный электрон на положительный энергетический уровень или вы удалите электрон из пятна с отрицательной энергией, то в обоих случаях вы получите более высокое энергетическое состояние.}

Если электрон удаляется с отрицательного энергетического уровня, образуется энергетическая дырка, которая теперь является положительной. Ее заряд будет противоположен заряду электрона. Таким образом, Дирак предсказал античастицу, связанную с электроном:\footnote{Дирак сначала подумал, что это может быть Протон, но это было невозможно: масса должна быть равна массе электрона, и позитрон и электрон должны быть способны аннигилировать друг друга, когда они приближаются друг к другу.} частицу с массой $m_e$ и зарядом $+e$, аналогичные массе $m_e$ и заряду $-e$ нормального электрона. Таким образом был предсказан позитрон.

У клеточных автоматов у нас та же проблема. В разделе \ref{ch14} будет объяснено, почему мы не можем поместить край энергетического спектра автомата на нулевой уровень - вакуумное состояние, которое тогда, естественно, будет самым низким энергетическим состоянием. Мы видим, что локальность требует очень гладкой энергетической функции. Если мы отразим спектр так, что $-\pi < E\delta t < \pi$, мы получим ту же проблему, с которой Дирак должен был справиться, и, действительно, мы можем использовать то же решение: вторичное квантование. Как это работает, будет объяснено в разд. \ref{ch15}. Мы берем $k$ фермионных частиц, каждая из которых может занимать N состояний. Если мы диагонализируем оператор $U$ для каждой «частицы», мы обнаружим, что половина состояний имеет положительную энергию, а вторая - отрицательную. Если мы выберем $k = \frac 1 2 N$, то в состоянии с самой низкой энергией все отрицательные энергетические уровни заполнены, а все положительные энергии пусты; так получится низкое энергетическое состояние, вакуум.

Возбужденные состояния получаются, если мы немного отклоняемся от вакуумной конфигурации. Это означает, что мы работаем с энергетическими уровнями, близкими к центру спектра, где мы видим, что Фурье-разложения (гл.\ref{ch14}) продолжает стремительно сближаться. Таким образом, мы получаем описание мира, где все чистые энергии положительны, а быстрая сходимость разложения Фурье гарантирует эффективную локальность.

Тогда это идеальное решение нашей проблемы? Почти, но не совсем. Во-первых, мы находим только оправданные описания вторично квантованных фермионов. Бозонный случай будет более тонким и еще не совсем понятным.\footnote{Но у нас хороший старт: бозоны - это кванты энергии гармонических осцилляторов, которые мы должны сначала заменить гармоническими ротаторами, см. \ref{ch12.1}-\ref{ch13}. Наша трудность состоит в том, чтобы построить гармонически связанные цепи таких вращателей. Наши процедуры достаточно хорошо работали в одном пространственно-временном измерении, но у нас нет бозонного эквивалента нейтринной модели (см. \ref{ch15.2}).} Во-вторых, мы должны заменить уравнение Дирака каким-то детерминированным законом эволюции, тогда как детерминистическая теория будет раскрыта в разд. \ref{ch15.2} описывает плоскости (листы), а не частицы. Мы не знаем, как описать локальные детерминированные взаимодействия между такими листами. На данный момент у нас имеется описание невзаимодействующих частиц. Перед введением взаимодействий, которые также являются детерминированными, уравнения листа должны быть заменены чем-то другим.

Предполагая, что эта проблема может быть решена, мы разрабатываем формализм для взаимодействий в разд. \ref{ch22.1}. Гамильтониан взаимодействия получается из детерминированного закона взаимодействий, использующих разложение БЧХ, которое не гарантируется сходимостью. Это может не быть проблемой, если взаимодействие слабое. Мы приводим аргументы, почему в этом случае конвергенция все еще может быть достаточно быстрой, чтобы получить полезную теорию. Теория тогда не бесконечно точна, но это не удивительно. Мы могли бы заявить, что действительно, эта проблема была с нами все время в квантовой теории поля: пертурбативное расширение теории хорошо, и это дает ответы, которые гораздо более точны, чем числа, которые можно получить из любого эксперимента, но они не являются бесконечно точными только потому, что разложение возмущений не сходится (это можно рассматривать как просто асимптотическое разложение). Таким образом, наша теория воспроизводит в точности то, что известно о квантовой механике и квантовой теории поля, просто говоря нам, что, если нам нужно более точное описание, нам, возможно, придется взглянуть на сам оригинальный автомат.

Нет необходимости подчеркивать, что некоторые из выдвинутых здесь идей в основном являются спекуляцией, они должны быть подтверждены более явными расчетами и моделями.

\subsection{Потеря информации и инверсия времени}\label{ch9.3}

Очень важное наблюдение, сделанное в разд. \ref{ch7} заключается в том, что, если мы введем в детерминированную модель потерю информации, общее количество ортогональных базисных элементов онтологического базиса может быть значительно уменьшено, хотя, тем не менее, получающаяся в результате квантовая система не будет показывать никаких признаков необратимости во времени. Однако классические состояния, относящиеся к результатам измерений и т.п., связаны с исходными онтологическими состояниями и, следовательно, обладают термодинамической стрелой времени.

Это вполне может объяснить, почему мы имеем необратимость времени в больших классических масштабах, в то время как микроскопические законы, насколько они известны сегодня, все еще кажутся совершенно обратимыми во времени.

Чтобы справиться с возникновением потери информации на субмикроскопическом уровне, мы ввели понятие классов информационной эквивалентности: все состояния, которые в течение определенного конечного промежутка времени переходят в одно и то же онтологическое состояние $\mid \psi(t) \rangle$, информационно эквивалентны, все они представлены как один и тот же квантовый базисный элемент $\mid \psi(0) \rangle$. Мы уже упоминали о сходстве классов инфоэквивалентности и классов локальной калибровочной эквивалентности. Может быть, мы говорим об одном и том же?

Если это так, то это будет значить, что локальные калибровочные преобразования некоторого состояния, представленного в виде онтологического состояния, могут фактически описывать различные онтологические состояния в данный момент времени $t = t_0$, тогда как два онтических состояния, которые отличаются друг от друга только локальным калибровочным преобразованием, могут обладать тем свойством, что они оба перейдут в одно и то же конечное состояние, что, естественно, объясняет, почему наблюдатели не смогут их различить.

Теперь мы осознаем тот факт, что эти заявления о чем-то принципиально ненаблюдаемом, поэтому их актуальность, несомненно, может быть поставлена под сомнение.

Тем не менее это предположение оправдано следующим образом. Можно заметить, что формулирование квантовой теории поля без использования принципа локальной калибровочной эквивалентности представляется практически невозможным \footnote{Взаимодействия должны были бы быть довольно слабыми, такими как электромагнитные.}, поэтому существование классов локальной калибровочной эквивалентности можно приписать математическим свойствам этих квантованных полей. Лишь в редких случаях можно заменить теорию локальной калибровочной эквивалентностью так, чтоб эта особенность отсутствовала или имела совершенно другую природу.\footnote{Известны примеры, в виде двойственных преобразований.} Точно то же самое можно сказать и о классах онтологической эквивалентности. Они будут одинаково ненаблюдаемы в больших масштабах - по определению. Однако перефразировать детерминистскую теорию, избегая в целом этих классов эквивалентности, может быть непомерно сложно (даже если они не исключаются в принципе). Итак, наш аргумент прост: в Природе эти классы эквивалентности настолько схожи, что могут иметь общее происхождение.

Это оставляет интересный вопрос: общая теория относительности также основана на принципе локальной калибровки - эквивалентности локально изогнутых координатных систем. Можем ли мы сказать то же самое об этом классе калибровочной эквивалентности? Может ли это быть также из-за потери информации? Это будет означать, что наша основная теория должна быть сформулирована в фиксированной локальной системе координат. Общая инвариантность координат будет тогда приписываться тому факту, что информация, которая определяет наши локальные координаты, может быть потеряна. Является ли такая идея жизнеспособной? Должны ли мы исследовать это?

Мой ответ - да! Это могло бы прояснить некоторые загадки в современной теории относительности и космологии. Почему космологическая постоянная такая маленькая? Почему Вселенная пространственно плоская (кроме локальных колебаний)? И, что касается нашего клеточного автомата: как нам описать автомат в искривленном пространстве-времени?
Ответы на эти вопросы: да, наша вселенная искривлена, но кривизна ограничена локальными эффектами. У нас есть важная переменная, играющая роль метрического тензора $g_{\mu\nu}(\vec x, t)$ в нашем автомате, но она живет в хорошо определенной системе координат, которая является плоской. То есть условие калибровки подчиняется $g_{i0} = g^{i0} = 0$, $i = 1, 2, 3$. Пусть тогда $\lambda^i$ - три собственных значения $g^{ij}$, пространственноподобных компонент обратной метрики (так что $(\lambda^i)^{-1}\equiv\lambda_i $, являются собственными значениями $g_{ij}$). Пусть $\lambda^0 = |g^{00}|$. Тогда локальная скорость света в терминах используемых координат определяется как $c^2 = | \vec\lambda | / \lambda^0$. Мы можем наложить неравенство на $c$: предположим, что значения метрического тензора вынуждены подчиняться $| c | \le 1$. Если теперь мы напишем $g_{\mu\nu} = \omega^2(\vec x, t)\hat g_{\mu\nu}$, с ограничением на $\hat g_{\mu\nu}$, таким как  $\mathrm{det}(\hat g_{\mu\nu}) = -1$ (см. Приложение \ref{chB}), то нет ограничения на значение $\omega$. Это означает, что вселенная может раздуваться до любого размера, но в среднем она может оставаться плоской.
Мы продолжаем эту тему в части II, раздел. \ref{ch22.4}.

\subsection{Голография и излучение Хокинга}\label{ch9.4}

Есть и другая причина ожидать, что потеря информации будет неизбежной особенностью окончательной теории физики, которая касается микроскопических черных дыр. Классически, черные дыры поглощают информацию, поэтому можно ожидать, что наша классическая система или система «prequantum» также имеет потерю информации. Еще более убедительной является оригинальная идея голографии [81, 117]. Важным выводом Стивена Хокинга было то, что черные дыры испускают частицы [47, 48] из-за квантовых флуктуаций вблизи горизонта. Однако его наблюдение привело к парадоксу:

\textit{Расчет показывает, что частицы возникают в тепловом состоянии с идеальной случайностью, независимо от того, как образовалась черная дыра. Не только в детерминистских теориях, но и в чисто квантовых теориях, расположение частиц, которые выходят, должно в некоторой степени зависеть от частиц, которые вошли в черную дыру.}

На самом деле, можно ожидать, что состояние всех частиц, возникающих из черной дыры, должно быть связано с состоянием всех частиц, которые образовали черную дыру, с помощью унитарной матрицы рассеяния $S$ [100, 103].

Свойства этой матрицы $S$ могут быть получены с использованием физических аргументов [106]. Используется тот факт, что выходящие частицы должны были пересечь все входящие частицы, и это включает в себя обычную матрицу рассеяния, определяемую их взаимодействиями в <<Стандартной модели>>. Теперь, когда задействованные здесь энергии центров масс часто намного больше, чем что-либо, что было проверено в лабораториях, многие остаются неуверенными в этой теории, но в целом можно сделать некоторые важные выводы:

\textit{Единственные события, которые имеют отношение к реакции черной дыры на проникшие частицы, происходят в области, очень близкой к горизонту событий. Этот горизонт двумерный.}

Это означает, что вся информация, которая обрабатывается вблизи черной дыры, должна быть эффективно представлена на горизонте черной дыры. Выражение Хокинга для энтропии черной дыры, полученной в результате его анализа излучения, ясно указывает на то, что он оставляет только один бит информации на каждый элемент поверхности примерно в квадрате длины Планка. В натуральных единицах:

\begin{equation}\label{9.8}
	S = \pi R^2 = \log W = (\log 2)^2\log W; \quad W = 2^{\Sigma/4\log 2}
\end{equation}
                         
где $\Sigma = 4\pi R^2$ - площадь поверхности горизонта.

Куда делась информация, которая вошла в объем пространства-времени внутри черной дыры? Мы думаем, что она была потеряна. Если мы сформулируем ситуацию таким образом, мы можем получить голографию, не теряя локальности закона физической эволюции. Этот закон эволюции, по-видимому, чрезвычайно эффективен в уничтожении информации.\footnote{Пожалуйста, не путайте это утверждение с вопросом, теряется ли квантовая информация вблизи горизонта черной дыры. Согласно сформулированной здесь гипотезе, квантовая информация - это то, что остается, если мы стираем всю избыточную информацию, объединяя состояния в классы эквивалентности. Микро-состояния черной дыры тогда соответствуют этим классам эквивалентности. По своей конструкции классы эквивалентности не теряются. Недавно было обнаружено, что для этого требуется антиподальная запутанность [127].}

Теперь мы можем добавить к этому, что мы не можем представить себе конфигурацию вещества в пространстве-времени, которая бы содержала больше информации на единицу объема, чем черная дыра с радиусом, соответствующим полной энергии вещества внутри. Поэтому черная дыра подчиняется пределу Бекенштейна [5]:

максимальный объем информации, который помещается в (сферический) объем V, определяется энтропией самой большой черной дыры, которая помещается в V.

Потеря информации в локальной теории поля должна рассматриваться следующим образом («голография»):

\textit{В любой конечной, односвязной области пространства информация, содержащаяся в объеме, постепенно исчезает, но то, что находится на поверхности, будет по-прежнему доступно, так что информация на поверхности может использоваться для характеристики классов информационной эквивалентности.}

На первый взгляд, использование этих классов информационной эквивалентности для представления базовых элементов квантового описания может показаться большим отклонением от нашей первоначальной теории, но мы должны понимать, что, если информация будет потеряна в масштабе Планка, то будет гораздо более трудно потерять какую-либо информацию в гораздо больших масштабах; Есть так много степеней свободы, что полное удаление информации очень трудно и невероятно; Скорее, мы имеем дело с вопросом, как именно представлена информация, и как именно мы считаем биты информации в классах информационной эквивалентности.

Обратите внимание, что на практике, когда мы изучаем материю во вселенной, количество рассматриваемой энергии намного меньше, чем то, что соответствует черной дыре, занимающей весь объем пространства. Таким образом, в большинстве практических случаев предел Бекенштейна несущественен, но мы должны помнить, что в этих случаях мы всегда рассматриваем вещество, которое все еще очень близко к вакуумному состоянию.

Потеря информации является в основном локальной особенностью; во всем мире информация сохраняется. Это означает, что наша идентификация базисных элементов гильбертова пространства с классами инфоэквивалентности оказывается не совсем локальной. С другой стороны, и классическая теория, и квантовая теория, естественно, запрещают распространение информации быстрее скорости света.

Давайте закончим этот раздел нашим взглядом на происхождение излучения Хокинга. Физические законы вблизи горизонта черной дыры должны быть выведены из законов, управляющих состоянием вакуума, видимым падающим наблюдателем. Этот вакуум находится в одном квантовом состоянии, но состоит из множества онтологических состояний, различимых при рассмотрении всех возможных колебания физических полей. Обычно все эти состояния образуют один класс эквивалентности.

Однако на горизонте сигналы, поступающие в черную дыру, не могут возвращаться, поэтому смесь информации, которая приводит к тому, что все эти состояния образуют единый класс эквивалентности, и существенно перегруппировываются присутствием черной дыры, настолько, что удаленный наблюдатель испытывает не один класс эквивалентности, а очень много классов. Таким образом, вакуум заменяется гораздо большим гильбертовым пространством, охватываемым всеми этими классами. Они вместе образуют богатый спектр физических частиц, которые, как видно, выходят из черной дыры.

\end{document}

