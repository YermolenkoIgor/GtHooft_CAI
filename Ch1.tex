\documentclass[main.tex]{subfiles}
\begin{document}


\section{Мотиация}\label{ch1}

Эта книга о теории, и о интерпретации. О теории, в ее нынешнем виде, весьма спекулятивной. Она рождена из неудовлетворенности существующими объяснениями устоявшегося факта, состоящего в том, что наша Вселенная контролируется законами квантовой механики. Квантовая механика выглядит странно, но тем не менее она обеспечивает очень прочную основу для выполнения расчетов всех видов, которые объясняют особенности атомного и субатомного мира. Теория, разработанная в этой книге, исходит из предположений, которые, на первый взгляд, кажутся естественными и понятными, и мы считаем, что они могут быть очень хорошо обоснованными. 

Независимо от того, является ли теория абсолютно правильной, частично правильной или вусмерть неправильной, можно заинтересоваться тем, как она соотносится с квантовой механикой. Мы предполагаем существование определенной "реальности", лежащей в основе квантово-механических описаний. Предположение о существовании этой реальности приводит к довольно приземленной интерпретации того, что нам говорят квантово-механические расчеты. Интерпретация работает красиво и, кажется, удаляет некоторые трудности, возникающие в других описаниях того, как можно интерпретировать измерения и их выводы. Мы предлагаем такое толкование, которое, по нашему мнению, превосходит другие существующие догмы. 

Тем не менее, многочисленные обширные исследования предоставили очень веские доказательства того, что предположения, которые вошли в нашу теорию, не могут быть полностью правильными. Самые ранние аргументы исходили от фон Неймана, но позже они вызывали оживленные обсуждения [6, 15, 49]. Наиболее убедительные аргументы пришли из теоремы Джона С. Белла, сформулированные с точки зрения неравенств, которые должны иметь место для любой локальной классической интерпретации квантовой механики, но сильно нарушаются квантовой механикой. Позже было найдено много других вариаций основной идеи Белла, некоторые из которых были еще более мощными. Мы будем обсуждать их неоднократно и подробно в этой работе. В принципе, все они, казалось, указывают в одном направлении: большинство исследователей утверждает, что законы природы не могут быть описаны локальным, детерминированным автоматом. Так для чего эта книга? 

Существует несколько причин, по которым автор продолжает придерживаться своих предположений. Первая причина заключается в том, что они очень хорошо сочетаются с квантовыми уравнениями различных очень простых моделей. Похоже, что сама природа говорит нам: "Подождите, этот подход не так уж и плох!". Вторая причина заключается в том, что можно рассматривать наш подход просто как первую попытку более реалистического, чем в других существующих подходах, описания природы. Мы всегда можем позже добавить некоторые навороты, вводящие неопределенность, нивелируя конфликт с вышеупомянутыми теоремами; эти ухищрения могут сильно отличаться от того, что ожидают многие эксперты, но в любом случае, мы могли бы выйти из этой борьбы победителями. Возможно, есть тонкая форма нелокальности в клеточных автоматах, возможно, есть какая-то квантовая изюминка в граничных условиях. Почему неравенства Белла должны запрещать мне исследовать этот закоулок? Я вот нахожу это неожиданно интересным. 

Но есть и третья причина. Есть сильное подозрение, что все те "модели со скрытыми переменными", которые сравнивают с мысленными и реальными экспериментами ужасно наивны.\footnote{Действительно, в своем стремлении исключить локальные, реалистичные и/или детерминированные теории, авторы редко утруждаются как следует определить, что эти теориии из себя представляют.} По настоящему детерминированные теории еще не исключены. Это теории из которых следует, что не только все наблюдаемые явления, но и сами наблюдатели контролируются детерминированными законами. Они, конечно, не имеют "свободной воли", их действия все имеют корни в прошлом, даже в далеком прошлом. Позволить наблюдателю иметь свободную волю, то есть допустить возможность независимости выбора от окружающей Вселенной, принципиально невозможно.\footnote{Позже в этой книге мы заменяем «свободную волю» менее эмоциональной, но более точной концепцией, которая, как можно видеть, приводит к тем же очевидным коллизиям, но более податлива к математической обработке.} 

Понятие утверждающее, что действия экспериментаторов и наблюдателей контролируются детерминистскими законами, называется супердетерминизмом. При обсуждении этих вопросов с коллегами у автора сложилось четкое впечатление, что именно здесь у многих срабатывает предвзятость с последубщей выработкой запрещающих теорем.\footnote{Следует уточнить наше понимание термина "детерминизм". Он всегда будет использоваться в том смысле, что: "ничего не происходит по воле случая (не существует абсолютной (объективной) случайности); все физические процессы полностью контролируются законами". Таким образом, основные законы Природы всегда регулируются определенностью, а не вероятностями, в отличие от сегодняшнего понимания квантовой механики. Ни детерминизм, ни "супердетерминизм" не подразумевают "предопределенность" в том смысле, что есть возможность точно предсказывать будущее, поскольку ни один человек и ни одна машина не могут рассчитывать быстрее, чем сама Природа.} Спешим добавить, что мы не первые, кто выдвигает подобные предположения [50 51]. Белл заметил, что супердетерминизм может обеспечить лазейку в его теореме, но, как и большинство современных исследователей, он не повременил, чтобы окрестить ее "абсурдной". Однако, как мы надеемся продемонстрировать, супердетерминизм может оказаться не столь абсурдным, как кажется. В любом случае, осознание такого рода фактов проливает новый свет на наши вопросы, и автор очень воодушевлен на продолжение изысканий. 

Сказав все это, я признаю, что то, что у нас есть, все еще только теория. Она не застрахована от критики, и собственно, очень даже активно ей подвергается. Я знаю, что некоторые читатели не будут убеждены. Если же мне удастся заинтересовать остальных, или даже вызвать у кого-то энтузиазм, то моя цель была достигнута. В несколько худшем сценарии, мои идеи будут просто использоваться в качестве наковальни, на которой другие следователи будут точить свои собственные, высшие взгляды. В то же время, мы разрабатываем математические понятия, которые кажутся последовательными и красивыми. Неудивительно, что мы сталкиваемся и с некоторыми проблемами в формализме, которые мы стараемся как можно точнее сформулировать. Они указывают на то, что проблема генерации квантовых явлений из классических уравнений на самом деле довольно сложна. Трудность в том, что, хотя все классические модели можно переформулировать в некоторые "квантовые системы", которые не имеют локального и ограниченного снизу Гамильтониана. Вполне возможно, что модели, которые производят приемлемые Гамильтонианы, потребуют включения гравитационных эффектов без возмущений, чей формализм в настоящее время недостаточно проработан.

Маловероятно, по мнению автора, что эти сложные схемы можно разгромить всего несколькими строками, как утверждают некоторые.\footnote{В различных разделах этой книги, мы объясняем, что не так с этими «несколькими строками».} Вместо этого, следовало бы прибегнуть к интенсивному расследованию. Как уже говорилось, теория с хорошим фундаментом, как правило, имеет элегантную основу, что можно сказать о интерпретации клеточного автомата квантовой механики. Как будет показано, мы можем получить Гамильтонианы почти что даже локальные и ограниченные снизу. Эти модели похожи на квантованные теории поля, которые, как хорошо известно, также страдают от математических недостатков. Мы утверждаем, что эти недостатки в квантовой теории поля, с одной стороны, и наш способ обработки квантовой механики, с другой, могут быть на самом деле связаны друг с другом. Кроме того, можно задаться вопросом, зачем вообще требовать локальности квантовой модели, пока лежащая в ее основе классическая модель явно локальна по конструкции. Что мы точно подразумеваем под всем этим будет объяснено, в основном в части II, где мы позволяем себе выполнить подробные расчеты.

\subsection{Зачем нужна интерпретация}\label{ch1.1}

Открытие квантовой механики, возможно, было самой важной научной революцией 20-го века. По-видимому не только мир атомов и субатомных частиц полностью управляется правилами квантовой механики, но и миры физики твердого тела, химии, термодинамики, и все радиационные явления могут быть поняты только при соблюдении законов квантов. Успехи квантовой механики феноменальны, и, кроме того, в теории, царит восхитительная и безупречная внутренняя математическая логика. Ничего удивительного в том, что это великое научное достижение привлекло внимание и ученых из других областей, и философов, и общественности в целом. Поэтому, возможно, несколько любопытно, что даже спустя почти целое столетие физики все еще не вполне согласны с тем, что теория говорит нам и чего она нам не говорит о реальности. 

Причина, по которой квантовая механика работает так хорошо, заключается в том, что практически во всех областях ее применения именно то, что означает реальность, оказывается нематериальным. Все, что говорит эта теория, касается реальности результатов эксперимента. Квантовая механика точно говорит нам, чего следует ожидать, как эти результаты могут быть распределены статистически и как они могут быть использованы для вывода деталей внутренних параметров явления. Основным объектом исследований здесь являются элементарные частицы. Была разработана теория \footnote{Взаимозаменяемо, мы используем слово «теория» для самой квантовой механики и для моделей взаимодействия частиц; поэтому, возможно, было бы лучше сослаться на квантовую механику как на структуру, помогающую нам в разработке теорий для подсистем, но мы ожидаем, что использование нами понятия «теория» не должно вызывать путаницы.}, так называемая стандартная модель, которая требует спецификации около 25 внутренних констант природы - параметров, которые не могут быть предсказаны с использованием современных знаний. Большинство этих параметров можно было определить по результатам экспериментов с различной точностью. 

Итак, квантовая механика со всеми ее особенностями по праву считается одним из самых глубоких открытий в области физики, революционизировавшим наше понимание многих особенностей атомного и субатомного мира. 

Но физика еще до конца не раскрыта. Несмотря на некоторые чрезмерно восторженные заявления перед самым началом века, \textit{Теория Всего} еще не была открыта, и есть другие открытые вопросы, напоминающие нам, что физики еще не закончили свою работу. Поэтому, воодушевленные великими достижениями, которые мы наблюдали в прошлом, ученые продолжают идти по пути, который был столь успешным. Разрабатываются новые эксперименты и новые теории, каждая из которых отличается все большей искусностью и изобретательностью. Мы хорошо научились включать каждую часть знаний, полученных в прошлом, в наши новые теории и даже в наши более дикие идеи. 

Но тогда возникает вопрос стратегии. По каким дорогам нам следует идти, если мы хотим поставить на место последние кусочки нашей головоломки? Или даже: как по нашим представлениям эти последние части мозайки будут выглядеть? И в частности: должны ли мы ожидать, что конечная будущая теория будет квантово-механической? 

Именно в этот момент мнения среди исследователей расходятся, как и должно быть в науке, поэтому мы не жалуемся. Напротив, нас вдохновляют поиски с предельной сосредоточенностью именно в тех местах, где никто раньше не брал на себя труд искать. Предметом этой книги является "реальность", стоящая за квантовой механикой. Мы подозреваем, что это может сильно отличаться от того, что можно прочитать в большинстве учебников. Мы на самом деле отстаиваем идею, что это может быть проще, чем все, что можно прочитать в учебниках. Если это действительно так, то это может значительно облегчить наш поиск лучшего теоретического понимания. 

Многие из идей, выраженных и разработанных в этом трактате, очень просты. Очевидно, что мы не первые, кто отстаивает такие идеи. Причина, по которой редко можно услышать об очевидных и простых наблюдениях, которые мы сделаем, заключается в том, что они были сделаны много раз, в недавнем и более древнем прошлом [86], и впоследствии были категорически отвергнуты.

Основная причина, по которой они были отклонены, заключается в том, что они были неудачными; классические детерминистские модели, которые дают те же результаты, что и квантовая механика, были разработаны, адаптированы и модифицированы, но все, что было предпринято, в конечном итоге выглядело намного уродливее, чем оригинальная теория, которая была простой квантовой механикой без лишних преукрас. Квантово-механическая теория, описывающая релятивистские субатомные частицы, называется квантовой теорией поля (см. 20), и она подчиняется таким фундаментальным условиям, как причинность, локальность и унитарность. Требование всех этих желательных свойств было ядром успехов квантовой теории поля, и это в конечном счете дало нам Стандартную модель субатомных частиц. Если мы попытаемся воспроизвести результаты квантовой теории поля в терминах некоторой детерминированной базовой теории, то, по-видимому, придется отказаться по крайней мере от одного из этих требований, что снимало бы большую часть красоты общепринятой теории; гораздо проще этого не делать, и поэтому легче пожертвовать "классичностью". 

Существование классического мира, лежащего в основе квантовой механики, считается не только излишним, но и невозможным. Ничего удивительного в том, что исследователи с презрением морщат носы. Правда для начала им следовало бы доказать невозможность детерминированных моделей, которые были бы предназначены для воспроизведения типичных квантово-механических эффектов. Одним из способов сделать это было обратиться к знаменитому мысленному эксперименту, разработанному Эйнштейном, Подольским и Розеном [33, 53]. Этот эксперимент предполагал, что квантовые частицы связаны не только с волновой функцией; чтобы заставить квантовую механику описать "реальность", казалось, требовались какие-то "скрытые переменные". Можно было доказать, что такие скрытые переменные противоречат друг другу. Мы называем это "теоремой о запрете". Самым известным и самым основным примером была теорема Белла[6]. Как мы уже упоминали, Белл изучил корреляции между измерениями запутанных частиц и обнаружил, что, если начальное состояние для этих частиц выбрано достаточно общим, корреляции, найденные в конце эксперимента, как предсказано квантовой механикой, никогда не могут быть воспроизведены носителями информации, которые транспортируют классическую информацию. Он выразил это в терминах так называемых неравенств Белла, позднее расширенных как неравенство CHSH[20]. Они подчиняются любой классической системе, но сильно нарушаются квантовой механикой. Казалось неизбежным сделать вывод, что мы должны отказаться от производства классических, локальных, реалистических теорий. Их не существует. 

Так зачем же настоящий трактат? Почти каждый день мы получаем письма от физиков-любителей, в которых они объясняют нам, почему существующая наука ошибочна и как, по их мнению, должна выглядеть "Теория всего". Теперь может показаться, что я ступаю по их стопам. Не хочу ли я сказать, что почти сто лет исследований квантовой механики были потрачены впустую? Нисколько. Я настаиваю на том, что последнее столетие исследований привело к великолепным результатам, и единственное, чего не хватало до сих пор, - это более радикального описания того, что было найдено. Не уравнения были неправильными, не технология, а только формулировка того, что часто называют копенгагенской интерпретацией, должна быть заменена. До сих пор теория квантовой механики состояла из набора очень строгих правил относительно того, как амплитуды волновых функций относятся к вероятностям различных результатов эксперимента. Было подчеркнуто, что они не имеют в виду "то, что происходит на самом деле". Не следует спрашивать, что происходит на самом деле, следует довольствоваться предсказаниями относительно результатов экспериментов. Мысль о том, что такой "реальности" вообще не должно существовать, звучит загадочно. Я намерен удалить из квантовой теории все до единой частицы мистики, и мы все равно намерены вывести факты о реальности.

Квантовая механика - один из самых блестящих результатов одного столетия науки, и я не намерен заменять ее каким-то изуродованным вариантом, каким бы незначительным ни было это увечье. Большинство учебников по квантовой механике нигде не нуждаются в малейшем пересмотре, за исключением, пожалуй, тех случаев, когда в них говорится, что вопросы о реальности запрещены. Все практические расчеты по многочисленным ошеломляющим квантовым явлениям можно оставить как есть. Действительно, в довольно многих конкурирующих теориях интерпретации квантовой механики авторы вынуждены вводить нелинейности в уравнение Шредингера или нарушения правила Борна, которые будут недопустимы в этой работе. Что касается "запутанных частиц", то, поскольку известно, как на практике возникают такие состояния, их странное поведение должно быть полностью учтено в нашем подходе. "Коллапс волновой функции" является типичной темой обсуждения, где некоторые исследователи считают, что требуется модификация уравнения Шредингера. Мы также находим удивительно естественные ответы на вопросы, касающиеся "кошки Шредингера" и "стрелы времени". А что касается теорем о запрете, то автор видит некоторые из них, преградами на пути дальнейшего прогресса. А также всегда нужно принимать во внимание предположения и оговорки в модели (как мелкий шрифт в контракте).  

\subsection{Наброски идей раскрытых в части I}\label{ch1.2}

Наша отправная точка будет чрезвычайно простой и понятной, на самом деле настолько, что некоторые читатели могут просто заключить, что я схожу с ума. Однако вопросы такого рода, которые я буду задавать, неизбежно возникают на начальном этапе. Мы начинаем с любой классической системы, которая смутно похожа на нашу Вселенную, с намерением усовершенствовать ее всякий раз, когда это потребуется. Понадобятся ли нам нелокальные взаимодействия? Нужна ли нам потеря информации? Должны ли мы включить какую-то версию гравитационной силы? Или весь проект пойдет наперекосяк? Мы не узнаем, пока не попробуем. Цена, которую мы платим, кажется скромной, но она должна быть упомянута: мы должны выбрать очень специальный набор взаимно ортогональных состояний в гильбертовом пространстве, которые наделены статусом "реальных". Этот набор представлен состояниями, в которых Вселенная может "реально" находиться. Во все времена Вселенная выбирает одно из этих состояний с вероятностью 1, в то время как все остальные состояния имеют вероятность 0. Мы называем эти состояния \textbf{онтологическими состояниями}, и они образуют особый базис гильбертова пространства - \textbf{онтологический базис}. Можно было бы сказать, что это просто формулировка, поэтому эта цена, которую мы платим, является приемлимой, но мы будем считать, что этот очень специальный базис имеет особые свойства.

Это означает, что квантовые теории, с которыми мы в конечном итоге сталкиваемся, образуют совершенно особое подмножество всех квантовых теорий. Таким образом, это может привести к новой физике, поэтому мы считаем, что наш подход заслуживает внимания: в конечном счете, наша цель - не просто переосмысление квантовой механики, но и открытие новых инструментов для построения моделей. Можно было бы ожидать, что наш подход, имеющий столь ненадежную связь как со стандартной квантовой механикой, так и с другими представлениями, касающимися интерпретации квантовой механики, должен быстро привести к противоречиям. Пожалуй, самое замечательное наблюдение, которое можно сделать, что на самом деле все выходит довольно гладко! Можно построить несколько моделей, воспроизводящих квантовую механику без малейших изменений, как будет показано более подробно в Части II все наши модели довольно просты. Многочисленные ответы, которые я получил, утверждающие, что модели, которые я создаю, "почему-то не являются реальной квантовой механикой", просто ошибочны. Модели действительно квантово-механические. Однако я буду первым, кто заметит, что тем не менее можно критиковать наши результаты: модели либо слишком просты, что означает, что они не описывают интересные взаимодействующие частицы, либо они, по-видимому, проявляют более тонкие дефекты. В частности, воспроизведение реалистичных квантовых моделей для локально взаимодействующих квантовых частиц предложенными путями до сих пор оказалось за пределами того, что мы можем сделать. В качестве оправдания я могу только сослаться на то, что это потребует не только воспроизведения полной, перенормируемой теоретической модели квантового поля, но и, кроме того, вполне может потребовать включения идеально квантованной версии гравитационной силы, поэтому никого не должно удивлять, что это трудно.

Были предприняты многочисленные попытки, чтобы найти дыры в аргументах, инициированных Беллом и подтвержденных другими. Большинство этих аргументов о фальсификации были справедливо отклонены. Но теперь настала наша очередь. Зная, какой должна быть локальная структура в наших моделях, и почему мы тем не менее думаем, что они воспроизводят квантовую механику, мы можем теперь попытаться найти причину этого очевидного расхождения. Это ошибка в наших моделях или в аргументах Белла? В чем может быть причина этого несоответствия? Если мы возьмем одну из наших классических моделей, что пойдет не так в эксперименте Белла с запутанными частицами? Были ли сделаны предположения, которые не соответствуют действительности? Может быть, частицы в наших моделях отказываются запутываться? Таким образом, мы надеемся внести свой вклад в продолжающуюся дискуссию. Целью настоящего исследования является разработка фундаментальных физических принципов. Некоторые из них являются почти такими же общими, как фундаментальная каноническая теория классической механики. Способ, которым мы отклоняемся от стандартных методов, заключается в том, что чаще, чем обычно, мы вводим дискретные кинетические переменные. Мы показываем, что такие модели не только имеют много общего с квантовой механикой. Во многих случаях они являются квантово-механическими, но в то же время и классическими. Некоторые из наших моделей занимают область между классической и квантовой механикой, область, которая часто считается пустой. 

Приведет ли это к революционному альтернативному взгляду на то, что такое квантовая механика? Трудности со знаком энергии и локальностью эффективных гамильтонианов в наших теориях до сих пор не разрешены. В реальном мире существует нижняя граница для полной энергии, так называемое состояние вакуума. Связанные с этим тонкости переносятся на Часть II, поскольку они требуют детальных расчетов. Подводя итог: мы подозреваем, что будет несколько способов преодолеть эту трудность, или еще лучше, что она сможет объясненить некоторые очевидные противоречия в квантовой механике. 

Полные и неоспоримые ответы на многие вопросы не даны в этом трактате, но мы возвращаемся к некоторым важным наблюдениям. Как и в других примерах запрещающих теорем, Белл и его последователи на самом деле всего лишь делали предположения, и в их случае предположения также казались совершенно разумными. Тем не менее мы теперь подозреваем, что некоторые из предпосылок, сделанных Беллом, возможно, придется ослабить. Наша теория еще не завершена, и читатель, решительно настроенный против того, что мы здесь пытаемся сделать, вполне может попытаться найти способ безвозвратно разнести теорию. Другие, я надеюсь, будут вдохновлены продолжать идти по этому пути.

Мы предлагаем читателю сделать свои собственные выводы. Мы намерены добиться того, чтобы вопросы, касающиеся более глубоких смыслов квантовой механики, были освещены с новой точки зрения. Подобное было сделано раньше, но большинство моделей, которые я видел, кажутся слишком надуманными, либо требующими существования бесконечного количества вселенных, мешающих друг другу, либо модифицирующих уравнения квантовой механики, в то время как исходные уравнения кажутся красиво когерентными и функциональными. Наши модели предполагают, что Эйнштейн, возможно, был прав, когда возражал против выводов, сделанных Бором и Гейзенбергом. Вполне возможно, что на самом базовом уровне в природе нет случайности, нет фундаментально статистического аспекта законов эволюции. Все, вплоть до мельчайших деталей, управляется неизменными законами. Каждое значительное событие в нашей Вселенной происходит по какой-то причине, оно было вызвано действием физического закона, а не просто случайно. Такова общая картина, передаваемая этой книгой. Мы знаем, что, похоже, неравенства Белла опровергли эту возможность, в частности потому, что мы не готовы отказаться от понятий локальности, поэтому да, они поднимают интересные и важные вопросы, которые мы будем рассматривать на различных уровнях. 

Может показаться, что я использую довольно длинные аргументы, чтобы изложить свою точку зрения.\footnote{Мудрый урок, который следует извлечь из жизненного опыта, состоит в том, что длинные аргументы часто гораздо более сомнительны, чем короткие.} Наиболее существенные элементы наших рассуждений покажутся короткими и простыми, но именно потому, что я хочу, чтобы главы этой книги были самостоятельными, хорошо читаемыми и понятными, здесь и там будут повторяться некоторые аргументы, за которые я приношу свои извинения. Я также прошу прощения за то, что некоторые части расчетов находятся на очень базовом уровне; надеюсь, что это также сделает эту работу доступной для более широкого класса ученых и студентов. 

Наиболее изящным способом обращения с квантовой механикой во всей ее общности является бракет-формализм Дирака (см. 1.6). Мы подчеркиваем, что гильбертово пространство является центральным инструментом физики, а не только квантовой механики. Он может быть применен в гораздо более общих системах, чем стандартные квантовые модели, такие как атом водорода, и он будет использоваться также в полностью детерминированных моделях (мы можем даже использовать его в описании планетной системы Ньютона, см. 5.7.1). 

В любом описании модели сначала выбирается базис в гильбертовом пространстве. Все что нужно - это гамильтониан, для того, чтобы описать динамику. Главной особенностью гильбертова пространства является то, что можно использовать любой базис, который нравится. Переход от одного базиса к другому является унитарным преобразованием, и мы будем часто использовать такие преобразования. 

В части I книги мы описываем философию интерпретации клеточных автоматов (CAI) без слишком большого количества технических расчетов. После введения мы сначала продемонстрируем самый основной прототип модели - модель зубчатого колеса. В главе II. мы начинаем заниматься реальным предметом исследования: вопросом интерпретации квантовой механики. Стандартный подход, называемый копенгагенской интерпретацией, рассматривается очень кратко, подчеркивая те моменты, где нам есть что сказать, в частности неравенство Белла и CHSH. Впоследствии мы сформулируем как можно яснее, что мы подразумеваем под детерминированной квантовой механикой. Интерпретация клеточного автомата квантовой механики должна звучать как богохульство для некоторых квантовых физиков, но это потому, что мы не согласны с некоторыми из общепринятых предположений. Мы заканчиваем часть 3 с одной из наиболее важных фундаментальных идей CAI: наши скрытые переменные действительно содержат "скрытую информацию" о будущем, в частности настройки, которые будут выбраны Алисой и Бобом, но это принципиально нелокальная информация, которую невозможно собрать даже в принципе. Это не должно рассматриваться как нарушение причинно-следственной связи. 

Даже если до сих пор неясно, имеют ли результаты этих корреляций конспиративный характер, можно основывать полезную и функциональную интерпретационную доктрину на предположении, что единственный заговор, который выполняют уравнения, - это обман некоторых современных физиков, при этом действуя в полной гармонии с устоявшимися физическими законами. Процесс измерения и коллапс волновой функции - это две загадки, которые полностью разрешаются этим предположением, как будет показано ниже. 

Мы надеемся вдохновить больше физиков серьезно рассмотреть возможность того, что квантовая механика, как мы ее знаем, не является фундаментальной, таинственной, непроницаемой особенностью нашего физического мира, а скорее инструментом для статистического описания мира, где физические законы, в своих самых основных корнях, вовсе не являются квантово-механическими. Конечно, мы не знаем, как сформулировать самые основные законы в настоящее время, но мы собираем указания на то, что классический мир, лежащий в основе квантовой механики, действительно существует. 

Наши модели показывают, как приостановить квантовую механику, когда мы строим такие модели, как теория струн и "квантовая" гравитация, и это может привести к гораздо лучшему пониманию нашего мира в масштабе планка. Ну, вы не должны принимать все как должное; есть еще нерешенные проблемы, и дальнейшие переулки должны быть исследованы. Они находятся в части \ref{ch9}, где можно увидеть, как различные вопросы появляются в расчетах. Часть II этой книги не предназначена для того, чтобы произвести впечатление на читателя или отпугнуть его. Явные вычисления, выполненные там, показаны для разработки и демонстрации наших инструментов расчета; только некоторые из этих результатов используются в более общих обсуждениях в первой части. Просто пропустите их, если они вам не нравятся.  

\subsection{Философия 19-го века}\label{ch1.3}

Давайте вернемся в 19 век. Представьте себе, что математика была на очень высоком уровне, но ничего из физики 20-го века не было известно. Предположим, кто-то сформулировал детальную гипотезу о том, что его мир - клеточный автомат.\footnote{Одним из таких людей является специалист по технике численных расчетов Е. Фредкин, с которым у меня была плодотворная беседа. Сама идея была, конечно, намного старше [92, 98].} Клеточный автомат будет точно определен в разделе \ref{ch5.1} и в Части II; на данный момент достаточно охарактеризовать его требованием, что состояния, в которых может находиться природа, задаются последовательностями целых чисел. Закон эволюции - это классический алгоритм, который однозначно говорит, как эти целые числа развиваются во времени. И никакой квантовой механики - неслыханная дикость! Закон эволюции достаточно нетривиален, чтобы заставить наш клеточный автомат вести себя как универсальный компьютер [37, 61]. Это означает, что в самом малом масштабе времени и пространства начальные состояния могут быть выбраны таким образом, что с их помощью можно решить любое математическое уравнение. Это означает, что будет невозможно точно определить, как автомат будет вести себя на больших временных интервалах; это будет слишком сложно. 

Математики поймут, что не следует даже пытаться точно определить, каковы будут свойства этой теории в больших временных масштабах и на больших расстояниях, но они могут решить попробовать что-то другое. Можно ли, сделать некоторые статистические заявления о крупномасштабном поведении? 

В первом приближении можно увидеть только белый шум, но при ближайшем рассмотрении система может развить нетривиальные корреляции в своей серии целых чисел; некоторые из корреляционных функций могут быть вычислены точно так же, как они могут быть вычислены в Ван-дер-Ваальсовом газе. Мы не можем строго вычислить траектории отдельных молекул в этом газе, но мы можем получить свободную энергию и давление газа в зависимости от плотности и температуры, мы можем получить его вязкость и другие объемные свойства. Это как раз то, что наши математики 19-го века должны делать со своей моделью клеточного автомата своего мира. В этой книге мы покажем, как физики и математики 20-го и 21-го веков могут сделать еще больше: у них есть инструмент под названием квантовая механика, чтобы вывести и понять еще более сложные детали, но даже они должны будут признать, что точные вычисления невозможны. Единственные эффективные, крупномасштабные законы, применимые в таких задачах, являются статистическими. Можно предсказать средние значения, но не результаты отдельных экспериментов; для этого уравнения эволюции слишком сложны в обращении. 

Короче говоря, наш воображаемый мир 19-го века будет казаться управляемым эффективными законами с большим стохастическим элементом в них. Это означает, что, в дополнение к эффективному детерминированному закону потребуются генераторы случайных чисел, которые принципиально непредсказуемы. На первый взгляд, эти эффективные законы вместе могут выглядеть довольно похожими на квантово-механические законы, которые мы имеем сегодня для субатомных частиц.

Приведенная выше метафора, конечно, не идеальна. Газ Ван-дер-Ваальса подчиняется общим уравнениям состояния, и можно понять, как ведут себя звуковые волны в таком газе, но это не квантовая механика. Можно было бы предположить, что это связано с тем, что микроскопические законы, предполагаемые в основе газа Ван-дер-Ваальса, очень отличаются от клеточного автомата, но неизвестно, может ли этого быть достаточно, чтобы объяснить, почему газ Ван-дер-Ваальса явно не является квантово-механическим. 

Из этого рассуждения следует, что как минимум одна особенность нашего мира не является загадочной: тот факт, что у нас есть эффективные законы, требующие стохастического элемента в виде эдакого генератора случайных чисел - это то, чему мы не должны удивляться. Наши физики 19-го века были бы довольны тому, что им дают их математики, и они были бы полностью готовы к выводам физиков 20-го века, которые бы подразумевали, что действительно эффективные законы, управляющие атомами водорода, содержат стохастический элемент, например, чтобы определить, в какой именно момент возбужденный атом решает испустить фотон. 

На самом деле здесь кроется более глубокая философия того, почему у нас есть квантовая механика: не все особенности клеточного автомата в основе нашего мира позволяют экстраполировать его свойства на большие масштабы. Очевидно, что изложение этой главы полностью нетехническое, и оно может быть плохим представлением всех тонкостей теории, которую мы сегодня называем квантовой механикой. Тем не менее, мы думаем, что в достаточной мере озвучили некоторые элементы истории, которую мы хотим рассказать. Наши физики 19-го века могли бы получить эффективную квантовую теорию для своей автоматной модели мира, если бы у них был доступ к нынешнему математическому аппарату. Смогут ли физики 19-го века проводить эксперименты с запутанными фотонами? Этот вопрос мы откложим на потом. 

Философствуя о различных поворотах, которые мог бы выбрать ход истории, представьте себе следующее. В 19 веке теория атомов уже существовала. Атомы - кванты материи, можно было бы рассматривать как первый успешный шаг физиков к дискретизации мира. Однако энергия, импульсы и угловые импульсы все еще считались непрерывными. Разве не было бы естественно подозревать, что они также являются дискретными? В нашем мире, это понимание пришло с открытием квантовой механики. Но даже сегодня пространство и время сами по себе остаются строго непрерывными сущностями. Когда же мы обнаружим, что все в физическом мире в конечном счете будет дискретным? Это был бы дискретный, детерминированный мир, лежащий в основе наших нынешних теорий, как рекламировал, среди прочего, Фредкин. В этом сценарии квантовая механика, как мы ее знаем сегодня, является несовершенной логикой, возникающей в результате неполной дискретизации.\footnote{Когда я пишу это, я ожидаю многочисленных писем от любителей (дилетантов), но будьте осторожны, так как было бы легко предложить какую-то полностью дискретную смесь, но очень трудно найти правильную теорию, которая поможет нам понять мир с использованием строгой математики.}

\subsection{Краткая история клеточного автомата}\label{ch1.4}

Клеточный автомат - это математическая модель физической системы, которая сводит физические переменные к дискретным целым числам, определенным на одномерной или многомерной сетке. Места на сетке называются "ячейками", позиции которых обозначаются серией целых чисел - координатами сетки. В такт времени все переменные в ячейках этой сетки обновляются, так что они зависят от времени. Правило, по которому они обновляются, отражает законы физики. Для каждой ячейки, как правило, обновленные значения зависят только от значений, которые ячейка имела ранее, и содержимого соседних ячеек. Это свойство мы называем "локальность". Самое раннее упоминание о такой концепции было сделано Джоном фон Нейманом[87] и Станиславом Уламом[84]. Обоих интересовал вопрос, как в физическом мире с простыми законами эволюции могут возникать структуры, которые воспроизводят себя: возникновение жизни. Это было в 1940-х годах. Однако клеточные автоматы обрели популярность в 1970-х годах, когда Джон Конвей [39, 40] предложил интересный пример автомата на двумерной сетке, названный игрой жизни Конвея. Правила эволюции или "законы физики" для этой системы были очень просты. Сетка была прямоугольной, так что каждая ячейка имела 4 ближайших соседей плюс 4 ближайших к ближайшим, разделенных по диагонали. Данные в каждой ячейке могут принимать только два значения: 0 и 1. Удобно эти два состояния клетки назвать "мертвым" и "живым". При каждом ударе часов каждая клетка возобновлялась следующим образом:

\begin{itemize}
\item Каждая живая клетка, у которой живы менее 2 из 8 соседей, умирает (от одиночества, наверное);
\item Каждая живая клетка, у которой живы 2 или 3 соседа, живет до следующего поколения;
\item Каждая живая клетка, у которой живы более 3 соседей, умирает (из-за перенаселения);
\item Любая мертвая клетка с тремя живыми соседями становится живой (репродукция \sout{(но почему три)})
\end{itemize} 

Начальное состояние можно задать произвольно. С каждым ударом часов (тик времени) каждая клетка возобновляется в соответствии с вышеприведенными правилами. За эволюцией всей системы можно следить бесконечно. В принципе, предполагается, что сетка имеет бесконечный размер, но можно было также рассмотреть любой тип граничных условий. 

Игра стала популярной, когда Мартин Гарднер описал ее в октябрьском номере журнала Scientific American за 1970 год [39, 40]. В то время физики могли наблюдать эволюцию таких автоматов на компьютерах, и они заметили, что "игра жизни" может служить примитивной моделью развивающейся вселенной с живыми существами в ней. Было обнаружено, что некоторые структуры, если они окружены пустыми ячейками, будут стабильными или периодическими. Другие конструкции, называемые ``планерами'' или ``космическими кораблями'', будут двигаться по горизонтальным, вертикальным или диагональным траекториям. 

Таким образом, было обнаружено, что относительно простые, первичные законы физики могут привести к сложности, и некоторые утверждали, что может возникнуть Вселенная с "сознанием" и "свободной волей". Правила клеточных автоматов были разделены на классы, чтобы различать, глобальные свойства: некоторые системы автоматов быстро эволюционировали в стабильные или невыразительные конечные состояния, некоторые быстро приводили к совершенно хаотичным конечным структурам. Наиболее интересные клеточные автоматы эволюционировали бы в узнаваемые структуры с возрастающей сложностью. Предполагалось, что эти автоматы применимы для выполнения сложных вычислений.

Большинство наиболее интересных примеров, таких как "игра жизни", не являются обратимыми во времени, поскольку многие различные начальные паттерны могут привести к одному и тому же конечному состоянию. Это делает их менее интересными для физики на первый взгляд, поскольку на атомном уровне большинство физических законов обратимы во времени. Большинство моделей, изученных в нашей книге, также обратимы во времени. Однако позже в нашем исследовании мы увидим важность необратимости времени в клеточных автоматах для физики, так что модели, такие как "игра жизни", снова войдут в картину. Большинство членов интересного класса 4 не являются обратимыми во времени, и это еще одна причина подозревать, что необратимость времени может добавить интересную форму стабильности нашим системам, что может повысить их значение для физики. Подробнее о необратимости времени в гл. 7. 

Клеточные автоматы часто используются в качестве моделей для физических систем, таких как жидкости или другие сложные смеси частиц. Однако был также интерес к использованию клеточных автоматов в качестве физических теорий. Может ли быть так, что физика, на ее самом изначальном уровне, основана на дискретных законах? В 1967 году эта идея была впервые выдвинута Конрадом Цузе [97] в его книге Rechnender Raum (вычисление пространства), где было высказано предположение, что вся Вселенная является результатом детерминированного закона вычисления в автомате. Действительно, эта идея не казалась такой уж безумной, учитывая тот факт, что фундаментальные частицы, по-видимому, ведут себя как отдельные биты информации, бегающие вокруг. В частности, фермионы выглядят как биты, когда они записаны в координатном представлении. 

Понятие было сформулировано как "it from Bit" Джоном Арчибальдом Уиллером [89, 90], что является идеей о том, что частицы материи вполне могут отождествляться с передаваемой ими информацией, которая в свою очередь необходима для их описания. 

Обширное исследование роли клеточных автоматов как моделей для решения научных вопросов было сделано Стивеном вольфрамом в его книге "Новый вид науки" [92]. Он придавал своему подходу особую философию. Поскольку клеточные автоматы обладают сложностью и вычислительной универсальностью, общими со многими моделями физических систем, Вольфрам предполагает, что эксперименты с клеточными автоматами сами по себе могут выявить многие особенности таких физических систем. У читателя может сложиться впечатление, что наша книга является продолжением новаторской работы Вольфрама, но у нас пока нет таких амбиций. Классы моделей, рассмотренные Вольфрамом, вполне могут быть слишком ограничительными для наших целей, и, кроме того, наш основной вопрос очень конкретно относится к происхождению квантово-механических явлений. 

И Цузе, и Вольфрам уже предположили, что квантово—механическое поведение должно быть объяснено в терминах клеточных автоматов, но не пытаясь добраться до сути того - как именно мы объясняем квантовую механику в терминах клеточного автомата? Нужен ли нам очень специальный автомат или каждый автомат рано или поздно производит квантово-механическое поведение? Ученые тесно занимающиеся информатикой изучили многие особенности клеточных автоматов, которые не будут использованы в этой работе; это связано с тем, что эти вопросы связаны с совершенно особыми начальными состояниями, в то время как квантовая механика заставит нас рассматривать прежде всего общие состояния.

\subsection{Современные представления о квантовой механике}\label{ch1.5}

Открытие законов квантовой механики сильно повлияло на то, как исследователи теперь думают о "реальности". Даже такие авторитеты, как Ричард Фейнман, были озадачены: "я думаю, что могу с уверенностью сказать, что никто сегодня не понимает квантовую механику" [36]. Меньше всего сомнений вызывал только факт, что теория полностью последовательна, и она удивительно хорошо согласуется с экспериментами. Было бы неплохо, если бы можно было найти закон эволюции для клеточного автомата, который генерирует частицы Стандартной модели и характеристики их взаимодействий, но большинство исследователей сегодня считают маловероятным, что мы скоро сможем идентифицировать такую систему. Можно сформулировать в информационном ключе: наши частицы представляют собой информацию, которая передается и обрабатывается. Сегодня мы воспринимаем эти процессы как квантово-механическую информацию: суперпозиции собственных состояний операторов, называемых наблюдаемыми. Если бы одна система носителей информации могла быть точно преобразована в другую систему носителей информации, с другими правилами обработки этой информации, то мы никогда не смогли бы решить, какая из этих систем является более "фундаментальной". Следовательно, мы могли бы закончить с классами клеточно-автоматных систем, на том, что мы не можем решить, какой элемент в одном конкретном классе представляет наш мир. Дэвид Дойч [28] формулирует эту ситуацию в своей теории конструктора. Ключевым моментом является различимость физических систем. 

Предложение, представленное в этой книге, состоит в том, что по крайней мере один элемент в таких классах должен оказаться классическим автоматом, но этот шаг обычно не делается. Чаще всего оказывается, что интерпретация "многих миров" кажется неизбежной [88]. Кроме того, идея о том, что нелинейные модификации уравнения Шредингера, независимо от того, насколько они малы, понадобятся для объяснения коллапса волновой функции, по-прежнему сохраняется. Матрица плотности, вычисленная из уравнения Шредингера, содержит недиагональные члены, и независимо от того, насколько быстро они могут колебаться или насколько нестабильны фазы этих членов, кажется, что чего-то не хватает, чтобы полностью стереть их. Мы покажем, что это не так в нашей теории. Опрос, проведенный A. Zeilinger et al. [74] относительно позиций, занятых участниками конференции по основам квантовой механики, был довольно показательным. Хотя, возможно, сами вопросы были несколько предвзятыми, оказалось, что большинство разошлось во мнениях по поводу точной формулировки, которую нужно выбрать, но соглашается с тем, что квантовая информация принципиально отличается от классической информации. Никто из участников не верил в основополагающую детерминистскую теорию. Большинство из них считали, что Эйнштейн в своей критике Боровской формулировки квантовой механики просто ошибался. 

В этой книге мы надеемся убедить читателя в том, что детерминистское обоснование вовсе не является невозможным, и, хотя Нильс Бор был прав в прагматическом смысле, в Копенгагенскую доктрину необходимо внести поправки. С высоты птичьего полета версия взглядов, разработанных в этой книге, была представлена в [109]. Другие, предварительные экскурсии настоящего автора описаны в [101, 119, 120] и [125]. Практически все исследователи [22, 23] придерживаются концепции свободы выбора, которая означает, что наблюдатель в любое время должен обладать свободой выбора, какое наблюдаемое свойство системы должно быть измерено. Цейлингер [94] утверждает, что эта свобода может быть гарантирована в экспериментах. Однако в нашей книге мы видим, что эта свобода выбора может быть сильно ограничена из-за очень сильных пространственных корреляций. Тщательно определив, что именно означает свобода выбора, мы заменим "свободную волю" чем-то математически более точным. Затем мы удостоверимся, что, хотя все наблюдатели в данный момент времени действительно имеют свободу выбора своих настроек, корреляционные функции при этом диктуют, нелокально, какими могут быть онтологические состояния наблюдаемых объектов, таких как элементарные фотоны. Короче говоря, выбор, сделанный наблюдателем, должен будет соответствовать корреляционным функциям, наложенным физическими законами. Законы являются локальными, а корреляционные функции - нет. Мы увидим, как эти корреляционные функции могут повлиять на наши выводы относительно тайн квантовой механики.  

\subsection{Примечания}\label{ch1.6}

В большинстве частей этой книги квантовая механика будет использоваться как набор инструментов, а не как теория. Наша теория может быть какой угодно; одним из наших инструментов будет гильбертово пространство и математические манипуляции, которые могут быть сделаны в этом пространстве. Хотя мы предполагаем, что читатель знаком с этими понятиями, мы кратко резюмируем, что такое гильбертово пространство. 

Гильбертово пространство $\mathcal{H}$ является комплексным\footnote{Некоторые критически настроенные читатели задавались вопросом, откуда должны взяться комплексные числа в квантовой механике, учитывая тот факт, что мы начинаем с классических теорий. Ответ прост: комплексные числа - это не что иное, как рукотворные изобретения, как и реальные числа. В гильбертовом пространстве они полезны всякий раз, когда мы обсуждаем что-то, что сохраняется во времени (например, барионное число), и когда мы хотим диагонализировать гамильтониан. Заметим, что квантовую механику можно сформулировать и без комплексных чисел, если принять, что гамильтониан является антисимметричной матрицей. Но тогда его собственные значения являются мнимыми. Мы подчеркиваем, что мнимые числа в основном используются для математики, и по этой причине они необходимы для физики.} векторным пространством, число измерений которого обычно бесконечно, но иногда мы допускаем, что это конечное число. Его элементы называются состояниями, обозначаемыми как $\left|\psi\right>$, $\left|\phi\right>$ или любой другой "Кет". У нас есть линейность: когда $\left|\psi_1\right>$ и $\left|\psi_2\right>$ являются состояниями в нашем гильбертовом пространстве, то

\begin{equation}\label{1.1}
	\left|\phi\right> = \lambda \left|\psi_1\right> + \mu\left|\psi_2\right>
\end{equation}

где $\lambda$ и $\mu$ комплексные числа, описывающие состояние в этом гильбертовом пространстве. Для каждого кет-состояния $\left|\psi\right>$ мы имеем сопряженное состояние $\left<\psi\right|$, находящийся в соряженном векторном пространстве $\left<\psi\right|$, $\left<\phi\right|$. Принимая во внимание (\ref{1.1}), можно записать

\begin{equation}\label{lincombconj}
	\left<\phi\right| = \lambda^* \left<\psi_1\right| + \mu^*\left<\psi_2\right|
\end{equation}

Кроме того, у нас есть скалярное произведение:

\begin{align}\label{inproduct}
	\left<\chi\right| \left( \lambda \left|\psi_1\right> + \mu\left|\psi_2\right> \right) = \lambda\left<\chi\mid\psi_1\right> + \mu\left<\chi\mid\psi_2\right> \\
	 \left<\chi\mid\psi\right> = \left<\psi\mid\chi\right>^*
\end{align}

Скалярное произведение кет состояния $\left|\psi\right>$ с его бра вещественное и положительное:

\begin{align}\label{1.2}
	||\psi||^2 \equiv \left<\psi\mid\psi\right> = real\ge 0,\\
	while \,\, \left<\psi\mid\psi\right> = 0 \,\leftrightarrow \, \left|\psi\right> = 0 
\end{align}

Поэтому скалярное произведение может быть использовано для определения нормы. Состояние $\left|\psi\right>$ называется физическим или нормализованным, если

\begin{equation}\label{norma}
	||\psi||^2 = \left<\psi\mid\psi\right> = 1
\end{equation}

Так как сочетание "физическое состояние" будет сбивать с толку, мы будем называть это состояние \textit{шаблонным}. Вся мощь нотации Дирака будет раскрыта во второй части.

Переменные иногда будут числами, а иногда операторами в Гильбертовом пространстве. В особо важных случаях внимание на этом будет заостряться, например, с помощью крышечки "$\hat O$". Таким образом определим \textit{матрицы Паули} $\vec\sigma = (\sigma_x, \sigma_y, \sigma_z)$:

\begin{equation}\label{norma}
	\hat \sigma_x = \begin{pmatrix}
 0&1 \\ 
 1&0 
\end{pmatrix}, \,\,
\hat \sigma_y = \begin{pmatrix}
 0&-i \\ 
 i&0 
\end{pmatrix}, \,\,
\hat \sigma_z = \begin{pmatrix}
1 &0 \\ 
 0& -1
\end{pmatrix}, \,\,
\end{equation}



\end{document}