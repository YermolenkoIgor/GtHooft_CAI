\documentclass[main.tex]{subfiles}
\begin{document}


\section{A Summary of Our Views on Conformal Gravity}

Whenever a fundamental difficulty is encountered in handling deterministic versions of quantum mechanics, we have to realize that the theory is intended in particular to apply to the Planck scale, and that is exactly where the gravitational force cannot be ignored. Gravity causes several complications when one tries to discretize space and time. One is the obvious fact that any regular lattice will be fundamentally flat, so we have to address the question where the Riemann curvature terms can come from. Clearly, we must have something more complicated than a regular lattice. A sensible suspicion is that we have a discretization that resembles a glassy lattice.

But this is not all. We commented earlier on the complications caused by having non-compact symmetry groups. The Lorentz group generates unlimited contractions both in the space- and in the time direction. This is also difficult to square with any lattice structure.1

Furthermore, gravity generates black hole states. The occurrence of stellar-sized black holes is an unavoidable consequence of the theory of General Relativity. They must be interpreted as exotic states of matter, whose mere existence will have to be accommodated for in any “complete” theory of Nature. It is conceivable that black holes are just large-size limits of more regular field configurations at much smaller scales, but this is also far from being a settled fact.Many theories regard black holes as fundamentally topologically distinct from other forms of matter such as the ones that occur in stars that are highly compressed but did not, or not yet, collapse. To make a link to any kind of cellular automaton (thinking of the glassy types, for instance), it seems reasonable first to construct a theory of gravity where space–time, and the fields defined on it, are topologically regular. Consider the standard Einstein–Hilbert action with a standard, renormalizable field theory action for matter added to it:

$$
\begin{aligned}
\mathcal{L}^{\text {total }}=& \sqrt{-g}\left(\mathcal{L}^{\mathrm{EH}}+\mathcal{L}^{\text {matter }}\right) \\
\mathcal{L}^{\mathrm{EH}}=& \frac{1}{16 \pi G_{N}}(R-2 \Lambda) \\
\mathcal{L}^{\text {matter }}=&-\frac{1}{4} G_{\mu \nu}^{a} G_{\mu \nu}^{a}-\bar{\psi} \gamma^{\mu} D_{\mu} \psi-\frac{1}{2}(D \varphi)^{2}-\frac{1}{2} m_{\varphi}^{2} \varphi^{2}-\frac{1}{12} R \varphi^{2} \\
&-V_{4}(\varphi)-V_{3}(\varphi)-\bar{\psi}\left(y_{i} \varphi_{i}+i y_{i}^{5} \gamma^{5} \varphi_{i}+m_{d}\right) \psi
\end{aligned}
$$
Here, $\Lambda$ is the cosmological constant, $\varphi$ stands for, possibly more than one, scalar matter fields, $V_{4}$ is a quartic interaction, $V_{3}$ a cubic one, $y_{i}$ and $u_{i}^{5}$ are scalar and pseudo-scalar Yukawa couplings, $m_{\varphi}$ and $m_{d}$ are mass terms, and the term $-\frac{1}{12} R \varphi^{2}$ is an interaction between the scalar fields and the Ricci scalar $R$ that is necessary to keep the kinetic terms for the $\varphi$ field conformally covariant. Subsequently, one rewrites
$$
\begin{aligned}
g_{\mu v} &=\omega^{2}(\vec{x}, t) \hat{g}_{\mu v}, & \varphi &=\omega^{-1} \hat{\varphi} \\
\psi &=w^{-3 / 2} \hat{\psi}, & \sqrt{-g} &=\omega^{4} \sqrt{-\hat{g}}
\end{aligned}
$$
and substitutes this everywhere in the total Lagrangian (B.1). This leaves a manifest, exact local Weyl invariance in the system:
$$
\begin{aligned}
\hat{g}_{\mu \nu} & \rightarrow \Omega^{2}(\vec{x}, t) \hat{g}_{\mu v}, & \omega & \rightarrow \Omega(\vec{x}, t)^{-1} \omega \\
\hat{\varphi} & \rightarrow \Omega(\vec{x}, t)^{-1} \hat{\varphi}, & \hat{\psi} \rightarrow \Omega(\vec{x}, t)^{-3 / 2} \hat{\psi}
\end{aligned}
$$

The substitution (B.2) turns the Einstein-Hilbert Lagrangian into
$$
\mathcal{L}^{\mathrm{EH}}=\frac{1}{16 \pi G_{N}}\left(\omega^{2} \hat{R}-2 \omega^{4} \Lambda+6 \hat{g}^{\mu v} \partial_{\mu} \omega \partial_{v} \omega\right)
$$
Rescaling the $\omega$ field: $\omega=\tilde{\kappa} \chi, \tilde{\kappa}^{2}=\frac{4}{3} \pi G_{N},$ turns this into
$$
\frac{1}{2} \hat{g}^{\mu v} \partial_{\mu} \chi \partial_{v} \chi+\frac{1}{12} \hat{R} \chi^{2}-\frac{1}{6} \tilde{\kappa}^{2} \Lambda \chi^{4}
$$
The resemblance between this Lagrangian for the $\chi$ field and the kinetic term of the scalar fields $\varphi$ in Eq. (B.1), suggests that no singularity should occur when $\chi \rightarrow 0,$ but we can also conclude directly from the requirement of exact conformal invariance that the coupling constants should not run, but keep constant values under (global or local) scale transformations. Note, that $\chi \rightarrow 0$ describes the smalldistance limit of the theory.

The theory was originally conceived as an attempt to mitigate the black hole information paradox $[111,112],$ then it was found that it could serve as a theory that determines the values of physical parameters that up to the present have been theoretically non calculable (this should follow from the requirement that all renormalization group functions $\beta_i$ should cancel out to be zero [112]). 

For this book, however, a third feature may be important: with judiciously chosen conformal gauge-fixing procedures, one may end up with models that feature upper limits on the amount of information that can be stowed in a given volume, or 4-volume, or surface area.




\subsection{}
\subsection{}
\subsection{}
\subsection{}
\subsection{}
\subsection{}
\subsection{}
\subsection{}




\begin{equation}\label{}
	
\end{equation}








\end{document}

