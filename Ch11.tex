\documentclass[main.tex]{subfiles}
\begin{document}


\section{Введение в часть II}\label{ch11}

Многие из технических расчетов и аргументов, были перенесены во вторую часть, чтобы облегчить чтение первой, сохраняя ее целостность. Цена, которую мы платим за это, состоит в том, что будет много повторений, за которые мы приносим свои извинения.

\subsection{План части II}\label{ch11.1}

Одна из наших основных тем заключается в том, что квантовая механика может рассматриваться как математический инструмент, а не как новая теория физических явлений. Действительно, в теории конденсированного состояния существует несколько моделей, в которых физическая установка и задаваемые вопросы являются в основном классическими, однако расчеты выполняются, с рассмотрением системы как квантово-механической. Двумерная модель Изинга является прекрасным примером [56].

Нет лучшего способа проиллюстрировать наш подход, чем на деле показать, как выполняются такие расчеты. Модель зубчатого колеса уже была представлена в разд. \ref{ch2.2}. Теперь в частях \ref{ch12.1}-\ref{ch13}, мы покажем еще несколько наших математических инструментов для построения квантовых гамильтонианов и достижения континуального предела. Здесь модель зубчатого колеса связана с гармоническим ротатором, и другими, общеизвестными «классическими» структурами, такими как планетная система, преобразующимися в модели, которые кажутся квантово-механическими.

Континуальный предел одного периодического зубчатого колеса является важным примером. Он стремится к обычному квантовому гармоническому осциллятору с тем же периодом $T$. Зубчатое колесо в континуальном пределе на самом деле является плавно вращающимся колесом. Эквивалентны ли эти классические вращающиеся колеса квантовому гармоническому осциллятору? В некотором смысле, да, но есть некоторые тонкости, которые нужно осознавать. Вот почему мы решили сделать этот предел в два этапа: сначала преобразовать зубчатое колесо в гармонический ротатор, позволив зубьям сформировать представление группы SU(2), и только потом рассматривать континуальный предел. Это позволяет нам распознавать операторы $x$ и $p$ подлинного гармонического осциллятора уже в конечных зубчатых колесах.

Как и другие технические расчеты в других местах этой книги, они были сделаны для того, чтобы проверить внутреннюю согласованность исследуемых систем. Было забавно делать эти вычисления, но они не предназначены, чтобы обескуражить читателя. Просто пропустите их, если вас больше интересует общая картина.

Вопрос о локальности гамильтониана более подробно рассматривается в гл. \ref{ch14}. Он часто встречается практически в любой детерминистической модели, и опять-таки нас ждет интересная математика. Мы видим, что многое зависит от построения вакуумного состояния. Это состояние с наименьшей энергией, и решение задачи минимизации энергии действительно порождает нелокальности. В действительности, как хорошо известно в квантовых теориях поля, сигналы не будут идти быстрее скорости света. В этой главе будет показано, что существует способ избежать нелокальности, когда объекты движутся в окружении вакуума, при условии, что используется теория первого квантования, в которой используется только центральная часть энергетического спектра. Следовательно, энергия может быть положительной или отрицательной. Впоследствии вводятся античастицы, так что состояния с отрицательной энергией фактически представляют дыры античастиц. Это всего лишь уловка Дирака, чтобы гарантировать, что физический вакуум имеет минимально возможную энергию.

Дирак впервые сформулировал свою теорию для фермионных частиц. Действительно, в этом отношении фермионы легче понять, чем бозоны. Поэтому сначала мы вводим фермионы как существенный элемент в наших моделях, см. гл. \ref{ch15}.

Так получилось, что уравнение Дирака для электрона хорошо подходит для демонстрации нашего рецепта поиска «beables» в квантовой теории. Раздел \ref{ch15.2} также начинается в легком темпе, но заканчивается длинными выводами. Здесь также читателю предлагается насладиться сложными особенностями модели «нейтрино», но их также можно пропустить.

Возьмем упрощенный случай уравнения Дирака для двухкомпонентного нейтрино. Это существенно проще, чем уравнение Дирака для электрона. Кроме того, мы предполагаем отсутствие взаимодействий. Математика начинается с простого, но результат поразителен: нейтрино - это конфигурации плоских мембран или «листов», а не частиц. Листы движутся классически. Это не теория, а математический факт, пока мы скрываем массовые термины и взаимодействия; они должны быть оставлены на потом.

Наблюдая это, мы задали вопрос, как перейти от листовых переменных обратно к квантовым операторам нейтрино, таким как положение $\vec x$, импульс $\vec p$ и спин $\vec \sigma$. Здесь математика усложняется, и она интересна как упражнение (разделы \ref{ch15.2.1} и \ref{ch15.2.2}). Нейтрино идеальны для применения вторичного квантования (раздел \ref{ch15.2.3}), хотя на этом языке мы пока не можем ввести взаимодействия для них.

Наши модели, обсуждаемые в гл. \ref{ch12} -\ref{ch17} и \ref{ch19} похожи тем, что они локальны, реалистичны и основаны на обычных процедурах в физике. Общим для них также является то, что они ограничены по объему, они не охватывают все признаки, о которых известно, что они существуют в реальном мире, такие как все виды частиц, все группы симметрии и, в частности, специальная и общая теория относительности. Модели должны быть предельно прозрачными, они указывают направления, на которые нужно смотреть, и, как и было нашей основной целью, они предлагают отличный подход к интерпретации квантово-механических законов, которые все нам так знакомы.

Теория PQ, гл. \ref{ch16} - первая попытка понять связи между теориями, основанными на действительных числах, и теориями, основанными на целых или дискретных числах. Идея состоит в том, чтобы установить чистый формализм, соединяющий оба, так чтоб он мог быть использован во многих случаях. Глава \ref{ch16} также показывает некоторые хорошие математические особенности, с использованием эллиптических тэта-функций. Расчеты выглядят более сложными, чем они должны быть, просто потому, что мы искали элегантный механизм, связывающий действительную линию с парами целых чисел с одной стороны и тором с другой, сохраняя симметрию между координатами и импульсами.

В гл.\ref{ch17}, мы находим некоторые другие интересные расширения того, что было сделано в гл. \ref{ch16}. Очень простой аргумент привлек наш интерес к теории струн и теории суперструн. Мы не сильно поддерживаем идею, что единственный способ сделать интересную физику в масштабе Планка - это верить тому, что говорят нам теоретики струн. Из нашей работы не ясно, что такие теории являются правильным путем, но заметим, что наша программа демонстрирует замечательные связи с теорией струн. В отсутствие взаимодействий, локальные уравнения теории струн и суперструн, по-видимому, позволяют строить beables точно по маршруту, который мы защищаем. Наиболее поразительная особенность, раскрытая здесь, заключается в том, что квантованные струны, записанные в обычной форме непрерывных квантовых теорий поля в одном пространстве и одном временном измерении, отображаются в классических струнных теориях, которые определены не в непрерывном целевом пространстве, а в пространстве-времени. решетка, где расстояние решетки $a$ определяется как $a = 2\pi\sqrt{\alpha'}$,

Симметрии, обсуждаемые в гл. \ref{ch18}, трудно понять в интерпретация КА квантовой механики. Однако в ИКА соображения симметрии так же важны, как и где-либо еще в физике. Большинство наших симметрий дискретны, но в некоторых случаях, особенно в теории струн, непрерывные симметрии, такие как группа Пуанкаре, могут быть восстановлены.

В гл. \ref{ch19}, мы рассматриваем проблему положительности гамильтониана с другой точки зрения. Там обычный гамильтонов формализм распространяется на дискретные переменные, опять же в пары $P_i$, $Q_i$, развивающиеся в дискретном времени. Когда мы впервые попытались изучить это, это казалось кошмаром, но так получилось, что «дискретный формализм Гамильтона» оказался почти таким же элегантным, как и обычная дифференциальная форма. И действительно, здесь может быть легко выбран гамильтониан ограниченный снизу.

В конце концов, мы хотим воспроизвести эффективные законы Природы, которые должны принять форму современных квантовых теорий поля, что довольно сложно. Это было причиной настройки наших процедур формальным образом, чтобы мы сохраняли гибкость в адаптации наших систем к тому, что, по-видимому, говорит нам Природа через многочисленные гениальные эксперименты, которые были проведены. Мы объясним некоторые из наиболее важных особенностей квантовой теории поля в гл. \ref{ch20}. В частности, в квантовых теориях поля никакой сигнал не может передавать полезную информацию быстрее скорости света, а вероятности всегда составляют единицу. Квантовая теория поля является полностью локальной, в своем собственном неповторимом квантовом способе. Эти особенности мы хотели бы воспроизвести в детерминированной квантовой теории.

Чтобы более подробно настроить интерпретацию клеточного автомата, мы сначала разработаем некоторые технические вопросы, касающиеся клеточных автоматов в целом (глава \ref{ch21}). Это не технические особенности, возникающие при написании компьютерных программ для таких систем; эксперты программного обеспечения не поймут большую часть нашего анализа. Это потому, что мы стремимся понять, как такие системы могут генерировать квантовую механику при очень большом временном и дистанционном ограничении, и как мы можем быть в состоянии соединиться с физикой элементарных частиц. Мы находим красивое выражение для квантового гамильтониана в терминах разложения, называемого разложением БКХ (Формула Бейкера-Кэмпбелла-Хаусдорфа). Все было бы идеально, если бы это было конвергентное расширение.

Однако легко увидеть, что расширение не сходится. Мы пробуем ряд альтернативных подходов с некоторыми скромными успехами, но не все проблемы будут решены, и возникло подозрение относительно источника наших трудностей: квантовые гравитационные эффекты могут иметь решающее значение, хотя именно эти эффекты все еще остаются не поняты в должной мере. Мы предлагаем использовать разложение БКХ для многих классов клеточных автоматов, чтобы продемонстрировать, как их можно использовать для интерпретации квантовой механики. Я знаю, что детали еще не совсем верны, но это, вероятно, связано с тем простым фактом, что мы опустили много вещей, в частности, специальную и общую теорию относительности

\subsection{Обозначения}\label{ch11.2}

Трудно сделать нашу запись совершенно однозначной. В гл. \ref{ch16}, мы имеем дело со многими различными типами переменных и операторов. Когда динамическая переменная является целым числом, мы будем использовать заглавные буквы $A, B, \ldots, P, Q,\ldots$.             

Одические переменные с периодом $2\pi$, или, по крайней мере, ограниченные интервалами, такими как $(—\pi, \pi]$, представляют собой углы, в основном обозначаемые греческими строчными буквами $\alpha, \beta, \ldots, \theta, \ldots$, тогда как действительные переменные чаще всего обозначаются строчными латинскими буквами $a, b,\ldots, х, у, \ldots$. Но иногда у нас заканчиваются символы и мы отклоняемся от этой схемы, если это кажется безвредным. Например, индексы все равно будут $i, j,\ldots$ для пространственноподобных компонент вектора, $\alpha, \beta,\ldots$ для спиноров и $\mu, \nu, \ldots$ для индексов Лоренца. Греческие буквы $\phi$ и $\psi$ также будут использоваться для волновых функций.

Тем не менее, трудно сохранить нашу запись полностью последовательной; в некоторых главах перед гл. \ref{ch16}, мы используем квантовые числа $l$ и $m$ представлений $SU$ (2) для обозначения целых чисел, которые ранее были обозначены как $k$ или $k - m$, а затем в гл. \ref{ch16} заменены на прописные.

Как и в первой части, операторы будут обозначаться крашечкой $(\hat{ } )$ или явно ($ ^\mathrm{op}$) чтобы отличить от обычной числовой переменной. Символ каретки $(\hat{ } )$ также будет зарезервирован для векторов единичной длины (или спиноров), стрелка - для более общих векторов, необязательно ограниченных трехмерным пространством. Только в гл. \ref{ch20}, где нормы векторов не возникают, мы используем каретку для преобразования Фурье функции \textbf{заменить на тильды}.

Константа Дирака $\hbar$ и скорость света $c$ почти всегда будут определяться равными единице в выбранных единицах. В предыдущей работе мы использовали пространственный символ для обозначения $e^{2\pi}$ в качестве альтернативного базиса для экспоненциальных функций. Это действительно иногда полезно для вычислений, когда мы используем дроби, которые лежат между 0 и 1, а не углами, и это потребовало бы, чтобы мы нормализовали исходную постоянную Планка $h$, а не $\hbar$ к единице, но в настоящей монографии мы вернемся к более обычным обозначениям.

Часто обсуждаемые понятия:
\begin{itemize}
	\item Дискретные переменные - это переменные, такие как целые числа, возможные значения которых могут быть подсчитаны. В противоположность непрерывным переменным, которые обычно представлены действительными или комплексными числами.
    
    \item Дробные переменные - это переменные, которые принимают значения в конечном интервале или по кругу. Интервал может быть $[0, 1)$, $[0, 2\pi)$, $(-\frac 1 2, \frac 1 2]$ или $(-\pi, \pi]$. Здесь квадратная скобка указывает границу, значение которой может быть включено, а круглая скобка исключает это значение. Вещественное число всегда можно разложить на целое (или дискретное) число и дробное.

    \item Теория онтологическая или «онтическая», если она описывает только «реально существующие» объекты; это просто классическая теория, такая как планетная система, в отсутствие квантовой механики. Теория не требует введения гильбертова пространства, хотя, как будет объяснено, гильбертово пространство может быть очень полезным. Но тогда теория сформулирована в терминах наблюдаемых, которые всегда коммутируют.

    \item Явление является контрфактуальным, когда предполагается, что оно существует, даже если по фундаментальным причинам оно не может быть косвенно замечено, если его не будут пытаться наблюдать, что приведет к тому что другое явление может перестать быть наблюдаемым и, следовательно, станет контрфактуальным. Такая ситуация обычно возникает, если рассмотреть измерение двух или более операторов, которые не коммутируют. Чаще в наших моделях мы сталкиваемся с явлениями, которые не могут быть контрфактуальными.

    \item Мы говорим о шаблонах, когда описываем частицы и поля как решения уравнения Шредингера в онтологической модели, как было объяснено в разд. \ref{ch4.3.1} . Шаблоны могут быть суперпозициями онтических состояний и / или других шаблонов, но все онтические состояния образуют ортонормированный набор; суперпозиции онтических состояний сами по себе никогда не являются онтическими.
\end{itemize}


\subsection{Подробнее о нотации Дирака для квантовой механики}\label{ch11.3}

Счетный набор состояний $\mid e_i \rangle$ называется ортонормированным базисом $\mathcal H$, если каждое состояние $\mid \psi \rangle \in \mathcal H$ может быть аппроксимировано линейной комбинацией конечного числа состояний $\mid e_i \rangle$ с любой требуемой точностью:

\begin{equation}\label{11.1}
	|\psi\rangle=\sum_{i=1}^{N(\varepsilon)} \lambda_{i}\left|e_{i}\right\rangle+|\varepsilon\rangle, \quad\|\varepsilon\|^{2}=\langle\varepsilon | \varepsilon\rangle<\varepsilon^{2}, \quad \text { for any } \varepsilon>0
\end{equation}

(свойство называется «полнота»), в то время как

\begin{equation}\label{11.2}
	\left\langle e_{i} | e_{j}\right\rangle=\delta_{i j}
\end{equation}
        
(это назыввается ортонормированностью). Из (\ref{11.1}) и (\ref{11.2}) 

\begin{equation}\label{11.3}
	\lambda_{i}=\left\langle e_{i} | \psi\right\rangle, \quad \sum_{i}\left|e_{i}\right\rangle\left\langle e_{i}\right| = \mathbb{I}
\end{equation}

где $\mathbb{I}$ - тождественный оператор: $\mathbb{I}\mid \psi \rangle = \mid \psi \rangle$ для всех $\mid \psi \rangle$.
Во многих случаях дискретная сумма в уравнениях (\ref{11.1}) и (\ref{11.3}) будут заменены интегралом, и дельтой Кронекера $\delta_{ij}$ в уравнении (\ref{11.2}) с помощью дельта-функции Дирака $\delta(x^1 - x^2)$. Мы все еще будем называть состояния $\mid e_{(x)} \rangle$ базисом, хотя он не счётный.

Типичным примером является набор волновых функций $\mid \psi(\vec x) \rangle$, описывающих частицу в позиционном пространстве. Они рассматриваются как векторы в гильбертовом пространстве, где в качестве базиса выбран набор волновых функций дельта-пика $\mid \vec x \rangle$:

\begin{equation}\label{11.4}
	\psi(\vec{x}) \equiv\langle\vec{x} | \psi\rangle, \quad\left\langle\vec{x} | \vec{x}^{\prime}\right\rangle=\delta^{3}\left(\vec{x}-\vec{x}^{\prime}\right)
\end{equation}
             
Преобразование Фурье теперь представляет собой простое вращение в гильбертовом пространстве или переход к базису импульса:

\begin{equation}\label{11.5}
	\langle\vec{x} | \psi\rangle=\int \mathrm{d}^{3} \vec{p}\langle\vec{x} | \vec{p}\rangle\langle\vec{p} | \psi\rangle ; \quad\langle\vec{x} | \vec{p}\rangle=\frac{1}{(2 \pi)^{3 / 2}} e^{i \vec{p} \cdot \vec{x}}
\end{equation}
                          
Многие специальные функции, такие как функции Эрмита, Лагерра, Лежандра и Бесселя, можно рассматривать как порождающие различные наборы базисных элементов гильбертова пространства.

Часто мы используем произведения гильбертовых пространств: $\mathcal{H}_1\bigotimes \mathcal{H}_2 = \mathcal{H}_3$, что означает, что состояния $\mid \phi\rangle$ в $\mathcal{H}_3$ можно рассматривать как нормальные произведения состояний $\left|\psi^{(1)}\right\rangle$ в $\mathcal{H}_1$ и $\left|\psi^{(2)}\right\rangle$ в $\mathcal{H}_2$

\begin{equation}\label{11.6}
	|\phi\rangle=\left|\psi^{(1)}\right\rangle\left|\psi^{(2)}\right\rangle
\end{equation}

и базис для $\mathcal{H}_3$ может быть получен путем объединения базиса в $\mathcal{H}_1$ с базисом в $\mathcal{H}_2$:
 
\begin{equation}\label{11.7}
	\left|e_{i j}^{(3)}\right\rangle=\left|e_{i}^{(1)}\right\rangle\left|e_{j}^{(2)}\right\rangle
\end{equation}

Часто некоторые или все из этих умножаемых гильбертовых пространств являются конечномерными векторными пространствами, которые, конечно, также допускают все манипуляции.\footnote{Термин гильбертово пространство часто ограничивается применением только к бесконечномерным векторным пространствам; здесь мы также включим конечномерные случаи.} Например, мы имеем двумерное векторное пространство, охватываемое частицами со спином 1. Базис образован двумя состояниями $\mid\uparrow\rangle$ и $\mid\downarrow\rangle$. В нем, матрицы Паули $\hat\sigma_{x,y,z}$ определены так же, как в части I, (\ref{1.7}). Состояния
 
\begin{equation}\label{11.8}
	|\rightarrow\rangle=\frac{1}{\sqrt{2}}\left(\begin{array}{l}{1} \\ {1}\end{array}\right), \quad|\leftarrow\rangle=\frac{1}{\sqrt{2}}\left(\begin{array}{c}{1} \\ {-1}\end{array}\right)
\end{equation}
 
формируют базис, где оператор $\sigma_x$ диагонален: $\hat\sigma_x\rightarrow\bigl(\begin{smallmatrix}
 1& 0\\ 
 0&-1 
\end{smallmatrix}\bigr)$             
Дирак получил слова «бра» и «кет» из того факта, что ожидаемое значение для оператора $\mathcal{O}^{\mathrm{op}}$ можно записать как оператор между скобками, или

\begin{equation}\label{11.9}
\langle\mathcal{O}^{\mathrm{op}}\rangle = \langle\psi\mid\mathcal{O}^{\mathrm{op}}\mid\psi\rangle
\end{equation}

         
В более общем случае нам понадобятся матричные элементы оператора в базисе $\left\{\mid e_i\rangle\right\}$:

\begin{equation}\label{11.10}
\mathcal{O}_{ij} = \langle e_i\mid\mathcal{O}^{\mathrm{op}}\mid e_j\rangle
\end{equation}


Преобразование из одного базиса $\left\{\left|e_{i}\right\rangle\right\}$ в другой, $\left\{\left|e_{i}^{\prime}\right\rangle\right\}$ является унитарным оператором $U_{i j}$ :
$$
\begin{aligned}\label{11.11}
\left|e_{i}^{\prime}\right\rangle &=\sum_{j} U_{i j}\left|e_{j}\right\rangle, \quad U_{i j}=\left\langle e_{j} | e_{i}^{\prime}\right\rangle \\
\sum_{k} U_{i k} U_{j k} &=\sum_{k}\left\langle e_{i}^{\prime} | e_{k}\right\rangle\left\langle e_{k} | e_{j}^{\prime}\right\rangle=\delta_{i j}
\end{aligned}
$$
что будет использоваться часто. Например, преобразование Фурье является унитарным:

\begin{equation}\label{}
\int \mathrm{d}^{3} \vec{p}\langle\vec{x} | \vec{p}\rangle\left\langle\vec{p} | \vec{x}^{\prime}\right\rangle=\frac{1}{(2 \pi)^{3}} \int \mathrm{d}^{3} \vec{p} e^{i \vec{p} \cdot \vec{x}-i \vec{p} \cdot \vec{x}^{\prime}}=\delta^{3}\left(\vec{x}-\vec{x}^{\prime}\right)
\end{equation}

Уравнение Шредингера будет представлено в виде

\begin{equation}\label{11.13}
\begin{aligned}
\frac{\mathrm{d}}{\mathrm{d} t}|\psi(t)\rangle &=-i H^{\mathrm{op}}|\psi(t)\rangle, & \frac{\mathrm{d}}{\mathrm{d} t}\left\langle\psi(t)\left|=\langle\psi(t)| i H^{\mathrm{op}}\right.\right.\\
|\psi(t)\rangle &=e^{-i H^{\mathrm{op}} t}|\psi(0)\rangle
\end{aligned}
\end{equation}


где $H^{\text {op }} $ - гамильтониан, определяемый его матричными элементами $H_{i j}=\left\langle e_{i}\left|H^{\text {op }}\right| e_{j}\right\rangle$

Обозначение Дирака может быть использовано для описания нерелятивистских волновых функций в трех пространственных измерениях, в позиционном пространстве, в импульсном пространстве или в некотором другом базисе, таком как разложение по парциальным волнам, которое может использоваться для частиц со спином. Оно может использоваться для систем множества частиц, а также для квантованных полей в теории твердого тела или в теории элементарных частиц. Переход от нотации пространства Фока, где базис охватывает состояния, содержащие фиксированное число N частиц (в координатном или в импульсном пространстве, возможно, также имеющих спин), к нотации, где базис охватывает функции, представляющие поля этих частиц, что есть просто вращение в гильбертовом пространстве от одного базиса к другому.




\end{document}

