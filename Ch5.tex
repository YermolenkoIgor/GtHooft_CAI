\documentclass[main.tex]{subfiles}
\begin{document}

\section{Краткое описание интерпретации КА}\label{ch5}

Конечно, мы предполагаем, что читатель знаком с ,,представлением Шредингера'', а также с ,,представлением Гейзенберга'' традиционной квантовой механики. Мы будем попеременно использовть оба представления.

\subsection{Обратимые по времени клеточные автоматы }\label{ch5.1}

Модель зубчатого колеса, которая, как описано в разд. \ref{ch2.2}, может быть применена для описания зеемановского атома с одинаково расположенными энергетическими уровнями. Это прототип автомата. Все наши детерминированные модели можно охарактеризовать как <<автоматы>>. Клеточный автомат - это автомат, в котором данные представляются для формирования дискретной d-мерной решетки в n = d + 1-мерном пространстве-времени. Элементы решетки называются <<ячейками>>, и каждая ячейка может содержать ограниченный объем информации. Данные $Q (\vec x, t)$ в каждой ячейке $(\vec x, t)$ могут быть представлены целым числом или набором целых чисел, возможно, но не обязательно ограниченным максимальным значением $N$. Закон эволюции переписывает значения ячеек в момент $t + 1$, используя значения для момента времени $t$ (или $t$ и $t-1$).\footnote{В последнем случае, онтологический базис должен состоять из данных в двух последовательных временных слоях.} Как правило, закон эволюции для данных в ячейке в пространстве-времени

\begin{equation}\label{5.1}
	(\vec{x}, t), \quad \vec{x}=\left(x^{1}, x^{2}, \ldots, x^{d}\right), \quad x^{i}, t \in \mathbb{Z}
\end{equation}
         
будет зависеть только от данных в соседних ячейках в $(\vec x', t-1)$ и, возможно, в $(\vec x', t-2)$. Если в этом законе эволюции соотношение $|| \vec x '- \vec x ||$ ограничено некоторой границей, то говорят, что клеточный автомат подчиняется локальности.
Кроме того, клеточный автомат считается обратимым во времени, если данные из прошлых ячеек могут быть восстановлены из данных в более поздние времена, и если правилом для этого является также клеточный автомат. Обратимость во времени можно легко гарантировать, если принять закон эволюции

\begin{equation}\label{5.2}
	Q(\vec{x}, t+1)=Q(\vec{x}, t-1)+F(\vec{x},\{Q(t)\})
\end{equation}

где $Q(\vec{x}, t)$ представляет данные в конкретной точке $(\vec{x}, t)$ решетки пространства-времени, а $F(\vec{x},\{Q(t)\})$ - некоторая заданная функция данных всех ячеек, соседних с точкой x в момент времени t. Здесь $+$ обозначает сложение, сложение по модулю некоторого целого числа N или некоторую другую простую, обратимую операцию в пространстве переменных $Q$. Конечно, тогда мы имеем обратимость времени:             


\begin{equation}\label{5.3}
	Q(\vec{x}, t-1)=Q(\vec{x}, t+1)-F(\vec{x},\{Q(t)\})
\end{equation}
            
где $-$ обратная операция к $+$.

Простая модель зубчатого колеса допускает как классическое, так и квантово-механическое описание, без каких-либо изменений физики. Теперь мы можем сделать то же самое для клеточного автомата. Классические состояния, в которых может находиться автомат, рассматриваются как ортонормированный набор базисных элементов онтологического базиса. Оператор эволюции $\hat U(\delta t)$ для одного временного шага, длительность которого составляет $\delta t$, является унитарным оператором, так что все его собственные состояния $|E_i\rangle$ унимодулярны:

\begin{equation}\label{5.4}
	U_{\mathrm{op}}(\delta t)\left|E_{i}\right\rangle= e^{-i \omega_{i}}\left|E_{i}\right\rangle, \quad 0 \leq \omega<2 \pi
\end{equation}

и можно найти оператор $\hat H$ такой, что

\begin{equation}\label{5.5}
	U_{\mathrm{op}}(\delta t)= e^{-i \hat H \delta t}, \quad 0 \leq \hat H<2 \pi/\delta t
\end{equation}

Однако можно свободно добавлять целые кратные $2\pi / \delta t$ к любому из собственных значений этого гамильтониана, не меняя выражения для $\hat U$, так что в определении $\hat H$ есть много свободы. Можно добавить произвольные фазовые углы к собственным состояниям, $|E_i\rangle \rightarrow e^{\mathrm i \phi_i}|E_i\rangle$, и эти модификации могут также зависеть от возможных сохраняющихся величин. Ясно, что можно немного изменить гамильтониан, не повредив его полезности для генерации оператора эволюции $\hat U$. В подразделе \ref{ch2.2.2} раздела \ref{ch2.2} показано, как довольно сложные энергетические спектры могут возникать таким образом в относительно простых обобщениях модели зубчатого колеса.

Подобные модификации в гамильтониане вполне могут потребоваться, если мы хотим отразить свойство локальности клеточного автомата в гамильтониане:

\begin{equation}\label{5.6}
	\hat H \stackrel{?}{=}\sum_{\vec{x}} \hat{\mathcal{H}}(\vec{x}), \quad\left[\hat{\mathcal{H}}\left(\vec{x}^{\prime}\right), \hat{\mathcal{H}}(\vec{x})\right] \rightarrow 0 \quad \text { if }\left\|\vec{x}^{\prime}-\vec{x}\right\| \gg 1
\end{equation}

но это важный математический вопрос, подчиняется ли гамильтониан формуле (\ref{5.6})? Этот вопрос был бы актуален, даже если классический клеточный автомат являлся бы локальным в смысле, описанном выше. Эта проблема обсуждается далее в разд. \ref{ch5.6.1}, в разд. \ref{ch9.1}, и во второй части - главы. 14 и 22. Там мы увидим, ситуация может стать довольно сложной. Очень хорошие приближения могут существовать для некоторых систем клеточных автоматов с возможно некоторыми измененными локальными свойствами и гамильтонианом, который приблизительно подчиняется принципу локальности, как в уравнении (\ref{5.6}).

Обратите внимание, что гамильтониан, который подчиняется (\ref{5.6}) в сочетании с инвариантностью Пуанкаре, будет соответствовать полностью перенормированной квантовой теории поля со всеми ее сложностями, и это гарантирует, что найти полное математическое решение набросанных задач будет непросто.


\subsection{ТеКА и ИнКА}\label{ch5.2}

Теория клеточного автомата (CAT) предполагает, что после того, как будет идентифицировано универсальное уравнение Шредингера, которое охватывает все мыслимые явления во вселенной (Великая единая теория или Теория для всего), она будет иметь онтологическую основу (базис), которая отображает систему в классический автомат. Вполне возможно, что вполне вероятно, что истинный автомат вселенной будет более сложным, чем <<обычный>> клеточный автомат, как показано здесь, но он вполне может иметь некоторые из своих основных характеристик. Правдоподобность теории должна быть выяснена в ходе дальнейших исследований.

Также трудно предвидеть как симметрии Природы будут отражены в этих классических правилах; Трудно представить, как Лоренц-инвариантность и инвариантность диффеоморфизма могут быть реализованы в этих классических правилах. Вероятно, они будут относиться к более общим квантовым базисам.

Тогда эта теория, кажется, является тем, что обычно называют "теорией скрытых переменных". Действительно, в некотором смысле наши переменные скрыты; если существуют преобразования симметрии, которые преобразуют наш базис в другой, который диагонализирует различные операторы, то для нас будет почти невозможно определить, какой из них является <<истинным>> онтологическим базисом, и поэтому у нас будут разные кандидаты в ,,скрытые переменные'', которые будет невозможно отличить на практике.

Интерпретация клеточных автоматов [104] считает само собой разумеющимся, что эта теория верна, даже если мы никогда не сможем явно идентифицировать какой-либо онтологический базис. Мы предполагаем, что шаблоны, используемые в настоящее время в квантовой механике, должны рассматриваться как суперпозиции онтологических состояний, и что классические состояния, которые описывают результаты наблюдений и измерений, являются классическими распределениями онтологических состояний. Если два классических состояния различимы, их распределения не имеют общего онтологического состояния (см. Рис. \ref{i4.1}, б). Вселенная находится в одном онтологическом состоянии, а не в суперпозиции таких состояний, но всякий раз, когда мы используем наши шаблоны (то есть, когда мы выполняем обычные квантово-механические вычисления), мы используем суперпозиции только потому, что они математически удобны. Обратите внимание, что поскольку суперпозиции являются линейными, наши шаблоны подчиняются тому же уравнению Шредингера, что и онтологические состояния.

В принципе, переход от обычного квантового базиса к онтологическому базису может быть довольно сложным. Только собственные значения гамильтониана не будут затронуты, а также в детерминированных моделях они могут формировать довольно общие спектры, см. Рис. \ref{i2.3}c. На языке теории групп гамильтонианы, которые мы получаем путем преобразования в различные базисные выборы, образуют один класс сопряженности, характеризуемый набором собственных значений.

В конце концов, наша квантовая система должна быть непосредственно связана с некоторой квантовой теорией поля в предельном континууме. Мы опишем квантовые теории поля в части II, гл. \ref{ch20}. На первый взгляд может показаться очевидным, что гамильтониан должен принимать форму уравнения (\ref{5.6}), но мы должны помнить, что определения гамильтониана (\ref{2.8}), (\ref{2.18}), (\ref{2.26}) и выражения, показанные на рис. \ref{i2.3}, хорошо определены только по модулю $2\pi/\delta t$, так что когда различные не взаимодействующие системы объединены, их гамильтонианы не обязательно должны складываться.

Эти выражения показывают, что гамильтониан $\hat H$ может быть выбран разными способами, так как можно добавить любой оператор, который коммутирует с  $\hat H$. Поэтому разумно ожидать, что можно определить гамильтониан подчиняющийся уравнению (\ref{5.6}). Подход более подробно объясняется в части II, гл. \ref{ch21}, но некоторая неопределенность сохраняется, и сходимость процедуры, даже если мы ограничиваемся состояниями с низкой энергией, далеко не очевидна.

Как несложно заметить, гамильтониан, подчиняющийся уравнению (\ref{5.6}) является суммой слагаемых, каждое из которых конечно и ограничено как снизу, так и сверху. Такой гамильтониан должен иметь основное состояние, то есть собственное состояние $|\O\rangle$ с наименьшим собственным значением, которое можно нормализовать до нуля. Это собственное состояние следует отождествлять с "вакуумом". Этот вакуум является стационарным, даже если сам автомат может не иметь стационарного решения. Следующее самое высокое собственное состояние может быть состоянием из одной частицы. В картине Гейзенберга поля $Q (\vec x, t)$ будут вести себя как операторы $\hat Q (\vec x, t)$, когда мы перейдем к базису шаблонных состояний, и поэтому они могут создавать одночастичные состояния из вакуума. Таким образом, мы приходим к чему-то, что напоминает подлинную квантовую теорию поля. Состояния являются квантовыми состояниями в полном соответствии с копенгагенской интерпретацией. Поля $\hat Q (\vec x, t)$ должны подчиняться аксиомам Вайтмана [55,79]. Квантовые теории поля будут дополнительно обсуждаться в гл. \ref{ch20}.

Однако, если мы начнем с любого клеточного автомата, есть три отличия полученной теории от обычных квантовых теорий поля. Одним из них является, конечно, то, что пространство и время дискретны. Ну, может быть, есть интересный ,,предел континуума'', в котором масса(-ы) частицы (-ц) значительно меньше, чем обратная величина кванта времени, но, если наши модели не выбраны очень тщательно, это не будет иметь место.

Во-вторых, общий клеточный автомат даже отдаленно не будет инвариантным по Лоренцу. Мало того, что одночастичные состояния не смогут проявить лоренц-инвариантность или даже инвариантность Галилея; состояния, в которых частицы могут двигаться относительно вакуумного состояния, будут полностью отличаться от статических одночастичных состояний. Кроме того, симметрия вращения будет уменьшена до некоторой дискретной группы вращения решетки, если вообще получится до чего-либо ее свести. Итак, знакомые симметрии релятивистских квантовых теорий поля будут полностью отсутствовать.

В-третьих, не ясно, может ли клеточный автомат быть связан с одной квантовой моделью или, возможно, многими неэквивалентными. Добавление или удаление других консервативных операторов в $\hat H$ аналогично добавлению терминов химического потенциала. В отсутствие лоренц-инвариантности будет трудно разделять различные типы <<вакуумных>> состояний, которые можно получить.

По всем этим причинам большинство моделей клеточных автоматов будут сильно отличаться от квантованных теорий поля для элементарных частиц. Однако основной вопрос, обсуждаемый в этой книге, заключается не в том, легко ли имитировать Стандартную модель в клеточном автомате, а в том, можно ли получить квантовую механику и нечто, похожее на квантовую теорию поля, по крайней мере, в принципе. Происхождение непрерывных симметрий Стандартной Модели останется за пределами того, с чем мы можем справиться в этой книге, но мы можем обсудить вопрос, в какой степени клеточные автоматы могут использоваться для аппроксимации и понимания квантовой природы этого мира. Смотрите наше обсуждение особенностей симметрии в части II, раздел \ref{ch18}.

Как будет объяснено позже, вполне возможно, что инвариантность при общих преобразованиях координат будет ключевым компонентом в объяснении непрерывных симметрий, поэтому вполне может быть, что окончательное объяснение квантовой механики также потребует полного решения проблемы квантовой гравитации. Не будем притворяться, что решили ее.

Многие из других моделей в этой книге будут явно интегрируемыми. Клеточные автоматы, с которых мы начали в первом разделе этой главы, показывают, что наша общая философия также применима к неинтегрируемым системам. Однако обычно считается, что обратимые по времени клеточные автоматы могут быть универсально вычислительными [37, 61]. Это означает, что любой такой автомат может быть организован в специальные подмножества состояний, которые подчиняются правилу любого другого вычислительного универсального клеточного автомата. Тогда можно было бы возразить, что любой вычислительный универсальный клеточный автомат может использоваться для имитации таких сложных систем, как Стандартная модель субатомных частиц. Однако в этом случае, будучи физиками, мы бы предпочли одну единственную специальную модель, которая более эффективна, чем любая другая, так что любой выбор начального состояния в этом автомате описывает физически реализуемую конфигурацию.


\subsection{Мотивация}\label{ch5.3}

Не слишком надумано ожидать, что однажды квантовая гравитация будет полностью формализована, то есть будет сформулирована краткая теория, которая показывает герметичное описание соответствующих физических степеней свободы, и будет предложена простая модель, которая показывает как эти физические переменные развиваются. Для этого нам может даже не потребоваться обычная переменная времени, но нам нужен недвусмысленный рецепт, говорящий нам о том, как физические степени свободы будут выглядеть в области пространства-времени, которая лежит в будущем области, описанной в более раннее время.

Полная теория, объясняющая квантовую механику, вероятно, не может быть сформулирована без учета квантовой гравитации, но мы можем сформулировать наше предложение и установить язык, который необходимо будет использовать. Сегодня наше описание молекул, атомов, полей и релятивистских субатомных частиц перемежается с волновыми функциями и операторами. Наши операторы не коммутируют с операторами, описывающими другие аспекты того же мира, и мы научились не удивляться этому, а просто выбирать набор базовых элементов, как нам угодно, угадывать разумно выглядящее уравнение Шредингера и вычислять то, что нужно найти, когда мы проводим измерения. Нам сказали не спрашивать, что такое реальность, и это оказалось полезным советом: мы рассчитываем и видим, что вычисления имеют смысл. Маловероятно, что какие-либо другие наблюдаемые аспекты полей и частиц когда-либо будут вычислены, это никогда не будет больше, чем то, что мы можем извлечь из квантовой механики. Например, изучая радиоактивную частицу, мы не можем точно рассчитать, в какой момент она распадется. Распад контролируется формой случайности, которую мы не можем контролировать, и эта случайность кажется гораздо более совершенной, чем та, которая может быть получена с использованием запрограммированных псевдослучайных последовательностей. Нам велено отказаться от всякой надежды перехитрить Природу в этом отношении.

Клеточно автоматная интерпретация (CAI) подсказывает нам, что мы на самом деле делаем, когда решаем уравнение Шредингера. Мы думали, что следуем бесконечному набору различных миров, каждому с определенной амплитудой, и конечные события, которые мы выводим из наших расчетов, зависят от того, что происходит во всех этих мирах. Это, по мнению ИнКА, иллюзия. Нет бесконечности разных миров, есть только один, но мы используем <<неправильный>>  базис, чтобы описать его. Слово <<неправильный>>  здесь используется не для того, чтобы критиковать отцов-основателей квантовой механики, которые сделали удивительные открытия, а для того, чтобы повторить то, что они, конечно, также обнаружили то, что базис, который они используют, не является онтологическим. Терминология, используемая для описания этого базиса, не раскрывает нам, как именно наш мир, наш единственный мир, <<действительно>> развивается во времени.

Существует много других интерпретаций квантовой механики. Они могут либо считать, что бесконечные числа разных миров все являются реальными, либо они требуют какой-то модификации, или скорее, искажения, чтобы понять, как волновая функция может разрушиться, чтобы произвести измеренные значения некоторых наблюдаемых, не допуская таинственных суперпозиций в классическом масштабе.

CAI предлагает использовать полный математический механизм, который разрабатывался годами для решения квантово-механических явлений. Он включает в себя именно копенгагенские предписания, чтобы перевести расчеты в точные прогнозы, с использованием экспериментальных данных, поэтому на данный момент, безусловно, никаких изменений не требуется.

Однако есть один важный принцип, в котором мы отклоняемся от Копенгагена. По мнению Копенгагена, некоторые вопросы не следует задавать:

\textit{Может ли быть так, что наш мир - это всего лишь один мир, где все происходит, согласно уравнениям эволюции, которые могут быть существенно проще, чем уравнение Шредингера, и есть ли способы узнать об этом? Можно ли убрать элемент статистического распределения вероятностей из основных законов квантовой механики?}

Согласно Копенгагену, не существует экспериментов, которые могли бы ответить на такие вопросы, и поэтому глупо даже задавать их. Но в этой работе я попытался показать, что не экспериментально, а теоретически мы можем найти ответы. Мы можем быть в состоянии идентифицировать модели, которые описывают единый классический мир, даже если он слишком сложен по сравнению с тем, к чему мы привыкли, и мы можем быть в состоянии идентифицировать его физические степени свободы с некоторыми квантовыми переменными, о которых мы уже знаем.

Клеточный автомат, описанный в предыдущих разделах, будет примером прототипа; это сложно, но вполне возможно, не достаточно сложно. У него есть симметрии, но в реальном мире существуют гораздо более крупные группы симметрии, такие как группа Лоренца или Пуанкаре, показывающие отношения между различными видами событий, и их, по общему признанию, трудно реализовать. Группы симметрии (представьте себе симметрию трансляции пространства-времени) на самом деле могут быть корнями таинственных особенностей, которые были обнаружены в нашем квантовом мире, так что они могут иметь естественные объяснения.

Зачем нам нужен только один мир с классическими уравнениями, описывающими его эволюцию? Что плохого в том, чтобы повиноваться изречению Копенгагена о постановке вопросов, если они все равно не будут экспериментально доступны?

По мнению автора, мотивы будут подавляющими: если классическая модель существует, она значительно упростит наше представление о мире, поможет нам раз и навсегда понять, что же на самом деле происходит, когда выполняется измерение и когда волновая функция "коллапсирует". Это раз и навсегда объяснит парадокс шредингеровской кошки.

Еще более важно то, что квантовые системы, которые допускают классическую интерпретацию, образуют чрезвычайно малое подмножество всех квантовых моделей. Если действительно верно, что наш мир попадает в этот класс, который можно считать вероятным в рамках всего вышеизложенного, то это настолько ограничивает наш набор допустимых моделей, что может позволить нам угадать правильную. Так что то, что мы действительно ищем, это новый подход к предположению, что такое "Теория мира". Мы сильно подозреваем, что без этого превосходного руководства мы никогда даже не приблизимся. Таким образом, наша реальная мотивация заключается не в том, чтобы лучше предсказать результаты экспериментов, которые могут произойти не скоро, а скорее в том, чтобы предсказать, какой класс моделей мы должны тщательно изучить, чтобы узнать больше об истине.

Подчеркнем еще раз, что это означает, что CAI / CAT будет в первую очередь иметь значение, если мы хотим расшифровать законы Природы в самых фундаментальных масштабах времени и расстояния, то есть в масштабе Планка. Таким образом, важный новый рубеж, где империя кванта встречается с классическим миром, провозглашается близким к планковским измерениям. Как мы также неоднократно указывали, CAI требует переформулировки нашего стандартного квантового языка также при описании другой важной границы: границы между квантовой империей и <<обычным>> классическим миром в масштабах расстояний, превышающих размеры атомов и молекул.

Как будет объяснено, CAI на самом деле имеет больше общего с оригинальной копенгагенской доктриной, чем многие другие подходы. Это покончит со ,,многими мирами'', более радикально, чем интерпретация Де Бройля Бома. CAI предполагает существование одной или нескольких моделей Природы, которые еще не были обнаружены. Мы обсудим несколько игрушечных моделей. Эти игрушечные модели недостаточно хороши, чтобы приблизиться к стандартной модели, но есть основания надеяться, что однажды такая модель будет найдена. В любом случае, CAI будет применяться только к крошечному подклассу всех квантово-механических моделей, обычно рассматриваемых для объяснения наблюдаемого мира, и по этой причине мы надеемся, что было бы полезно определить правильную процедуру, чтобы прийти к правильной теории.
Другие модели были представлены в этой работе, просто чтобы показать набор инструментов, которые можно выбрать для использования.

\subsubsection{Волновая функция Вселенной}\label{ch5.3.1}

Стандартная квантовая механика может поставить перед нами вопрос, на который, кажется, трудно ответить: имеет ли Вселенная в целом волновую функцию? Можем ли мы описать эту волновую функцию? Можно дать несколько ответов на этот вопрос:

\begin{enumerate}
\item Я не знаю, и мне все равно. Квантовая механика - это теория о наблюдениях и измерениях. Вы не можете измерить всю вселенную.
\item Мне не все равно, но я не знаю. Такая волновая функция может быть настолько запутана, что ее невозможно описать. Может быть, у вселенной есть матрица плотности, а не волновая функция.
\item Вселенная не имеет фиксированной волновой функции. Каждый раз, когда проводится наблюдение или измерение, волновая функция коллапсирует, и этот феномен коллапса не следует никакому уравнению Шредингера.
\item Да, вселенная имеет волновую функцию. Она подчиняется уравнению Шредингера, и вероятность того, что какое-либо состояние $|\psi\rangle$ всегда реализуется, определяется нормой квадрата скалярного произведения. Всякий раз, когда происходит какой-либо  <<коллапс>>, он всегда подчиняется уравнению Шредингера.
\end{enumerate}
              
Агностические ответы (1) и (2), конечно, научно оправданы. Мы должны ограничиться наблюдениями и не задавать глупых вопросов. Тем не менее, они, кажется, признают, что квантовая механика может не иметь универсального действия. Это не относится к вселенной в целом. Почему нет? Где именно находится предел достоверности квантовой механики? Идеи, выраженные в этой работе, подверглись нападкам, потому что они якобы не согласуются с наблюдениями, но все наблюдения, когда-либо сделанные в атомной и субатомной науке, по-видимому, подтверждают правильность квантовой механики, тогда как ответы (1) и (2) предполагают, что квантовая механика должна сломаться в какой-то момент. В CAI мы предполагаем, что математические правила для применения квантовой механики имеют абсолютную ценность. Мы считаем, что это не безумное предположение.

В том же духе мы также исключаем ответ (3). Коллапс не должен рассматриваться как отдельная аксиома квантовой механики, которая аннулирует уравнение Шредингера всякий раз, когда имеет место наблюдение, измерение и, следовательно, коллапс. Следовательно, согласно нашей теории, единственным правильным ответом является ответ (4). Очевидная проблема с ним состояла бы в том, что коллапс потребовал бы <<заговора>>, совершенно особого выбора начального состояния, потому что в противном случае мы могли бы случайно прийти к волновым функциям, которые являются квантовыми суперпозициями разных коллапсирующих состояний. Здесь на помощь приходят онтологические базисы. Если вселенная находится в онтологическом состоянии, ее волновая функция будет разрушаться автоматически при необходимости. В результате классические конфигурации, такие как положения и скорости планет, всегда описываются волновыми функциями, которые дельтаподобны при этих значениях данных, тогда как волновые функции, которые являются суперпозициями планет в разных местах, никогда не будут онтологическими.

Вывод этого подраздела состоит в том, что, пока мы работаем с шаблонами, наши амплитуды являются псиэпистимическими, как они были в представлении Копенгагена, но существует пси-онтическая волновая функция: волновая функция самой вселенной. Она эпистемологична и онтологична одновременно.

Теперь давайте вернемся в Копенгаген и сформулируем правила. Как видит читатель, в некоторых отношениях мы даже более консервативны, чем самые неприятные квантовые догматики.

\subsection{Правила}\label{ch5.4}

Что касается копенгагенских правил, которые мы соблюдаем, мы выделяем те, которые наиболее важны для нас:

\textbf{(i)} {\it Для описания физического явления использование любого базиса столь же законно, как и любого другого. Мы можем выполнять любое преобразование, которое нам нравится, и перефразировать уравнение Шредингера, или, скорее, используемый в нем гамильтониан, соответственно. В каждом базисе мы можем найти полезное описание переменных, таких как положения частиц или значения их импульсов, или энергетических состояний, в которых они находятся, или полей, в которых эти частицы являются квантами энергии. Все эти описания одинаково <<реальны>>.}

Но ни одно из обычно используемых описаний не является полностью реалестичным. Мы часто видим, что возникают суперпозиции, и иногда можно измерять фазовые углы этих суперпозиций. В таких случаях используемый базис не является онтологическим. На практике мы узнали, что это просто прекрасно; мы используем все эти различные базовые варианты, чтобы иметь шаблоны. Мы не будем налагать никаких ограничений на то, какой шаблон <<разрешен>> или какой из них может представлять истину <<лучше>>, чем другие. Поскольку это всего лишь шаблоны, реальность может оказаться суперпозицией различных шаблонов.

Любопытно, что даже среди приверженцев квантовой механики это часто считалось неочевидным. ,,Фотоны - это не частицы, а протоны и электроны - да'', - утверждали некоторые исследователи. Фотоны должны рассматриваться как кванты энергии. Они, конечно, думали, что глупо рассматривать \textit{фонон} как частицу; это просто квант звука. Иногда утверждают, что электрические и магнитные поля являются <<истинными>> переменными, а фотоны - просто абстрактными понятиями. Были споры о том, является ли частица истинной частицей в координатном представлении или в представлении импульса и так далее. Мы покончим со всем этим. Все варианты базиса эквивалентны. Они - не что иное как система координат в гильбертовом пространстве. Пока используемый гамильтониан, по-видимому, таков, что эволюционные операторы конечного времени превращают диагональные операторы (beables) в недиагональные (superimposables), эти варианты базиса явно неонтологичны; Ни один из них не описывает, что на самом деле происходит. Что касается энергетической основы, см. Раздел \ref{ch5.6.3}.

Обратите внимание, что это означает, что нас будет интересовать не гамильтониан, а его класс сопряженности. Если гамильтониан $H$ преобразован в новый гамильтониан преобразованием

\begin{equation}\label{5.7}
	\tilde H = GHG^{-1}
\end{equation}

           
где $G$ унитарный оператор, то новый гамильтониан $\tilde H$ в новом базисе так же действителен, как и предыдущий. На практике мы будем искать базис или его оператор $G$, который дает максимально краткое выражение для $\tilde H$.

\textbf{(ii)} {\it Учитывая кет-состояние $|\psi\rangle$, вероятность того, что результат измерения будет описан заданным состоянием $|a\rangle$ в гильбертовом пространстве, точно определяется абсолютным квадратом скалярного произведения $\langle a\mid \psi\rangle$.}

Это известное правило Борна. Мы никогда не изменим его математическую форму; можно использовать только квадраты скалярных произведений. Однако есть принципиальное ограничение: состояние бра $\langle a\mid$ должно быть онтологическим состоянием. На практике это всегда так: брашки $\langle a\mid$ обычно представлены классическими наблюдениями. Правило Борна часто выделяется как Отдельная аксиома Копенгагенской интерпретации. На наш взгляд, это неизбежное следствие математической природы квантовой механики как инструмента для выполнения расчетов, см. раздел \ref{ch4.3}.

Самым важным моментом при попытке избавиться от ограничений копенгагенской интерпретации, является то, что мы делаем некоторые фундаментальные предположения:

\textbf{(a)} {\it Мы постулируем существование онтологического базиса. Это ортонормированный базис гильбертова пространства, который действительно превосходит выборы базисов, с которыми мы знакомы. С точки зрения онтологического базиса, оператор эволюции для достаточно тонкой сетки временных переменных не более чем перестановка состояний.}

Как именно определить сетку переменных времени, мы не знаем в настоящее время, и это вполне может стать предметом дискуссий, особенно с учетом известного факта, что в пространстве-времени встроена общая координатная инвариантность. Мы не претендуем на то, чтобы знать, как прийти к этой конструкции - это слишком сложно. В любом случае ожидается, что система будет вести себя как классическая конструкция. Наше основное предположение состоит в том, что существует классический закон эволюции, точно определяющий, как эволюционирует пространство-время и все его содержимое. Эволюция детерминирована в том смысле, что ничто не оставлено на волю случая. Вероятности появляются только в том случае, если из-за нашего невежества мы ищем избавления в неонтологических базисах.

\textbf{(b)} {\it Когда мы выполняем традиционный квантово-механический расчет, мы используем набор шаблонов того, на что, как мы думали, похожа волновая функция. Эти шаблоны, такие как ортонормированный набор собственных энергетических состояний атома водорода, просто являются состояниями, для которых мы знаем, как они развиваются. Однако они лежат в основе довольно сложной унитарной трансформации онтологического базиса.}

Человечество обнаружило, что эти шаблоны подчиняются уравнениям Шредингера, и мы используем их для вычисления вероятностей для результатов экспериментов. Эти уравнения верны c очень высокой точностью, но они ложно предполагают, что существует <<мультивселенная>> из множества разных миров, которые мешают друг другу. Сегодня эти шаблоны являются лучшими у нас.

\textbf{(c)} {\it Весьма вероятно, что существует несколько разных вариантов выбора онтологического базиса, связанных друг с другом непрерывными преобразованиями симметрии Природы, такими как элементы группы Пуанкаре, но, возможно, также с группой локального диффеоморфизма, используемой в Общей теории относительности. Только один из этих онтологических базисов будет <<действительно>> онтологическим.}

Какой из них будет действительно онтологическим, будет трудно или невозможно определить. Тот факт, что мы не сможем разграничить различные возможные онтологические базисы, исключит возможность использования этих знаний для выполнения предсказаний, выходящих за рамки обычных квантово-механических. В любом случае это не было нашим намерением. Мотивация для этого исследования всегда заключалась в том, что мы ищем новые ключи для построения моделей, более совершенных, чем стандартная модель.

Преобразования симметрии, которые связывают различные (но часто эквивалентные) варианты выбора онтологий, вероятно, будут действительно квантово-механическими: операторы, диагональные в одном из этих онтологических базисов, могут быть недиагональными в другом. Однако в самом конце мы будем использовать только <<реальный>> онтологический базис. Это будет очевидно в аксиоме (е).

\textbf{(d)} {\it Классические состояния являются онтологическими, что означает, что классические наблюдаемые всегда диагональны в <<истинно>> онтологическом базисе. }
          
Это было бы труднее <<доказать>> из первых принципов, поэтому мы действительно представим это как аксиому. Тем не менее, кажется, что этого очень трудно избежать: трудно представить, что два разных классических состояния, будущее развитие которых в конечном итоге будет совершенно другим, могут иметь неисчезающие скалярные произведения с одинаковым онтологическим состоянием.

\textbf{(e)} {\it Вселенная с самого начала была и всегда будет развиваться в едином онтологическом состоянии. Это означает, что не только наблюдаемые являются диагональными в онтологическом базисе, но и волновая функция всегда принимает простейший возможный вид: она является одним из элементов самого базиса, поэтому эта волновая функция содержит только одну единицу, а все остальные нули (\textit{One-hot вектора})}.

Обратите внимание, что это выделяет <<истинный>> онтологический базис из других вариантов, где физические степени свободы также могут быть представлены <<beables>>, то есть операторами, которые всегда коммутируют. Таким образом, впредь мы будем называть этот <<истинный>> онтологический базис просто <<Онтологическим>>.\footnote{,,the'' ontological basis}

Что наиболее важно, последние две аксиомы полностью решают проблему измерения [93], вопрос коллапса и парадокс Шредингера. Теперь аргумент заключается в том, что Природа всегда находится в одном онтологическом состоянии, и поэтому она должна эволюционировать в единое классическое состояние.

\subsection{Особенности интерпретации клеточного автомата (CAI)}\label{ch5.5}

Особенностью CAI является то, что онтологические состояния никогда не образуют суперпозиций. Начиная с нулевого момента времени, вселенная должна быть в единственном эволюционирующем онтологическом состоянии. Это также причина, почему она никогда не превращается в суперпозицию классических состояний. Теперь запомните, что уравнение Шредингера подчиняется онтологическим состояниям, в которых может находиться Вселенная. Именно поэтому эта теория, не отходя от уравнения Шредингера, автоматически генерирует <<сколлапсированные волновые функции>>, которые описывают результаты измерения. По той же причине онтологические состояния никогда не могут развиться в суперпозицию мертвого кота и живого кота. С этой точки зрения, на самом деле трудно понять, каким образом любая другая интерпретация квантовой механики могла бы выжить в литературе: сама по себе квантовая механика предсказала бы, что если состояния $|\psi\rangle$ и $|\chi\rangle$ могут быть использованы в качестве начальных состояний, то можно использовать и $\alpha|\psi\rangle + \beta|\chi\rangle$. Однако суперпозиция мертвого кота и живого кота не может служить для описания конечного состояния. Если $|\psi\rangle$ превращается в живого кота, а $|\chi\rangle$ в мертвого, то во что превращается состояние $\alpha|\psi\rangle + \beta|\chi\rangle$? Стандартные ответы на такие вопросы не могут быть правильными.\footnote{Я здесь ссылаюсь на аргумент, что декогеренция, так или иначе, делает свою работу. См. \ref{ch3.5}}

Интерпретация клеточных автоматов добавляет некоторые понятия в квантовую механику, которые не имеют какого-либо особого значения в обычном Копенгагенском представлении. Мы представили Онтологический базис как в некотором смысле приоритетный перед любым другим базисом. Естественно можно утверждать, что это будет шагом назад в физике. Разве Копенгаген не подчеркивал, что все варианты базиса эквивалентны?
Почему один из них должен выделяться?

Действительно, в правиле (i) в разд. \ref{ch5.4} было сказано, что все варианты базиса эквивалентны, но на самом деле мы имели в виду, что все \textit{обычно используемые}  наборы базисов эквивалентны. Как только мы примем копенгагенскую доктрину, уже не имеет значения, какой базис выбирать. И все же в копенгагенском формализме есть одна проблема, которая широко обсуждается в литературе и в настоящее время действительно признана слабостью: коллапс волновой функции и обработка измерений. В этих точках аксиома суперпозиции терпит неудачу. Как только мы признаем, что существует один предпочтительный базис, эта слабость исчезает. Все волновые функции могут использоваться для описания физического процесса, но тогда мы должны терпеть аксиому коллапса, если мы не работаем в онтологическом базисе.

CAI позволяет нам использовать этот особый базис. Он выделяется тем, что в нем мы распознаем особые волновые функции: онтологические волновые функции. В онтологическом базисе онтологические волновые функции - это волновые функции, соответствующие элементам базиса; каждая из этих волновых функций предсталена одной единицей и остальными нулями. Онтологические волновые функции снова эволюционируют исключительно в онтологические волновые функции. Исходя из этого, больше нет места для случайности, и суперпозиций можно полностью избежать.

Тот факт, что и в обычном формализме квантовой механики состояния, которые начинаются с классического описания, такого как направленные друг на друга пучки частиц, в конечном итоге оказываются классическими вероятностными распределениями частиц, выходящих из области взаимодействия, на наш взгляд может рассматриваться как свидетельство <<закона сохранения онтологий>>, закона, который говорит, что существует онтологический базис, такой, что истинные онтологические состояния в один момент времени всегда превращаются в истинные онтологические состояния в более поздние времена. Это новый закон сохранения. Заманчиво сделать вывод, что CAI неизбежна.

В онтологическом базисе эволюция детерминирована. Однако этот термин следует использовать с осторожностью. <<Детерминизм>> не может означать, что результат процесса эволюции можно предвидеть. Ни один человек, или даже любое другое мыслимое разумное существо, не сможет вычислить быстрее, чем сама Природа. Причина этого очевидна: наше разумное существо также должно было бы использовать законы Природы, и у нас нет оснований ожидать, что Природа сможет дублировать свои собственные действия более эффективно, чем вначале. Вот как можно восстановить концепцию <<свободной воли>>: все, что происходит в нашем мозгу, уникально и непредсказуемо кем-либо или чем-либо.

\subsubsection{Beables, Changeables и Superimposables}\label{ch5.5.1}

Имея специальный базис и специальные волновые функции, мы можем также различать специальные наблюдаемые или операторы. В стандартной квантовой механике мы узнали, что операторы и наблюдаемые неразличимы, поэтому мы используем эти понятия взаимозаменяемо. Теперь нам нужно научиться восстанавливать различия. Мы повторяем то, что было сказано в разд. 2.1.1 операторы могут быть трех разных форм:

(I) beables $\hat{\mathcal{B}}$: они обозначают свойство онтологических состояний, так что beables диагонализованы в онтологическом базисе $\{\mid A\rangle, \mid B\rangle,...\}$ гильбертова пространства:             

\begin{equation}\label{5.8}
	\hat{\mathcal{B}}^{a}|A\rangle=\mathcal{B}^{a}(A)|A\rangle, \quad \text { (beable) }
\end{equation}
                          
Beables будет всегда коммутировать друг с другом, в любые моменты времени:

\begin{equation}\label{5.9}
	\left[\hat{\mathcal{B}}^{a}\left(\vec{x}_{1}, t_{1}\right), \hat{\mathcal{B}}^{b}\left(\vec{x}_{2}, t_{2}\right)\right]=0 \quad \forall \vec{x}_{1}, \vec{x}_{2}, t_{1}, t_{2}
\end{equation}

Квантованные поля, обильно присутствующие во всех теориях элементарных частиц, подчиняются уравнению (\ref{5.9}), но только вне светового конуса (где $| t_1 - t_2 | <| \vec x_1 - \vec x_2 |$), но не внутри этого конуса, где уравнения (20.29), не выполняются, что легко может быть получено из явных вычислений. Таким образом, квантованные поля в целом отличаются от beables.

(II) changeables $\hat{\mathcal{C}}$: операторы, которые заменяют онтологическое состояние $\mid A\rangle$ другим онтологическим состоянием $\mid B\rangle$, как, например, оператор перестановки:  

\begin{equation}\label{5.10}
	\hat{\mathcal{C}}|A\rangle=|B\rangle, \quad \text { (changeable) }
\end{equation}         

Changeables не коммутируют, но они имеют особые отношения с beables; они обмениваются ими:

\begin{equation}\label{5.11}
	\hat{\mathcal{B}}^{(1)} \hat{\mathcal{C}} = \hat{\mathcal{C}}  \hat{\mathcal{B}}^{(2)} 
\end{equation}
       
Мы можем захотеть сделать исключение для бесконечно малых changeables, таких как гамильтониан:

\begin{equation}\label{5.12}
	[\hat{\mathcal{B}}, \hat H] = \mathrm i \frac \partial {\partial t} \hat{\mathcal{B}} 
\end{equation}
            
(III) superimposables $\hat{\mathcal{S}}$: они отображают онтологическое состояние на любую другую, более общую суперпозицию онтологических состояний:  

\begin{equation}\label{5.13}
	\hat{\mathcal{S}}|A\rangle = \lambda_{1}|A\rangle+\lambda_{2}|B\rangle+\cdots, \quad \text { (superimposable) }
\end{equation}
            
Все обычно используемые операторы являются superimposables, даже самые простые операторы координаты или импульса в квантовой механике базового университетского курса. В этом легко удостовериться, проверив зависящие от времени правила коммутации (в представлении Гейзенберга). В общем:\footnote{Редким исключением, например, является гармонический осциллятор, когда временной интервал является целым кратным периоду T}

\begin{equation}\label{5.14}
	\left[\vec{x}\left(t_{1}\right), \vec{x}\left(t_{2}\right)\right] \neq 0, \quad \text { if } \quad t_{1} \neq t_{2}
\end{equation}


\subsubsection{Наблюдатели и Наблюдаемые}\label{ch5.5.2}

Стандартная квантовая механика преподнесла нам ряд важных уроков. Один из них заключается в том, что нельзя произвести наблюдение без нарушения наблюдаемого объекта. Этот момент также актуален и в CAI. Если измерение положения частицы означает проверку волновой функции $\vec{x} \stackrel{?}{=} \vec{x}^{(1)}$, это можно интерпретировать как действие оператора $\hat P(\vec{x}^{(1)})$ на состояние:

\begin{equation}\label{5.15}
	|\psi\rangle \rightarrow \hat P\left(\vec{x}^{(1)}\right)|\psi\rangle, \quad \hat P\left(\vec{x}^{(1)}\right)=\delta\left(\hat{\vec{x}}-\vec{x}^{(1)}\right)
\end{equation}
          
Это изменяет состояние, и, следовательно, все операторы, действующие на него после этого, могут давать результаты, отличные от того, что они делали до «измерения».

Однако, когда подлинный beable действует в онтологическом состоянии, состояние просто умножается на найденное значение, но будет развиваться так же, как и раньше (при условии, что мы выбрали «истинный» онтологический базис, см. Аксиому (с) в разделе \ref{ch5.4}.). Таким образом, измерения beables являются, в некотором смысле, классическими измерениями. Они будут единственными измерениями, которые не нарушают волновую функцию, но, конечно, такие измерения не могут быть выполнены на практике.

Другие измерения, которые кажутся полностью законными в соответствии с традиционной квантовой механикой, не будут возможны в CAI. В CAI, как и в обычной квантовой механике, мы можем рассмотреть любой оператор и изучить его ожидаемое значение. Но поскольку класс физически реализуемых волновых функций теперь меньше, чем в стандартной КМ, некоторые состояния больше не могут быть реализованы, и мы не можем сказать, каким может быть результат такого измерения. Подумайте о любом (не бесконечно малом) changeable $\hat{\mathcal{C}}$. Все онтологические состояния будут давать «ожидаемое значение» ноль , но мы можем рассмотреть его собственные значения, которые в общем случае не приведут к нулевому значению. Соответствующие собственные состояния определенно не являются онтологическими (см. Раздел \ref{ch5.7.1}).

Значит ли это, что стандартная квантовая механика находится в конфликте с CAI? Подчеркнем, что это не так. Следует понимать, что и в традиционной квантовой механике можно считать приемлемым сказать, что Вселенная была и всегда будет находиться в одном и том же квантовом состоянии, эволюционируя во времени в соответствии с уравнением Шредингера (в представлении Шредингера) или остается прежней (в представлении Гейзенберга). Если это состояние является одним из наших онтологических состояний, то оно ведет себя именно так, как и должно. Обычная квантовая механика использует шаблонные состояния, большинство из которых не являются онтологическими, но в конечном итоге предполагается, что реальный мир находится в суперпозиции этих шаблонных состояний, так что онтологическое состояние все равно всплывает, и наши очевидные разногласия исчезают.

\subsubsection{Скалярные произведения шаблонных состояний}\label{ch5.5.3}

При выполнении технических вычислений мы выполняем преобразования и суперпозиции, которые приводят к большим наборам квантовых состояний, которые мы сейчас рассматриваем как шаблоны или модели-кандидаты для (суб) атомных процессов, которые всегда могут быть наложены на более поздней стадии для описания наблюдаемых нами явлений. Внутренние (скалярные) произведения шаблонов можно изучить обычным способом.

Состояние шаблона $\mid \psi \rangle$ может использоваться в качестве модели некоторого реально наблюдаемого явления. Это может быть любая квантовая суперпозиция онтологических состояний $\mid A \rangle$. Скалярное произведение $|\langle A \mid \psi \rangle|^2$ представляет вероятность того, что онтологическое состояние $\mid A \rangle$ действительно реализуется.

Согласно копенгагенскому правилу III, разд. \ref{ch5.4} вероятность того, что состояние шаблона $\mid \psi_1 \rangle$ окажется равным состоянию $\mid \psi_2 \rangle$, определяется как $|\langle \psi_1 \mid \psi_2 \rangle|^2$. Тем не менее, уже в начале, раздел \ref{ch2.1} мы заявили, что внутреннее произведение $\langle \psi_1 \mid \psi_2 \rangle$ не должно интерпретироваться таким образом. Даже если их скалярное произведение исчезает, шаблонные состояния $\mid \psi_1 \rangle$ и $\mid \psi_2 \rangle$ могут иметь неисчезающий коэффициент с идентичным онтологическим состоянием $\mid A \rangle$. Это не означает, что мы отступаем от копенгагенского правила (ii), но истинная волновая функция не может быть просто общим шаблоном; это всегда онтологическое состояние $\mid A \rangle$. Это означает, что правило скалярного произведения верно только в том случае, если либо $\mid \psi_1 \rangle$, либо $\mid \psi_2 \rangle$ является онтологическим состоянием, а другое может быть шаблоном. Затем мы рассматриваем вероятность того, что состояние шаблона точно совпадает с одним онтологическим состоянием $\mid A \rangle$.

Таким образом, мы используем борновскую интерпретацию скалярного произведения $|\langle \psi_1 \mid \psi_2 \rangle|^2$, если один из двух шаблонов рассматривается как кандидат в онтологическое состояние. Это законно, так как мы знаем, что онтологические состояния - это сложные суперпозиции наших шаблонов. Существуют ненаблюдаемые степени свободы, и то, как они связаны с этим состоянием, становится несущественным. Таким образом, можно предположить, что одно из наших состояний шаблона представляет распределение вероятностного состояния во вселенной, а другое представляет собой модель онтологического состояния.

Мы видим, что правило внутреннего произведения может использоваться двумя способами; один - описать распределение вероятностей начальных состояний рассматриваемой системы, а другой - описать вероятность достижения заданного классического состояния в конце квантового процесса. Если для описания начальных вероятностей используется правило Борна, то же правило можно использовать для расчета вероятностей для конечных состояний.

\subsubsection{Матрицы плотности}\label{ch5.5.4}

Матрицы плотности используются, когда мы не знаем ни онтологических состояний, ни шаблонов. Берется набор шаблонов $\mid \psi_i \rangle$ и им приписываются  вероятности $W_i \le 0$, такие, что $\sum_i W_i = 1$. Это называется смешанным состоянием. В стандартной квантовой механике можно найти ожидаемые значения оператора c помощью:

\begin{equation}\label{5.16}
	\langle\mathcal{O}\rangle=\sum_{i} W_{i}\left\langle\psi_{i}|\mathcal{O}| \psi_{i}\right\rangle=\operatorname{Tr}(\varrho \mathcal{O}) ; \quad \varrho=\sum_{i} W_{i}\left|\psi_{i}\right\rangle\left\langle\psi_{i}\right|
\end{equation}

Оператор $\varrho$ называется матрицей плотности.

Обратите внимание, что если онтологический базис известен и оператор $\varrho$ является beable, то вероятности неотличимы от вероятностей, сгенерированных шаблоном,


\begin{equation}\label{5.17}
	|\psi\rangle=\sum_{i} \lambda_{i}\left|\psi_{i}^{\mathrm{ont}}\right\rangle, \quad\left|\lambda_{i}\right|^{2}=W_{i}
\end{equation}

поскольку в обоих случаях матрица плотности в этом базисе диагональна:

\begin{equation}\label{5.18}
	\varrho=\sum_{i} W_{i}\left|\psi_{i}^{\mathrm{ont}}\right\rangle\left\langle\psi_{i}^{\mathrm{ont}}\right|
\end{equation}

Если $\varrho$ не является beable, недиагональные элементы матрицы плотности могут показаться существенными, но мы должны помнить, что не-beable операторы в принципе не могут быть измерены, и это означает, что формальное различие между матрицами плотности и шаблонами исчезает.

Тем не менее использование матриц плотности важно в физике. На практике матрица плотности используется для описания ситуаций, в которых можно предсказать меньше, чем то, что можно получить максимально в идеальных наблюдениях. Взять хотя бы один кубит. Если мы рассмотрим ожидаемые значения матриц Паули $\sigma_x$, $\sigma_y$ и $\sigma_z$, вычисление с использованием кубита даст


\begin{equation}\label{5.19}
	\left|\left\langle\sigma_{x}\right\rangle\right|^{2}+\left|\left\langle\sigma_{y}\right\rangle\right|^{2}+\left|\left\langle\sigma_{z}\right\rangle\right|^{2}=1
\end{equation}
            
тогда как смешанное состояние даст

\begin{equation}\label{5.20}
	\left|\left\langle\sigma_{x}\right\rangle\right|^{2}+\left|\left\langle\sigma_{y}\right\rangle\right|^{2}+\left|\left\langle\sigma_{z}\right\rangle\right|^{2} < 1
\end{equation}

Это равносильно потере информации в дополнение к обычной квантовой неопределенности.

\subsection{Гамильтониан}\label{ch5.6}
            
Как было объяснено во введении к этой главе, разд. \ref{ch5.1}, существует много способов выбора гамильтонова оператора, который правильно выдает уравнение Шредингера для временной зависимости (клеточного) автомата. И все же гамильтониан для квантового мира, каким мы его знаем, и в частности для Стандартной модели, является совершенно уникальным. Как получить «правильный» гамильтониан?

Конечно, в Стандартной модели есть консервативные величины, такие как химические потенциалы, глобальные и локальные заряды, и кинематические величины, такие как угловой момент. Они могут быть добавлены к гамильтониану с произвольными коэффициентами, но они, как правило, весьма отличаются от того, что мы склонны называть «энергией», поэтому их можно будет ликвидировать. Тогда существует много нелокальных консервативных величин, что объясняет большое количество возможных сдвигов $\delta E_i$ на рис. \ref{i2.3}, разд. \ref{ch2.2.2}. Большинство таких двусмысленностей будут устранены, если потребовать, чтобы гамильтониан был локальным.

\subsubsection{Локальность}\label{ch5.6.1}

Наши начальные выражения для гамильтониана детерминированной системы - это уравнения (\ref{2.8}) и (\ref{2.26}). Они, однако, сходятся очень медленно при больших значениях $n$. Если мы применим такие разложения к клеточному автомату (уравнения (\ref{5.2}) и (\ref{5.3})), то видим, что $n$-й член будет включать взаимодействия над соседями, которые разделены на $n$ шагов. Если мы напишем полный гамильтониан $H$ как

\begin{equation}\label{5.21}
	H=\sum_{\vec{x}} \mathcal{H}(\vec{x}), \quad \mathcal{H}(\vec{x})=\sum_{n=1}^{\infty} \mathcal{H}_{n}(\vec{x})
\end{equation}
            
то увидим вклады $\mathcal{H}_n(\vec x)$, которые включают взаимодействия по $n$ соседям, коэффициенты которых обычно падают как $1 / n$. Как правило,

\begin{equation}\label{5.22}
	\left[\mathcal{H}_{n}(\vec{x}), \mathcal{H}_{m}\left(\vec{x}^{\prime}\right)\right]=0 \quad \text { only if } \quad\left|\vec{x}-\vec{x}^{\prime}\right|>n+m
\end{equation}

в то время как в релятивистских квантовых теориях поля мы имеем $\left[\mathcal{H}(\vec{x}), \mathcal{H}\left(\vec{x}^{\prime}\right)\right]=0$, как только $\vec{x} \neq \vec{x}^{\prime}$. Учитывая, что число взаимодействующих соседних ячеек, которые вписываются в размерную сферу с радиусом $n$, может быстро расти с ростом $n$, в то время как главные члены начинают с порядка энергии Планка, мы видим, что эта сходимость слишком медленная: большие вклады распространятся на большие расстояния. Это не гамильтониан, имеющий локальную структуру, типичную для стандартной модели.

Теперь это не должно удивлять. Собственные значения гамильтонианов (\ref{2.8}) и (\ref{2.26}) ограничены областью $(0,2\pi / \delta t)$, в то время как любой гамильтониан, описываемый уравнениями, такими как (\ref{5.21}), должен быть экстенсивным: его собственные значения могут расти пропорционально объему пространства.

Более совершенная конструкция для клеточного автомата рассматривается в части II, гл. \ref{ch21}. Там мы впервые представляем расширение Baker Campbell Hausdorff. При этом также самые низкие члены соответствуют полностью локальной гамильтоновой плотности, в то время как все члены являются экстенсивными. К сожалению, это также связано с ценой, которая слишком высока для нас: серия Baker Campbell Hausdorff, похоже, не сходится вообще в этом случае. Можно утверждать, что, если использовать это дело только для состояний, где полные энергии намного меньше, чем энергия Планка, ряды должны сходиться, но у нас нет никаких доказательств этого. Наша проблема в том, что в используемых выражениях промежуточные состояния могут легко представлять более высокие энергии.

Было предпринято несколько попыток получить положительный гамильтониан, который также является пространственным интегралом по локальным гамильтоновым плотностям. Проблема в том, что клеточный автомат определяет только локальный оператор эволюции за конечные временные шаги, а не локальный гамильтониан, который задает бесконечно малую временную эволюцию. Кажется, что общепринятые формальные процедуры терпят неудачу, но если мы будем придерживаться процедур, похожих на теории возмущенных квантовых полей, мы подойдем к интересным описаниям, которые почти решают проблему. В гл. \ref{ch22} части II мы используем вторичное квантование. Эту процедуру можно суммировать следующим образом: сначала рассмотрим клеточный автомат, который описывает различные типы частиц, все они не взаимодействуют. Этот автомат будет интегрируемым, а его гамильтониан $H_0$ непременно будет подчиняться локальным свойствам и будет иметь нижнюю оценку. Далее нужно вводить взаимодействия как крошечные возмущения. Это не должно быть сложно в клеточных автоматах; просто введите небольшие отклонения от закона эволюции свободных частиц. Эти небольшие возмущения, даже если они дискретные и детерминированные, могут обрабатываться пертурбативно, предполагая, что возмущение возникает нечасто в достаточно разнесенных точках в пространстве и времени. Это должно привести к чему-то, что может воспроизвести теории возмущенного квантового поля, такие как стандартная модель.

Обратите внимание, что в большинстве квантовых теорий поля разложения возмущений использовались с большим успехом (например, расчет аномального магнитного момента электрона $g - 2$, который можно рассчитать и сравнить с экспериментом с превосходной точностью), в то время как до сих пор подозревается, что разложения в конечном итоге не сходятся. Однако неконвергенция проявляется в очень высоких порядках, выходящих за пределы сегодняшних практических пределов точности экспериментов.

Теперь мы считаем, что это будет наилучшим способом построения гамильтониана со свойствами, которые можно сравнить с экспериментально установленными описаниями частиц. Но это всего лишь стратегия; было невозможно выработать детали, потому что необходимые детерминированные теории свободных частиц еще недостаточно понятны.

Таким образом, в этой области еще предстоит проделать большую работу. Вопросы технически довольно сложны, и поэтому мы отложим детали до части II этой книги.

\subsubsection{Двойная роль гамильтониана}\label{ch5.6.2}

Без гамильтониана теоретическая физика выглядела бы совершенно иначе. В классической механике у нас есть центральная проблема, что механическая система подчиняется закону сохранения энергии. Энергия является неотрицательной, аддитивной величиной, которая локально хорошо определена. Именно эти свойства гарантируют устойчивость механических систем к полному разрушению или взрывоопасным решениям.

Классический гамильтонов принцип основан на превосходном способе реализации этого механизма. Все, что нужно, - это постулировать выражение для неотрицательной, сохраняющейся величины, называемой энергией, которая превращается в гамильтониан $H (\vec x,\vec p)$, если мы тщательно определим динамические величины, от которых она зависит, будучи каноническими парами положений $x_i$ и импульсов $p_i$. Гениальная идея состояла в том, чтобы взять в качестве уравнения движения уравнения Гамильтона-Якоби

\begin{equation}\label{5.23}
	\frac{\mathrm{d}}{\mathrm{d} t} x_{i}(t)=\frac{\partial}{\partial p_{i}} H(\vec{x}, \vec{p}), \quad \frac{\mathrm{d}}{\mathrm{d} t} p_{i}(t)=-\frac{\partial}{\partial x_{i}} H(\vec{x}, \vec{p})
\end{equation}

Это гарантирует, что $\frac d {dt} H (\vec x,\vec p) = 0$. Тот факт, что уравнения (\ref{5.23}) допускают большой набор математических преобразований, делает принцип еще более мощным.

В квантовой механике, как должен знать читатель, можно использовать одну и ту же гамильтонову функцию $H$ для определения уравнения Шредингера с тем же свойством: оператор $H$ сохраняется во времени. Если $H$ ограничен снизу, это гарантирует существование основного состояния, обычно вакуума.

Таким образом, как квантовая, так и классическая механика основаны на этом мощном принципе, когда одна физическая переменная, гамильтониан, делает две вещи: она генерирует уравнения движения и дает локально сохраняющуюся энергетическую функцию, которая стабилизирует решения уравнений движения. Вот как принцип Гамильтона описывает уравнения движения или эволюционные уравнения, решения которых гарантированно устойчивы.

Теперь, как это работает в дискретных, детерминированных системах того типа, который мы здесь изучали? Наша проблема в том, что в дискретной классической системе энергия также должна быть дискретной, но генератор эволюции должен быть оператором с непрерывными собственными значениями. Непрерывные дифференциальные уравнения (\ref{5.23}) должны быть заменены чем-то другим. В принципе, мы могли бы попытаться восстановить непрерывную временную переменную и определить, как наша система развивается в терминах этой временной переменной. Что нам действительно нужно, так это оператор $H$, который частично представляет положительную, сохраненную энергию, а частично - генератор бесконечно малых изменений времени. Мы подробно рассмотрим этот вопрос в части II, гл. \ref{ch19}, где, среди прочего, мы строим классический дискретизированный гамильтониан, чтобы применить версию принципа Гамильтона для клеточных автоматов.

\subsubsection{Энергетический базис}\label{ch5.6.3}

В разделе \ref{ch5.5.1} было объяснено, что детерминированная модель квантово-механической системы получается, если мы можем найти набор коммутирующих beable операторов (\ref{5.8}) см. уравнение (\ref{5.9}). Тогда онтологические состояния являются собственными состояниями этих beables. Существует тривиальный пример таких операторов и таких состояний в реальном мире: гамильтониан и его собственные состояния. Согласно нашему определению, они образуют набор beables, но, к сожалению, они тривиальны: есть только один гамильтониан, а собственные состояния энергии не меняются во времени вообще. Это описывает статический классический мир. Что с этим не так?

Поскольку мы объявили, что суперпозиции онтологических состояний не являются онтологическими, это решение также говорит нам о том, что если собственные энергетические состояния будут рассматриваться как онтологические, их суперпозиции не будут допущены, в то время как вся обычная физика возникает только тогда, когда мы рассматриваем суперпозиции энергетических состояний. Только суперпозиции различных энергетических состояний могут зависеть от времени, так что да, это решение, но нет, это не то решение, которое нам нужно. Решение на основе энергии возникает, например, если мы возьмем модель разд. 2.2.2, рис. \ref{i2.2} \ref{i2.3}, где мы заменяем все петли на тривиальные петли, имеющие только одно состояние, в то время как вся физика помещается в произвольные числа $\delta E_i$. Это соответствует правилам, но бесполезно.

Таким образом, выбор энергетического базиса представляет собой крайний предел, который часто бесполезен на практике. Мы видим, что это также происходит, когда рассматривается очень важная процедура: кажется, что нам придется разделить энергию на две части: с одной стороны, существует большой классический компонент, в котором энергия, эквивалентная массе, действует как источник гравитационного поля и как таковой должен быть онтологическим, то есть классическим. Эта часть, вероятно, должна быть дискретизирована. С другой стороны, у нас есть меньшие компоненты энергии, которые действуют как собственные значения оператора эволюции, в течение достаточно больших временных шагов (намного больше, чем время Планка). Они должны формировать непрерывный спектр.

Если бы мы рассматривали энергетический базис как наш онтологический, мы бы рассматривали всю энергию как классическую, но тогда часть, описывающая эволюцию, исчезает; это не хорошая физика. См. Часть II, Рис. \ref{i19.1}, в разд. \ref{19.4.1}. Замкнутые контуры на этой фигуре должны быть нетривиальными.

\subsection{Разное}\label{ch5.7}

\subsubsection{Оператор обмена Земля-Марс}\label{ch5.7.1}

CAI предполагает, что существуют квантовые модели, которые можно рассматривать как скрытые классические системы. Если внимательно посмотреть на эти классические системы, то создается впечатление, что любую классическую систему можно перефразировать на квантовом языке: мы просто постулируем элемент базиса гильбертова пространства, чтобы он соответствовал каждому классическому физическому состоянию, которое допускается в системе. Оператор эволюции - это перестановка, которая заменяет состояние своим преемником во времени, и мы можем или не можем принять решение позже рассмотреть непрерывный предел времени.

Поэтому, естественно, мы задаем вопрос, можно ли обратить вспять CAI и построить квантовые теории для систем, которые обычно считаются классическими. Ответ - да. Чтобы проиллюстрировать это, давайте рассмотрим планетную систему. Это прототип классической теории. Мы рассматриваем большие планеты, вращающиеся вокруг Солнца, и мы игнорируем неньютоновские поправки, такие как специальная и общая теория относительности, или действительно квантовые эффекты, причем все они являются незначительными поправками. Мы рассматриваем планеты как точечные частицы, даже если они могут иметь сложные погодные условия или жизнь; мы просто смотрим на их классические, ньютоновские уравнения движения.

Онтологические состояния определяются путем перечисления положений $\vec x_i$ и скоростей $\vec v_i$ планет (которые коммутируют), и наблюдаемые, описывающие их, являются правдоподобными. Тем не менее, эта система также позволяет вводить changeables и superimposables. Квантовый гамильтониан здесь не классическая вещь, но

\begin{equation}\label{5.24}
	H^{\text {quant }}=\sum_{i}\left(\hat{\vec{p}}_{x, i} \cdot \vec{v}_{i}+\hat{\vec{p}}_{v, i} \cdot \vec{F}_{i}(\mathbf{x}) / m_{i}\right)
\end{equation}

где

\begin{equation}\label{5.25}
	\hat{\vec{p}}_{x, i}=-i \frac{\partial}{\partial \vec{x}_{i}}, \quad \hat{\vec{p}}_{v, i}=-i \frac{\partial}{\partial \vec{v}_{i}}, \quad \text { and } \quad \mathbf{x}=\left\{\vec{x}_{i}\right\}
\end{equation}

Здесь $\vec F_i(\mathbf{x})$ - классические силы на планетах, которые зависят от всех положений. Уравнение (\ref{5.24}) можно записать более элегантно как


\begin{equation}\label{5.26}
	H^{\mathrm{quant}}=\sum_{i}\left(\hat{\vec{p}}_{x, i} \cdot \frac{\partial H^{\mathrm{class}}}{\partial \vec{p}_{i}}-\hat{\vec{p}}_{p, i} \cdot \frac{\partial H^{\mathrm{class}}}{\partial \vec{x}_{i}}\right)
\end{equation}

где $\hat{\vec{p}}_{p,i} = m_i^{-1} \hat{\vec{p}}_{v,i}$. Ясно, что $\hat{\vec{p}}_{xi}$, $\hat{\vec{p}}_{vi}$ и $\hat{\vec{p}}_{pi}$ являются бесконечно малыми переменными, как и, конечно, гамильтониан $H^{\mathrm{quant}}$. Планеты теперь охватывают гильбертово пространство и ведут себя так, как будто они являются квантовыми объектами. Мы не модифицировали физику системы.

Мы можем продолжать определять еще больше changeables операторов, и спрашивать, как они развиваются во времени. Один из фаворитов автора - оператор обмена Earth-Mars. Он помещает Землю там, где сейчас находится Марс, и помещает Марс туда, где находится планета Земля. Скорости также взаимозаменяемы\footnote{ На какое-то мгновение мы перестали обращать внимание на луны; мы могли бы тащить их за собой или держать там, где они есть. Они не играют никакой роли в этом споре.}. Если бы Земля и Марс имели одинаковую массу, планеты продолжали бы развиваться, как будто ничего не произошло. Однако, поскольку массы различны, этот оператор будет иметь довольно сложные свойства, по ходу времени. Он не стационарен во времени.

Собственные значения оператора обмена EarthMars $X_{EM}$ легко вычислить:

\begin{equation}\label{5.27}
	X_{EM} = \pm1
\end{equation}

просто потому, что квадрат этого оператора равен единице. В стандартном языке КМ $X_{EM}$ является наблюдаемым. Он не коммутирует с гамильтонианом из-за различий в массе, но в определенный момент $t = t_1$ мы можем рассмотреть одно из состояний и спросить, как он развивается.

Почему все это звучит так странно? Как мы наблюдаем $X_{EM}$? Никто не может физически обменять планету Земля с Марсом. Но тогда никто не может обменять два электрона, и все же, в квантовой механике, это важный оператор. Ответ на эти вопросы заключается в том, что касается планетной системы, мы случайно знаем, каковы beables: они являются позициями и скоростями планет, и это превращает их в онтологические наблюдаемые. Базис, в котором эти наблюдаемые являются диагональными операторами, является нашим предпочтительным базисом. Элементами этого базиса являются онтологические состояния планет. Если в квантовом мире исследователи однажды обнаружат, что такое онтологические beables, все будет выражено через них, и любые другие операторы больше не будут актуальны.

Важно понимать, что, несмотря на тот факт, что в копенгагенском языке $X_{EM}$ является наблюдаемым (поскольку он эрмитов), мы не можем измерить его, чтобы увидеть, равен ли он +1 или -1. Это потому, что мы знаем волновую функцию $| ont\rangle$. Здесь 1 для фактических положений Земли и Марса, 0, когда мы меняем их местами. Это суперпозиция

\begin{equation}\label{5.28}
	| \text {ont}\rangle=\frac{1}{\sqrt{2}}\left(\left|X_{E M}=1\right\rangle+\left|X_{E M}=-1\right\rangle\right)
\end{equation}
            
которая является суперпозицией двух шаблонных состояний. Согласно Копенгагену, измерение даст $\pm 1$ с вероятностью $50\% / 50\%$.

\subsubsection{Отказ от локальной контрфактуальной определенности и свободы воли}\label{ch5.7.2}

Аргументы, обычно используемые для вывода о том, что локальные скрытые переменные не могут существовать, начинаются с предположения, что такие скрытые переменные должны подразумевать локальную контр-фактическую определенность. Можно представить себе такую установку, как эксперимент EPR-Bell, который мы раскрыли в разд. \ref{ch3.6}. Предполагается, что у Алисы и Боба есть «свобода воли» выбирать ориентацию своих поляризационных фильтров в любое время и в любом случае по своему усмотрению, и им нет необходимости консультироваться с кем-либо или чем-либо, чтобы принять свое (произвольное) решение. Квантовое состояние фотона, который они собираются исследовать, не должно зависеть от этих выборов, а также состояние фотона не должно зависеть от выборов, сделанных на другой стороне, или от результатов этих измерений.

Это означает, что результаты измерений должны быть определены каким-либо алгоритмом задолго до того, как будут выполнены фактические измерения, а также задолго до того, как был сделан выбор, что измерять. Именно этот алгоритм создает конфликты с ожиданиями, вычисленными в простом квантовом расчете. Это противоречит действительности, что означает, что может быть одно «фактическое» измерение, но было бы много возможных альтернативных измерений, измерений, которые фактически не выполняются, но результаты которых также должны были быть определены. Это то, что обычно называют контрфактуальной определенностью, и это по существу было опровергнуто простой логикой.

Теперь, как уже неоднократно указывалось, нарушение контрфактуальной определенности вовсе не является признаком, ограниченным квантовой теорией. В нашем примере планетной системы, разд. \ref{ch5.7.1}, нет априорного ответа на вопрос, в каком из двух собственных состояний оператора обмена EarthMars, в +1 или в -1, мы находимся. Это запрещенный контрфактивный вопрос. Но в случае с планетарной системой мы знаем, что такое beables (положения и скорости планет), тогда как в Стандартной модели мы этого не знаем. Там нелегитимность контрафактных заявлений не сразу очевидна. По сути, мы должны утверждать, что Алиса и Боб не имеют свободной воли для изменения ориентации своих фильтров; или, если мы говорим, что их решения должны принимать их корни и, предположим, они влияют на возможные состояния, в которых может находиться фотон. Короче говоря, Алиса и Боб принимают решения, которые коррелируют с поляризацией фотонов, как объяснено в разд. \ref{ch3.6}.

Более точное определение «свободной воли», как свободы выбора своего состояния в любой момент времени, которое должно использоваться в этих аргументах, было объяснено в разд. \ref{ch3.8}.

\subsubsection{Запутанность и Супердетерминизм}\label{ch5.7.3}

Часто возражение против общей философии, отстаиваемой в этой работе, состоит в том, что она никогда не позволит приспособиться к запутанным частицам. Однако внимательный читатель должен понять, что в принципе не должно быть проблем такого рода. Любое квантовое состояние можно рассматривать как шаблон, и эволюция этих шаблонов будет регулироваться реальным уравнением Шредингера. Если связь между онтологическим базисом и более традиционным выбором базиса достаточно сложна, мы столкнемся с superimposed (наложенными) состояниями всех видов, так что однозначно ожидаются также состояния, в которых частицы ведут себя как «количественно запутанные».

Таким образом, в принципе, легко записать ортонормированные преобразования, которые превращают онтические состояния в запутанные шаблонные состояния.

Однако есть некоторые проблемы и загадки. Парадокс ЭПР и теорема Белла являются примерами. Как объяснено в разд. \ref{ch3.6}, кажущиеся противоречия могут быть устранены только в том случае, если мы предположим довольно обширные корреляции между «скрытыми переменными» повсюду во Вселенной. Отображение онтических состояний в запутанные состояния, по-видимому, зависит от настроек, выбранных Алисой и Бобом в отдаленном будущем.
Кажется, что имеет место заговор: из-за некоторых чудесных корреляций в событиях в момент времени t = 0 пара фотонов «заранее знает», какими будут поляризационные углы фильтров, с которыми они столкнутся позже, и как они должны пройти через них. Где и как это вошло в наш формализм, и как достаточно естественная система без какого-либо заговора на классическом уровне порождает это странное поведение?

Это не только особенность запутанных частиц, которая может показаться очень проблематичной. Концептуальная трудность уже проявляется на гораздо более базовом уровне. Рассмотрим один фотон, независимо от того, запутан ли он с другими частицами или нет. Наше описание этого фотона в терминах beables предполагает, что эти библы ведут себя классически. То, что происходит позже, на поляризационном фильтре (фильтрах), также продиктовано классическими законами. Эти классические законы фактически диктуют, как мириады переменных колеблются в том, что мы называем шкалой Планка, или, точнее, в наименьшем масштабе, где можно распознать различимые физические степени свободы, которые могут быть или не быть близки к тому, что обычно называют планковским масштабом. Поскольку запутанные частицы возникают в реальных экспериментах, мы утверждаем, что базисные преобразования будут достаточно сложными, чтобы преобразовать состояния, являющиеся аттическими в масштабе Планка, в запутанные состояния.

Но это не ответ на поставленный вопрос. Обычный способ сформулировать вопрос - спросить, как передается «информация». Эта информация классическая или квантовая? Если это правда, что шаблоны являются в основном квантовыми шаблонами, мы склонны сказать, что передаваемая информация является квантовой информацией. И все же она сводится к классической информации в масштабе Планка, и обычно считалось, что это невозможно.

Это должно быть ошибкой. Как мы видели в разд. \ref{ch3.6}, основное техническое противоречие исчезнет, если мы предположим, что существуют сильные корреляции между «квантовыми» флуктуациями, включая вакуумные флуктуации, на всех масштабах расстояний (включая корреляции между флуктуациями, генерируемыми в квазарах, которые разделены миллиардами световых лет). Мы думаем, что дело в следующем. Когда мы используем шаблоны, мы заранее не знаем, какой базис следует выбрать, чтобы они выглядели как онтологические степени свободы, насколько это возможно. Для фотона, проходящего через поляризационный фильтр, базис, ближайший к онтологическому, является базисом, в котором выбираются координаты для выравнивания с фильтром. Но этот фотон мог быть испущен квазаром миллиарды лет назад, как квазар узнал, что такое онтологический базис?

Ответ заключается в том, что квазар действительно знает, что такое онтологический базис, потому что наша теория распространяется и на эти квазары. Информация «это онтологическое состояние, а любой набор наложенных (superimposed) состояний - нет» является частью информации, которая, согласно нашей теории, абсолютно консервативна во времени. Итак, если это окажется базисом сейчас, то это было базисом и миллиарды лет назад. Кажется, что квазары замышляют заговор, чтобы одурачить наших экспериментаторов, но в действительности они просто соблюдают закон сохранения: информация о том, какие квантовые состояния образуют онтологический базис, сохраняется во времени. Подобно закону момента импульса или любому другому точно законсервированному объекту, этот закон сохранения говорит нам, что будет с переменной в будущем, если она известна в прошлом, и наоборот.

Эта же особенность может быть проиллюстрирована мысленным экспериментом, где мы измеряем флуктуации фотонов, излучаемых квазаром, но сначала мы посылаем фотоны через поляризационный фильтр. Фотографии, которые мы сделаем, будут классическими объектами. Здесь также мы должны сделать вывод, что испущенные квазаром фотоны уже «знали», каковы были их поляризации, когда они уходили миллиарды лет назад. Это не заговор, это просто следствие нашего закона сохранения: онтологические состояния, именно только онтологические состояния, эволюционируют в другие онтологические состояния.
Мы должны сделать вывод, что, если кажется, что в нашем квантовом описании реальности существует заговор, то это следует рассматривать как особенность наших квантовых методов, а не физической системы, на которую мы смотрим. В классическом описании клеточного автомата нет заговора. Кажущийся заговор - это особенность, а не ошибка.

Ответ, данный здесь, часто называют супердетерминизмом. Идея заключается в том, что Алиса и Боб могут выбирать только измеряемые онтологические состояния, а не наложенные (superimposed) состояния, которые мы используем в качестве шаблонов. В некотором смысле их действия были предопределены, но, конечно, совершенно ненаблюдаемым образом. Супердетерминизм выглядит странно, только если придерживаться описания запутанных частиц как квантовых систем, описываемых их квантовыми состояниями. Онтологическое описание не использует квантовые состояния. В этом описании частицы ведут себя нормальным, причинным образом. Однако мы должны помнить, что эти частицы и все остальное, включая умы Боба и Алисы, тесно взаимосвязаны. В настоящее время они коррелируют так же сильно, как и при излучении фотонов источником, как это было установлено\footnote{Функция mousedropping не достигает $100 \%$ корреляции. Эта особенность обусловлена несколько искусственным характером модели, используемой для детектирования. Если мы используем более реалистичные модели для детектирования, то функция отбросов мыши может состоять из более резких пиков.} в функции mousedropping, рис. \ref{i3.2} в разд. \ref{ch3.7}.
В разделе \ref{ch3.8}, было объяснено в явном физическом смысле, что на самом деле должно означать «свобода воли» и как она соотносится с супердетерминизмом.

\subsubsection{Принцип суперпозиции в квантовой механике}\label{ch5.7.4}

Что именно произошло с принципом суперпозиции в CA Интерпретация квантовой механики? Критики нашей работы выдвинули предположение, что CAI запрещает суперпозицию, в то время как принцип суперпозиции, очевидно, служит достаточно хорошей основой для квантовой механики. Многочисленные эксперименты подтверждают, что если у нас есть два разных состояния, также может быть реализована любая суперпозиция этих состояний. Хотя читатель уже должен был понять, как ответить на этот вопрос, попробуем еще раз прояснить ситуацию.
На самом базовом уровне физического закона мы предполагаем возникновение только онтологических состояний, и любая их суперпозиция в принципе не соответствует онтологическому состоянию. В лучшем случае суперпозиция может использоваться для описания вероятностных распределений состояний (мы называем эти «состояниями шаблонов», которые используются, когда у нас нет точной информации, чтобы с абсолютной уверенностью определить, какое онтологическое состояние мы рассматриваем). В нашем описании Стандартной модели или любой другой известной физической системы, такой как атомы и молекулы, мы используем не онтологические состояния, а шаблоны, которые можно рассматривать как суперпозиции онтологических состояний. Атом водорода является шаблоном, все элементарные частицы, о которых мы знаем, являются шаблонами, и это означает, что волновая функция вселенной, которая является онтологическим состоянием, должна быть суперпозицией наших шаблонов. Какая суперпозиция? Что ж, мы встретим много разных суперпозиций при выполнении повторных экспериментов. Это объясняет, почему мы поверили, что все суперпозиции всегда разрешены.

Но не буквально все суперпозиции могут возникнуть. Суперпозиции искусственные. Наши шаблоны являются суперпозициями, но это потому, что они представляют только очень крошечный сектор гильбертова пространства, который мы понимаем сегодня. В то время вся вселенная находится только в одном онтологическом состоянии, и она, конечно, не может входить в свои суперпозиции. Этот факт теперь становится очевидным, когда мы рассматриваем «классический предел». В классическом пределе мы снова имеем дело с определенностью. Классические состояния также онтологические. Когда мы проводим измерение, сравнивая вычисленное «шаблонное состояние» с онтологическими классическими состояниями, которые мы ожидаем в конце, мы восстанавливаем вероятности, беря норму квадрата амплитуд.
Похоже, что многим ученым это трудно принять. В течение целого столетия нам промывали мозги тем, что суперпозиции встречаются повсюду в квантовой механике. В то же время нам сказали, что если вы попытаетесь наложить классические состояния, вы получите вместо этого вероятностные распределения. Именно здесь наша нынешняя теория является более точной: если бы мы точно знали волновую функцию вселенной, мы бы обнаружили, что она всегда развивается только в одно классическое состояние, без неопределенностей и без суперпозиций.

Конечно, это не означает, что стандартная квантовая механика была бы неправильной. Наши знания о шаблонных состояниях и о том, как они развиваются, сегодня очень точны. Только потому, что еще не известно, как связать эти состояния шаблонов с онтологическими состояниями, мы должны выполнять суперпозиции все время, когда выполняем квантово-механические вычисления. Они приводят к статистическим распределениям в наших окончательных предсказаниях, а не к определенности. Это может измениться только в том случае, если мы найдем онтологические состояния, но, поскольку ожидается, что даже вакуумное состояние будет шаблоном, и как сложная суперпозиция несчетного числа онтических состояний, мы должны ожидать, что квантовая механика останется с нами навсегда - но как математический инструмент, а не как мистический отход от обычной, «классической» логики.

\subsubsection{Вакуумное состояние}\label{ch5.7.5}

Является ли вакуумное состояние онтологическим состоянием? Состояние вакуума обычно определяется как состояние с наименьшей энергией. Это также означает, что никакие частицы не могут быть найдены в этом состоянии просто потому, что частицы представляют энергию, а в вакуумном состоянии у нас недостаточно энергии даже для того, чтобы допустить присутствие одной частицы.

Дискретизированный гамильтониан введен только в разд. \ref{ch19} часть II. Это возможно, но, будучи дискретным, в лучшем случае это может быть лишь грубое приближение квантового гамильтониана, а его низшее энергетическое состояние сильно вырождено. Таким образом, этого недостаточно для определения вакуума. Более того, гамильтониан, определенный в разд. \ref{ch19.2} квантуется в единицах, которые кажутся такими же большими, как масса Планка. Будет ясно, что гамильтониан, который будет использоваться в любом реалистическом уравнении Шредингера, имеет гораздо более плотный, в основном непрерывный спектр собственных значений. Квантовый гамильтониан определенно не является правдоподобным, как мы объясняли ранее в разд. \ref{ch5.6.3}. Следовательно, вакуум не является онтологическим состоянием.
На самом деле, согласно квантовым теориям поля, вакуум содержит много виртуальных частиц или пар частиц и частиц, которые колеблются в пространстве и существуют везде в пространстве-времени. Это типично для квантовых суперпозиций онтологических состояний. Кроме того, самые легкие частицы в наших теориях намного легче, чем масса Планка. Они не являются онтологическими, и требование их отсутствия в нашем вакуумном состоянии неизбежно превращает сам наш вакуум также в неонтологическое состояние.

Это замечательно, потому что наше вакуумное состояние имеет еще одно своеобразное свойство: его энергетическая плотность почти полностью исчезает. Из-за космологической константы в нашем вакуумном состоянии есть энергия, но она составляет всего лишь около 6 протонов на кубический метр, что крайне мало, учитывая тот факт, что в  физике частиц действо происходит в масштабах гораздо меньших метра. Это очень маленькое, но неисчезающее число - одна из самых больших загадок Природы.

И все же вакуум кажется неонтологическим, так что он должен быть местом, полным активности. Как согласовать все эти явно противоречивые черты, совсем не понятно\footnote{Здесь есть место для множества спекуляций. Во-первых, вакуумное состояние, как шаблонное состояние, представляет собой суперпозицию очень многих онтических состояний, каждое из которых порождает лишь незначительное количество гравитации, так что вакуум остается практически плоским. Мы решили не углубляться дальше в этот вопрос, пока квантовая гравитация не будет лучше понята.}.

Колебания вакуума можно рассматривать как одну из основных причин неисчезающих, нелокальных корреляций, таких как функция мышинных каках в разд. \ref{ch3.6}. Без флуктуаций вакуума было бы трудно понять, как можно поддерживать эти корреляции.

\subsubsection{Замечание о шкалах }\label{ch5.7.6}

Ранее мы поднимали вопрос о том, чем наш квантовый мир отличается от более классической хаотической системы, такой как газ Ван-дер-Ваальса. Есть один важный аспект, который на самом деле может пролить свет на некоторые «квантовые особенности», с которыми мы сталкиваемся.

Картина нашего мира выглядит следующим образом. Представьте себе экран, отображающий эволюцию нашего клеточного автомата. Мы представляем, что его пиксели имеют размер примерно одну планковскую длину, $10^{-33}$ см. Кажется, что возникают все возможные состояния, поэтому наш экран может отображать в основном только белый шум. Теперь взглянем на масштаб атомов, молекул и субатомных частиц, который составляет примерно от $10^{-8}$ до $10^{-15}$ см, то есть, примерно на 10 - 20 порядков больше. Это как если бы мы смотрели на типичный экран компьютера примерно с расстояния одного светового года.

А теперь представьте, что мы инвертируем один пиксель из 0 в 1 или обратно, не касаясь ни одного из его соседей. Это действие оператора, который модифицирует энергию системы, как правило, на одну единицу энергии Планка, или на кинетическую энергию самолета среднего размера. Следовательно:

\textit{изменение одного пикселя оказывает огромное влияние на исследуемое состояние.}

Если мы хотим сделать менее масштабные энергетические изменения, нам нужно инвертировать информацию в области с гораздо меньшим количеством энергии, или, как правило, в тысячи раз больше, Планковской площади. Это значит, что

\textit{Состояния, которые вероятней встретить в обычных системах, потребуют изменения миллионов пикселей, а не одного.}

Не совсем очевидно, что из этого следует. Вполне возможно, что мы должны связать это наблюдение с нашими идеями о потере информации: при внесении изменений, представляющих информацию, которую нелегко потерять, мы обнаружим, что задействованы миллионы пикселей (микросостояний).

Наконец, когда мы применяем оператор, воздействующий на $(10^{20})^3$ пикселя или около того, мы достигаем состояния, когда крошечный атом или молекула переворачивается в другое квантовое состояние. Таким образом, хотя наша система и является детерминированной, запрещено изменять только один пиксель; такого никак не реализовать в простых квантовых экспериментах. Важно осознать этот факт при обсуждении вакуумных корреляционных функций в связи с теоремой Белла и аналогичными темами.

\subsubsection{Экспоненциальный спад}\label{ch5.7.7}

Вакуумные флуктуации должны быть основной причиной, почему изолированные системы, такие как атомы и молекулы в пустом пространстве, показывают типичные квантовые особенности. Особое квантово-механическое свойство многих частиц заключается в том, как они могут распадаться на две или более других частиц. Почти всегда этот распад следует идеальному закону экспоненциального распада: вероятность $P(t)$ того, что частица данного типа еще не распалась после истечения времени t, подчиняется правилу

\begin{equation}\label{5.29}
	\frac{\mathrm{d} P(t)}{\mathrm{d} t}=-\lambda P(t), \quad \rightarrow \quad P(t)=P(0) e^{-\lambda t}
\end{equation}
                        
где $\lambda$ - коэффициент, который часто вообще не зависит от внешних обстоятельств или от времени. Если мы начнем с $N_0$ частиц, то ожидаемое значение $\langle N(t) \rangle$ числа частиц $N (t)$ после времени t следует тому же закону:


\begin{equation}\label{5.30}
	\langle N(t) \rangle = N_0 e^{-\lambda t}
\end{equation}

             
Если существуют различные режимы распада частиц, то мы имеем $\lambda = \lambda_1 + \lambda_2 + ...$, и отношения $\lambda_i$ равны отношениям наблюдаемых распадов.

А как это объяснить в детерминированной теории, такой как клеточный автомат? В общем, это было бы невозможно, если бы вакуум был единственным онтологическим состоянием. Рассмотрим частицы данного типа. Каждая отдельная частица будет распадаться через разное время, точно в соответствии с формулой. (\ref{5.29}). Кроме того, направления, в которых разлетятся продукты распада, будут различными для каждой отдельной частицы, в то время как, если задействовано три или более продукта распада, распределение между ними энергии тоже будет иметь вероятностный характер. Для многих существующих частиц эти распределения могут быть точно рассчитаны в соответствии с предписаниями квантовой механики.

В детерминистской теории все эти различные режимы распада должны были бы соответствовать различным начальным состояниям. Было бы невозможно, чтобы каждая отдельная частица вела себя как "планерное решение" клеточного автомата, поскольку все эти различные функции распада должны были бы быть представлены различными планерами. Нетрудно понять, что должны были бы существовать миллионы или миллиарды различных режимов планера.

Единственное объяснение этой особенности должно состоять в том, что частицы окружены вакуумом, который каждый раз находится в другом онтологическом состоянии. Радиоактивная частица постоянно подвергается воздействию колеблющихся элементов в окружающем ее вакууме. Это постоянное внешней воздействие, как таковое, должно быть представлено почти идеальными генераторами случайных чисел. Таким образом, закон распада (\ref{5.29}) возникает естественным образом.

Таким образом, неизбежно, что вакуумное состояние должно рассматриваться как одно шаблонное состояние, которое, однако, состоит из бесконечного числа онтологических состояний. Состояния, состоящие из одной частицы, бродящей в вакууме, образуют простой набор различных шаблонных состояний, ортогональных к вакуумному шаблонному состоянию, как диктуется в описании пространства Фока состояний частиц в квантовой теории поля.


\subsubsection{Один фотон, проходящий через последовательность поляризаторов}\label{ch5.7.8}

Наиболее показательным будет свести проблему к ее самой основной форме. Концептуальная трудность, которую каждый чувствует в эксперименте Белла, начинает проявляться, если рассмотреть один фотон, проходящий через последовательность поляризационных фильтров $F_1, ..., F_N$. Представьте, что эти фильтры вращаются на углы $\phi_1, \phi_2, ..., \phi_N$, и каждый раз, когда фотон попадает в один из этих фильтров, скажем, $F_n$, есть вероятность, равная $\sin^2 \psi_n$ с $\psi_n = \varphi_{n-1} - \varphi_n$, что фотон поглощается этим фильтром. Таким образом, фотон может быть поглощен любым поляризатором. Как формализовать подобную установку?

Обратите внимание, что описанная здесь установка напоминает наше описание радиоактивной частицы, см. Предыдущий подраздел. Там мы предположили, что частица непрерывно взаимодействует с окружающим вакуумом. Здесь проще всего предположить, что фотон взаимодействует со всеми фильтрами. Тот факт, что фотон попадает на фильтр $F_n$ в виде суперпозиции двух состояний, одно из которых будет проходить, а другое поглотится, означает, что на языке онтологической теории у нас есть начальное состояние, которого мы не совсем знаем; есть вероятность $\cos^2 \psi_n$, что у нас есть фотон проходящего типа, и вероятность $\sin^2 \psi_n$ того, что он будет поглощен. Если фотон проходит, его поляризация хорошо определена, как $\varphi_n$. Это будет определять распределение возможных состояний, которые могут проходить или не проходить через следующий фильтр.

Мы заключаем, что фильтр, будучи по существу классическим, может находиться в очень большом количестве различных онтологических состояний. Простейшая из всех онтологических теорий гласит, что фотон приходит с углом поляризации $\psi_n$ относительно фильтра. В зависимости от онтологического состояния фильтра у нас есть вероятность $\cos^2 \psi_n$, что фотон пропущен, но повернут в направлении $\varphi_n$, и вероятность $\sin^2 \psi_n$, что он поглощен (или отражен и повернут).

Теперь вернемся к двум запутанным фотонам, наблюдаемым Алисой и Бобом в установке Белла. Вот что происходит в простейшей из всех онтологических теорий: фильтры Алисы и Боба действуют точно так же, как и выше. Два фотона несут в себе одну и ту же информацию в форме одного угла, $c$. Фильтр Алисы имеет угол $a$, у Боба угол $b$.Как мы видели в разд. \ref{ch3.7}, существует трехточечная корреляция между $a$, $b$ и $c$, определяемая функцией mousedropping (\ref{3.23}) и рис. \ref{i3.2}.

Теперь обратите внимание, что функция mousedropping инвариантна при поворотах $a$, $b$ и / или $c$ на 90 °. Природа онтологического состояния очень точно зависит от углов $a$, $b$ и $c$, но каждое из этих состояний отличается от других на кратность 90 ° в этих углах. Следовательно, как только мы выберем желаемый ортонормированный базис, элементы базиса будут полностью некоррелированы. Это делает корреляции ненаблюдаемыми, когда мы работаем с шаблонами. Предполагая, что здесь действует закон сохранения онтологий, мы обнаруживаем, что онтологическая природа углов $a$, $b$ и $c$ коррелирует, но они не является физическими наблюдаемыми. Следует ожидать, что такого рода корреляции будут пронизывать всю физику.

Наше описание фотона, проходящего через последовательность поляризационных фильтров, требует, чтобы онтологическое начальное состояние включало информацию о том, какой из фильтров фактически поглощает (или отражает) фотон. Согласно стандартной квантовой механике, это в принципе непредсказуемо. По-видимому, это означает, что точное онтологическое состояние исходного фотона не может быть известно, до измерения. Это делает наши «скрытые переменные» невидимыми. Из-за заметных корреляционных функций этих начальных состояний наблюдатель скрытой переменной будет иметь доступ к информации, которая запрещена копенгагенской доктриной. Мы подозреваем, что это является особым - и очень важным - свойством клеточного автомата.


\subsubsection{Двухщелевой эксперимент}\label{ch5.7.9}

Теперь, когда у нас есть представление о том, как следует объяснять квантовую механику в терминах клеточного автомата, можно рассмотреть такие штуки, как эксперимент с двумя щелями. На самом деле имеет смысл рассмотреть более общие оптические настройки с экранами и отверстиями в них, линзами, поляризаторами, двулучепреломляющими устройствами, разделителями Штерна-Герлаха и т. д.

Общий вопрос заключается в том, что нужно понять, как заданные $| in\rangle$ состояния приводят к выходным $| out\rangle$ состояниям. Более конкретный вопрос заключается в том, как в рамках нашей модели получить интерференционные картины и, в частности, как они будут зависеть от фазовых углов, которые обычно рассматриваются как типичные квантовые явления.
Что касается первого вопроса, то наша общая теория гласит, что число возможных состояний огромно, и между ними могут происходить переходы. Как было объяснено ранее, амплитуды, которые получены в конце, фактически представляют вероятности, которые были заложены, когда мы строили начальные состояния; Вот как в конечном итоге появились вероятности Борна.

Фазы возникают прежде всего при рассмотрении зависимости состояний от времени. Все онтологические состояния постулировались для подчинения эволюционным уравнениям, где операторы эволюции $U(t)$ были записаны как $e^{-iHt}$, где $H$ - гамильтониан. В наших упрощенных онтологических моделях мы видим, что в действительности означают эти фазовые углы: если наша система имеет тенденцию становиться периодической после времени T, фазовый угол $e^{-iHt}$ возвращается к единице. Таким образом, фаза указывает положение онтологической переменной на ее периодической орбите.

Это должно все объяснить. Все онтологические переменные состоят из базовых элементов , периодических по времени. Вопрос о вероятности превращения данного состояния в другое конкретное состояние зависит от того,  куда оно попадает на своей периодической орбите. Если существует два или более путей от заданного внутреннего состояния к заданному внешнему состоянию, вероятность возрастает, когда эти два пути находятся в фазе, и уменьшается, если они по фазе отличаются. Конечно, это верно, если эти онтологические переменные были бы классическими волнами, и тогда это стандартное явление интерференции, такое как в случае с фотонами. Онтологические переменные, связанные с фотонами, по существу являются полями Максвелла. Теперь мы видим, что это более общая истина. Все онтологические переменные в их наиболее первозданной форме, по-видимому, должны быть периодическими во времени, и если существует много способов для развития одного онтологического состояния в другое онтологическое состояние, то вероятности зависят от степени, в которой один фазовый угол достигается более различными способами (более вероятно), чем противоположный фазовый угол.

Мы подчеркиваем, что это не очень тривиальная формулировка для классических процессов. То, что мы получаем в итоге, - это конечная физика, близкая к планковским шкалам, где многое из того, что происходит, будет новым для физиков.


\subsection{Квантовый компьютер}\label{ch5.8}

Квантовая механика часто наделена таинственными чертами. Предпринимаются активные попытки получить из них преимущества. Один из примеров - квантовый компьютер. Идея состоит в том, чтобы использовать запутанные состояния носителей информации, которые могут быть фотонами, электронами или чем-то еще, чтобы представлять значительно больше информации, чем обычные биты и байты.
Поскольку машины, которые планируют построить исследователи, будут подчиняться обычной квантовой механике, они должны вести себя полностью в соответствии с нашими теориями и моделями. Однако это, кажется, приводит к противоречиям.

В отличие от обычных компьютеров, объем информации, который может переноситься кубитами в квантовом компьютере, в принципе увеличивается экспоненциально с увеличением количества ячеек, и, следовательно, ожидается, что квантовые компьютеры смогут выполнять вычисления, которые в корне невозможно в обычных компьютерах. Обычный классический компьютер никогда не сможет победить квантовый компьютер, даже если в принципе он принимает размер вселенной. 

Наша проблема заключается в том, что наши модели, лежащие в основе квантовой механики, являются классическими, и поэтому они могут имитироваться классическими компьютерами, даже если экспериментатор построит «квантовый компьютер» в таком мире. Что-то не так.
Однако квантовые компьютеры до сих пор не созданы. Кажется, есть много практических трудностей. Одна из трудностей - это почти неизбежное явление декогеренции. Чтобы квантовый компьютер функционировал безупречно, нужны идеальные кубиты.

Общепринято, что нельзя получить идеальные кубиты, но можно, нивелировать ошибки отдавая часть кубитов на борьбу с ними. На обычном компьютере ошибки могут быть легко исправлены с помощью небольшого избытка информации для проверки неисправных участков памяти. Можно ли исправить ошибки кубитов? Есть утверждения, что это можно сделать, но, несмотря на это, у нас все еще нет работающего квантового компьютера, не говоря уже о квантовом компьютере, который может побить все классические компьютеры. Наша теория приходит с твердым прогнозом:

\textit{Да, хорошо используя квантовые технологии, в принципе можно будет построить компьютер, значительно превосходящий обычные компьютеры, но нет, они не смогут функционировать лучше, чем классический компьютер, если бы его ячейки памяти были уменьшены до одной на элемент объема Планка (или, возможно, с учетом голографического принципа, один участок памяти на элемент поверхности Планка), и, если скорость их обработки будет соответственно увеличена, до одной операции на единицу времени Планка ($10^{-43}$ секунды).}


Такие масштабные классические компьютеры, конечно, не могут быть построены, так что этому квантовому компьютеру все еще будет позволено совершать вычислительные чудеса, но разделение числа с миллионами цифр на его основные множители будет невозможно - если только не появятся фундаментально улучшенные классические алгоритмы. Если инженерам когда-либо удастся создать такие квантовые компьютеры, мне кажется, что CAT будет сфальсифицирована; никакая классическая теория не может объяснить совершенные квантовые компьютеры.


\end{document}