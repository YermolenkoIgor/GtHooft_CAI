\documentclass[main.tex]{subfiles}
\begin{document}


\section{Подробнее о зубчатых колесах}\label{ch12}

\subsection{Группа \textit{SU} (2) и гармонический ротатор}\label{ch12.1}

Вернемся к исходному зубчатому колесу с $N$ зубьями, как это представлено в разд. \ref{ch2.2}. Может быть очень полезно определить постоянную $\ell = (N-1)/2$ и ввести операторы $L_1$, $L_2$ и $L_3$ следующим образом:

\begin{equation}\label{12.1}
	\begin{aligned} L_{3} &=\frac{N}{2 \pi} H_{\mathrm{op}}-\ell=k-\ell \\ L_{1} &=\frac{1}{2}\left(L_{+}+L_{-}\right), \quad L_{2}=-\frac{1}{2} i\left(L_{+}-L_{-}\right) \\ L_{+}|k\rangle_{H} &=\sqrt{(k+1)(2 \ell-k)} | k+1\rangle_{H} \\ L_{-}|k\rangle_{H} &=\sqrt{k(2 \ell+1-k)} | k-1\rangle_{H} \end{aligned}
\end{equation}
$k = 0, 1, \ldots, 2\ell$ - главное квантовое число; $\delta t = 1$ - шаг по времени.

Используя квантовое число $m = k - \ell = L_3$, получаем более привычные выражения для операторов углового момента $L_a$, $a = 1, 2, 3$, подчиняющихся коммутационным соотношениям

\begin{equation}\label{12.2}
	\left[L_{a}, L_{b}\right]=i \varepsilon_{a b c} L_{c}
\end{equation}
            
Исходные онтологические состояния $\mid n \rangle_{ont}$ могут быть получены из состояний углового момента с помощью правил преобразования (\ref{2.21}) и (\ref{2.22}). Только они развиваются как онтологические состояния. Однако могут быть очень полезны другие операторы . Взять, к примеру,
   
\begin{equation}\label{12.3}
	x=\frac{1}{\sqrt{\ell}} L_{1}, \quad p=-\frac{1}{\sqrt{\ell}} L_{2}, \quad[x, p]=i\left(1-\frac{2 \ell+1}{2 \pi \ell} H_{\mathrm{op}}\right)
\end{equation}

тогда для состояний, где энергия $\langle H_{\mathrm{op}} \rangle \ll 1$, мы имеем знакомые правила коммутации для положений $x$ и импульсов $p$, тогда как из соотношения $L_1^2 + L_2^2 + L_3^2 = L^2 = \ell(\ell +1)$ следует, что, когда $\langle H_{\mathrm{op}} \rangle \ll 1/\sqrt{\ell}$,

\begin{equation}\label{12.4}
	H \rightarrow \frac{2 \pi}{N} \frac{1}{2}\left(p^{2}+x^{2}-1\right)
\end{equation}
         
что является гамильтонианом для гармонического осциллятора (энергия нулевой точки была вычтена, поскольку собственное состояние с самой низкой энергией было установлено равным нулю). Кроме того, на низких значениях квантового числа $k$, мы видим, что $L_\pm$ приближаются к операторам создания и уничтожения гармонического осциллятора (см. Уравнения (\ref{12.1})):

\begin{equation}\label{12.5}
	L_{-} \rightarrow \sqrt{2 \ell+1} a, \quad L_{+} \rightarrow \sqrt{2 \ell+1} a^{\dagger}
\end{equation}
                        
Таким образом, мы видим, что самые низкие энергетические состояния зубчатого колеса приближаются к самым низким энергетическим состояниям гармонического осциллятора. Это будет очень полезным наблюдением, если мы хотим построить модели для квантовых теорий поля, начиная с детерминированных зубчатых колес. Модель, описанная уравнениями (\ref{12.1}) - (\ref{12.3}) будет называться гармоническим ротатором. Зеемановский атом (разд. \ref{ch2.2}) - частный случай с $\ell = 1$.

Обратите внимание, что спектр гамильтониана гармонического ротатора точно такой же, как у гармонического осциллятора, за исключением того, что существует верхний предел $H_{\mathrm{op}} < 2\pi$. По конструкции, период $T = (2\ell + 1)\delta t$ гармонического ротатора такой же, как и у гармонического осциллятора, и точно такой же, как у периодического зубчатого колеса.

Гамильтониан, который мы связываем с гармоническим ротатором, также подходит для вращающегося объекта, который проявляет прецессию из-за крутящего момента на его оси. Таким образом, физически мы видим, что осциллятор, рисующий круги в своем $(x, p)$ фазовом пространстве, здесь заменяется вершиной прецессии. На самых низких энергетических уровнях они подчиняются тем же уравнениям.

Из этого раздела мы заключаем, что зубчатое колесо с $N$ состояниями можно рассматривать как представление группы $SU$(2) с полным угловым моментом $\ell$ и $N = 2\ell + 1$. Важность этого подхода состоит в том, что представление является унитарным, и что есть естественное основное состояние. Основное состояние гармонического генератора. В отличие от гармонического осциллятора, гармонический ротатор также имеет верхнюю границу своего гамильтониана. Обычные операторы уничтожения и создания, $a$ и $a^\dagger$, заменяются $L_-$ и $L_+$, коммутатор которых уже не постоянен, а пропорционален $L_3$, и, следовательно, меняет знак для состояний $\mid k \rangle$ с $\ell <k <2\ell$. Это изменение знака гарантирует, что спектр ограничен как снизу, так и сверху, как следствие модифицированной алгебры (\ref{12.2}).

\subsection{Бесконечные, дискретные зубчатые колеса}\label{ch12.2}

Дискретные модели с бесконечным числом состояний могут иметь новую особенность, заключающуюся в том, что некоторые орбиты могут не быть периодическими. Затем они содержат по крайней мере одну непериодическую «стойку». Для этого общего случая существует универсальное определение квантового гамильтониана, хотя оно не является уникальным. Определив оператор эволюции с обратимым по времени наименьшим дискретным шагом по времени как оператор $U_{\mathrm{op}}(1)$, мы теперь построим простейший гамильтониан $H_{\mathrm{op}}$, такой что $U_{\mathrm{op}}(1) = e^{-iH_{\mathrm{op}}}$. Для этого мы используем эволюцию за $n$ шагов, где $n$ положительно или отрицательно:

\begin{equation}\label{12.6}
	U_{\mathrm{op}}(n)=U_{\mathrm{op}}(1)^{n}=e^{-i n H_{\mathrm{op}}}
\end{equation}
         
Предположим, что собственные значения $\omega$ этого гамильтониана лежат между 0 и $2\pi$. Затем мы можем рассмотреть гамильтониан в том случае, когда $U(1)$ и $H$ диагональны. Запишем

\begin{equation}\label{12.7}
	e^{-i n \omega}=\cos (n \omega)-i \sin (n \omega)
\end{equation}
          
и воспользуемся преобразованием Фурье, чтобы получить

\begin{equation}\label{12.8}
	x=2 \sum_{n=1}^{\infty} \frac{(-1)^{n-1} \sin (n x)}{n}
\end{equation}
при $-\pi < х < \pi$.

Затем запишем $H_{\mathrm{op}} = \omega = x + \pi$, чтобы получить из (\ref{12.8})              

\begin{equation}\label{12.9}
	\omega=\pi-2 \sum_{n=1}^{\infty} \frac{\sin (n \omega)}{n}
\end{equation}
           
Следовательно, как в формуле (\ref{2.8}),

\begin{equation}\label{12.10}
	\begin{aligned} \omega &=\pi-\sum_{n-1}^{\infty} \frac{i}{n}(U(n \delta t)-U(-n \delta t)) \text { and } \\ H_{\mathrm{op}} &=\pi-\sum_{n-1}^{\infty} \frac{i}{n}\left(U_{\mathrm{op}}(n \delta t)-U_{\mathrm{op}}(-n \delta t)\right) \end{aligned}
\end{equation}

Очень часто мы не будем довольствоваться этим гамильтонианом, так как он не имеет собственных значений за пределами диапазона $(0,2\pi)$. Как только в арсенале появляются сохраняющиеся величины, то можно добавить их функции по желанию к гамильтониану для сравнения, что часто делается с химическими потенциалами. Клеточные автоматы в целом будут иметь много таких законов сохранения. См. Рис. \ref{i2.3}, где каждая замкнутая орбита представляет собой нечто, что сохраняется: метка орбиты.

В разделе \ref{ch13}, мы рассматриваем другой предел континуума, $\delta t \rightarrow 0$ для модели зубчатого колеса. Для начала рассмотрим континуальные теории более широко.

\subsection{Автоматы, которые непрерывны во времени}\label{ch12.3}

В физическом мире у нас нет прямых указаний на то, что время действительно дискретно. Поэтому заманчиво рассмотреть предел $\delta t \rightarrow 0$. На первый взгляд можно подумать, что этот предел должен быть таким же, как и наличие постоянной степени свободы, подчиняющейся дифференциальным уравнениям во времени, но это не совсем так, как будет объяснено позже в этой главе. Сначала в этом разделе мы рассмотрим строго непрерывные детерминированные системы. Затем мы сравниваем их с континуальным пределом дискретных систем.

Рассмотрим онтологическую теорию, описываемую наличием непрерывного многомерного пространства степеней свободы $\vec q(t)$, зависящего от одной непрерывной переменной времени $t$, и его эволюцией во времени с использованием классических дифференциальных уравнений:

\begin{equation}\label{12.11}
	\frac{\mathrm{d}}{\mathrm{d} t} q_{i}(t)=f_{i}(\vec{q})
\end{equation}

где $f_i(\vec q)$ может быть почти любой функцией переменных $q_j$.

Примером является описание массивных объектов, подчиняющихся классической механике в $N$ измерениях. Пусть $a = 1,\ldots, N$ и $i = 1,\ldots,2N$:

\begin{equation}\label{12.12}
	\begin{aligned}\{i\}=\{a\} \oplus\{a+N\}, & q_{a}(t)=x_{a}(t), \quad q_{a+N}(t)=p_{a}(t) \\ f_{a}(\vec{q})=\frac{\partial H_{\mathrm{class}}(\vec{x}, \vec{p})}{\partial p_{a}}, & f_{a+N}(\vec{q})=-\frac{\partial H_{\mathrm{class}}(\vec{x}, \vec{p})}{\partial x_{a}} \end{aligned}
\end{equation}

где $H_{class}$ - классический гамильтониан.

Другим примером является квантовая волновая функция частицы в одном измерении:

\begin{equation}\label{12.13}
	\{i\}=\{x\}, \quad q_{i}(t)=\psi(x, t) ; \quad f_{i}(\vec{q})=-i H_{S} \psi(x, t)
\end{equation}

где теперь $H_S$ - гамильтониан Шредингера. Отметим, однако, что в этом случае функция $\psi(x, t)$ будет рассматриваться как онтологический объект, так что уравнение Шредингера и полученный в конечном итоге гамильтониан будут весьма отличаться от уравнения Шредингера, с которого мы начинаем; на самом деле это будет больше похоже на соответствующую вторично квантованную систему (см. позже).

Мы сейчас заинтересованы в превращении уравнения (\ref{12.11}) в квантовую систему путем изменения обозначений, а не физики. Тогда онтологическим базисом является множество состояний $\mid \vec q \rangle$, подчиняющихся свойству ортогональности

\begin{equation}\label{12.14}
	\left\langle\vec{q} | \vec{q}^{\prime}\right\rangle=\delta^{N}\left(\vec{q}-\vec{q}^{\prime}\right)
\end{equation}
           
где $\delta$ - теперь дельта-распределение Дирака, а $N$ - размерность векторов $\vec q$.

Если бы мы написали

\begin{equation}\label{12.15}
	\frac{\mathrm{d}}{\mathrm{d} t} \psi(\vec{q}) \stackrel{?}{=} -f_{i}(\vec{q}) \frac{\partial}{\partial q_{i}} \psi(\vec{q}) \stackrel{?}{=} -i H_{\mathrm{op}} \psi(\vec{q})
\end{equation}

где индекс $i$ суммируется, то тогда бы

\begin{equation}\label{12.16}
	H_{\mathrm{op}} \stackrel{?}{=}-i f_{i}(\vec{q}) \frac{\partial}{\partial q_{i}}=f_{i}(\vec{q}) p_{i}, \quad p_{i}=-i \frac{\partial}{\partial q_{i}}
\end{equation}

Это, однако, не совсем правильный гамильтониан, поскольку он может нарушать герметичность: $\hat H \neq \hat H^\dagger$. Правильный гамильтониан получается, если положить, что вероятности сохраняются, так что в случае, когда якобиан функции $\vec f (\vec q)$ не обращается в нуль, интеграл $\int d^N \vec q \psi^\dagger(\vec q)\psi(\vec q)$ все еще сохраняется:

\begin{equation}\label{12.17}
	\begin{aligned} \frac{\mathrm{d}}{\mathrm{d} t} \psi(\vec{q}) &=-f_{i}(\vec{q}) \frac{\partial}{\partial q_{i}} \psi(\vec{q})-\frac{1}{2}\left(\frac{\partial f_{i}(\vec{q})}{\partial q_{i}}\right) \psi(\vec{q})=-i H_{\mathrm{op}} \psi(\vec{q}) \\ H_{\mathrm{op}} &=-i f_{i}(\vec{q}) \frac{\partial}{\partial q_{i}}-\frac{1}{2} i\left(\frac{\partial f_{i}(\vec{q})}{\partial q_{i}}\right) \\
   & =\frac{1}{2}\left(f_{i}(\vec{q}) p_{i}+p_{i} f_{i}(\vec{q})\right) \equiv \frac{1}{2}\left\{f_{i}(\vec{q}), p_{i}\right\}
    \end{aligned}
\end{equation}
           
1/2 в формуле (\ref{12.17}) гарантирует, что произведение $\psi^\dagger\psi$ эволюционирует с правильным якобианом. 

Отметим, что этот гамильтониан является эрмитовым, и эволюционное уравнение (\ref{12.11}) следует непосредственно из правил коммутации

\begin{equation}\label{12.19}
	\left[q_{i}, p_{j}\right]=i \delta_{i j} ; \quad \frac{\mathrm{d}}{\mathrm{d} t} \mathcal{O}_{\mathrm{op}}(t)=-i\left[\mathcal{O}_{\mathrm{op}}, H_{\mathrm{op}}\right]
\end{equation}

Однако теперь мы сталкиваемся с очень важной трудностью: этот гамильтониан не имеет нижней границы. Поэтому его нельзя использовать для стабилизации волновых функций. Без нижней границы нельзя заниматься термодинамикой. Эта особенность превратит нашу модель в нечто очень непохожее на квантовую механику, какой мы ее знаем.

Если мы возьмем пространство $\vec q$ либо одномерным, либо в некоторых случаях двумерным, мы можем сделать нашу систему периодической. Тогда пусть $T$ будет наименьшим положительным числом таким, чтобы

\begin{equation}\label{12.20}
	\vec q(t) = \vec q(0)
\end{equation}

Следовательно, мы имеем

\begin{equation}\label{12.21}
	e^{-i H T}|\vec{q}(0)\rangle=|\vec{q}(0)\rangle
\end{equation}
           
и, следовательно, в этих состояниях,

\begin{equation}\label{12.22}
H_{\mathrm{op}}\mid \vec q\rangle = \sum_{n=-\infty}^{\infty} \frac{2 \pi n}{T}\mid n\rangle{_H} {_H} \langle n \mid \vec{q}\rangle
\end{equation}

Таким образом, спектр собственных значений энергии собственных состояний $\mid n\rangle_H$ пробегает все целые числа от $-\infty$ до $\infty$.

В дискретном случае гамильтониан имеет конечное число собственных состояний с собственными значениями $2\pi k / (N\delta t) + \delta E$, где $k = 0,\ldots,N - 1$, что означает, что они лежат в интервале $[\delta E, 2\pi/T + \delta E]$, где $T$ - период, и $\delta E$ можно выбрать произвольным. Таким образом, здесь у нас всегда есть нижняя граница, и состояние с этой энергией можно назвать «основным состоянием» или «вакуумом».

В зависимости от того, как взят предел континуума, мы можем или не можем сохранить эту нижнюю границу. Нижняя граница энергии кажется искусственной, потому что все энергетические состояния выглядят совершенно одинаково. Именно здесь может быть более полезна формулировка $SU$(2) для гармонических ротаторов, рассматриваемая в разд. \ref{ch12.1}.

В качестве другого средства против этой проблемы можно потребовать аналитичности, при этом выбрав время комплексным, и ограничив волновые функции в нижней половине комплексного временного периода. Это исключило бы состояния с отрицательной энергией и все же позволило бы нам представить все распределения вероятностей с помощью волновых функций. Эквивалентно, можно рассматривать комплексные значения для переменной (ых) $\vec q$ и требовать отсутствия сингулярностей в комплексной плоскости ниже действительной оси. Такие ограничения аналитичности, однако, кажутся довольно произвольными; их трудно поддерживать, как только взаимодействие будет введено, поэтому к ним, безусловно, следует относиться с осторожностью.

Одним из очень многообещающих подходов к решению проблемы основного состояния является отличная идея Дирака о вторичном квантовании: взять неограниченное число объектов q, то есть гильбертово пространство, охватываемое всеми состояниями $| \vec q^{(1)}, \vec q^{(2)},\ldots, q^{(n)}\rangle$, для всех чисел частиц $n$ и рассматривать конфигурации с отрицательной энергией как «дырки» античастиц. Это мы предлагаем сделать в нашей «нейтринной модели», разд. \ref{ch15.2} и в последующих главах.

В качестве альтернативы, мы могли бы рассмотреть предел континуума дискретной теории более тщательно. Это мы попробуем в начале следующего раздела. Подчеркнем еще раз: в общем, исключение таких состояний с отрицательной энергией не всегда является хорошей идеей, потому что любое возмущение системы может вызвать переходы в эти состояния с отрицательной энергией, а их пропуск может нарушить унитарность.

Важность основного состояния гамильтониана обсуждалась в разд. \ref{ch9} части I. Гамильтониан (\ref{12.18}) является важным выражением для фундаментальных дискуссий по квантовой механике.

Как и в дискретном, так и в случае детерминированных моделей с непрерывным законом эволюции, можно найти дискретные и непрерывные собственные значения в зависимости от того, является ли система периодической. В пределе $\delta t \rightarrow 0$ дискретной периодической онтологической модели собственные значения представляют собой целые числа, кратные $2\pi / T$, и это также спектр гармонического осциллятора с периодом $T$, как объясняется в гл. \ref{ch13}. Гармонический осциллятор может рассматриваться как замаскированная детерминированная система.

Более общая непрерывная модель - это система, полученная, во-первых, с помощью (конечного или бесконечного) числа гармонических осцилляторов, что означает, что наша система состоит из множества периодических подструктур, и, во-вторых, путем принятия (конечного или бесконечного) числа сохраняющихся величин, от которых зависят периоды осцилляторов. Примером является область невзаимодействующих частиц; Квантовая теория поля соответствует наличию бесконечного числа колебательных мод этого поля. Частицы могут быть фермионными или бозонными; фермионный случай также является набором осцилляторов, если фермионы помещаются в коробку с периодическими граничными условиями. Взаимодействующие квантовые частицы будут встречаться позже (глава \ref{ch19} и далее).


\end{document}